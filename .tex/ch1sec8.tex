\documentclass[a4paper,12pt]{article}
\usepackage{mystyle}
\usepackage{commands}
\mathtoolsset{showonlyrefs=true}

\begin{document}
\section{The Principle of Recursive Definition}
\setcounter{exe}{0}
\begin{exe}
	Let \( (b_1,b_2,\cdots) \) be an infinite sequence of real numbers.
	The sum \( \sum_{k=1}^n b_k\) is defined by induction as follows:
	\begin{eqnarray*}
		\sum_{k=1}^1 b_k &=& b_1,\\
		\sum_{k=1}^n b_k &=& \left( \sum_{k=1}^{n-1} b_k \right) + b_n
	\end{eqnarray*}
	for \( n >1 \).
	Let \( A \) be the set of real numbers; choose \( \rho \) so that Theorem 8.4 applies to define the sum rigorously.
\end{exe}
\begin{sol}
	For
	\( f \in \func{\left\{1, \cdots, m \right\}}{\mathbb{R}} \),
	define
	\( \rho(f):=f(m) + b_{m+1} \).
	Let \( b_1 \) be the initial value for \( h \).
	\qed\end{sol}

\begin{exe}
	Let \( (b_1,b_2,\cdots) \) be an infinite sequence of real numbers.
	We define the product \( \prod_{k=1}^{n} b_k \) by the equations
	\begin{eqnarray*}
		\prod_{k=1}^1 b_k &=& b_1,\\
		\prod_{k=1}^n b_k &=& \left( \prod_{k=1}^{n-1} b_k \right) \cdot b_n
	\end{eqnarray*}
	for \( n>1 \).
	Use Theorem 8.4 to define this product rigorously.
\end{exe}
\begin{sol}
	For
	\( f \in \func{\left\{1, \cdots, m \right\}}{\mathbb{R}} \),
	define
	\( \rho(f):=f(m) \cdot b_{m+1} \).
	Let \( b_1 \) be the initial value for \( h \).
	Theorem 8.4 gives
	\( h\in \func{\mathbb{Z}_{+}}{\mathbb{R}} \)
	such that
	\begin{eqnarray*}
		h(1)&=&b_1,\\
		h(n)&=&\prod_{k=1}^n b_k
	\end{eqnarray*}
	for
	\( n \ge 2 \).
	\qed\end{sol}

\begin{exe}
	Obtain the definitions of \( a^n \) and \( n! \) for \( n \in \mathbb{Z}_{+} \)
	as special cases of Exercise 2. 
\end{exe}
\begin{sol}
	For
	\( a^n \),
	assume \( b_n=a \)
	for all
	\( n \in \mathbb{Z}_{+} \).
	For
	\( n! \),
	assume \( b_n=n \)
	for all
	\( n \in \mathbb{Z}_{+} \).
	\qed\end{sol}

\begin{exe}
	The \textit{Fibonacci numbers} of number theory are defined recursively by the formula
	\begin{eqnarray*}
		\lambda_1 &=& \lambda_2 = 1,\\
		\lambda_{n}&=& \lambda_{n-1} + \lambda_{n-2}
	\end{eqnarray*}
	for \( n>2 \).
	Define them rigorously by use of Theorem 8.4.
\end{exe}
\begin{sol}%4
	For
	\( f \in \func{\left\{1, \cdots, m \right\}}{\mathbb{R}} \),
	define
	\begin{equation*}
		\rho(f):=	\begin{cases}
			1           & \;\mathrm{:}\;f\in \func{S_2}{\mathbb{R}}             \\
			f(m)+f(m-1) & \;\mathrm{:}\;f\in \func{S_m}{\mathbb{R}}\wedge m >2.
		\end{cases}
	\end{equation*}
	Let \( a_0:=1 \) be the initial value for \( h \).
	\qed\end{sol}

\begin{exe}
	Show that there is a unique function
	\( h:\mathbb{Z}_{+} \to \mathbb{R} \)
	satisfying the formula
	\begin{eqnarray*}
		h(1)&=&3,\\
		h(i)&=& \left[ h(i-1)+1 \right]^{1/2}
	\end{eqnarray*}
	for \( i>1 \).
\end{exe}
\begin{sol}
	We only have to construct a good function
	\( \rho \)
	since theorem 8.4 then gives the required function
	\( h \).
	To this end, for
	\( f \in \func{\left\{1, \cdots, m \right\}}{\mathbb{R}} \),
	set
	\( \rho(f):=\left( f(m) + 1 \right)^{1/2} \).
	\qed\end{sol}

\begin{exe}\leavevmode \par
	\begin{enumerate}
		\item
		      Show that there is no function \( h:\mathbb{Z}_{+} \to \mathbb{R} \)
		      satisfying the formula
		      \begin{eqnarray*}
			      h(1)&=&3,\\
			      h(i)&=& \left[ h(i-1)-1 \right]^{1/2}
		      \end{eqnarray*}
		      Explain why this example does not violate the principle of recursive definition.
		      
		\item
		      Consider the recursive formula
		      \begin{eqnarray*}
			      h(1)&=&3,\\
			      h(i)&=& \begin{cases}
				      \left[ h(i-1)-1 \right]^{1/2} & \;\mathrm{:}\; h(i-1)>1    \\
				      5                             & \;\mathrm{:}\; h(i-1)\le 1
			      \end{cases}
		      \end{eqnarray*}
		      for \( i>1 \).
	\end{enumerate}
	Show that there exists a unique function
	\( h:\mathbb{Z}_{+} \to \mathbb{R} \)
	satisfying this formula.
\end{exe}
\begin{sol}\leavevmode \par
	\fbox{(a)}
	The given formula requires
	\( h(1)=3\),
	\( h(2)=\sqrt{2} \),
	\( h(3)=\left( \sqrt{2} - 1 \right)^{1/2} \),
	but
	\( \left( \left( \sqrt{2} - 1 \right)^{1/2} - 1 \right)^{1/2}  \)
	does not exist in \( \mathbb{R} \),
	which implies
	\( h(4) \)
	cannot be defined.
	
	Note that the corresponding \( \rho \) should be given as
	\begin{equation*}
		\rho(f):=\left( f(m)-1 \right)^{1/2}
	\end{equation*}
	for
	\( f \in \func{\left\{1, \cdots, m \right\}}{\mathbb{R}} \),
	but
	\( \rho \)
	is not a function from
	\( \bigcup_{m\in \mathbb{Z}_{+}}\func{\left\{1, \cdots, m \right\}}{\mathbb{R}} \)
	to
	''\( \mathbb{R}_{+} \).''
	
	This trouble is because of the absence of a good function \( \rho \),
	not a violation of the principle of recursive definition,
	which gives a requested function
	\( h \)
	if the hypothesis, for instance, existence of 
	\( \rho \),
	is satisfied, and says nothing if otherwise.
	In fact, the principle of recursive definition can be expressed as
	\begin{eqnarray*}
		&&a_0\in A
		\wedge
		\rho \in \func{\bigcup_{m\in \mathbb{Z}_{+}}\func{\left\{1, \cdots, m \right\}}{A}}{A}\\
		&\Rightarrow&
		\unique{ h \in \func{\mathbb{Z}_{+}}{A} }
		\left[
			h(1)=a_0
			\wedge
			\Forall{ i \in \mathbb{Z}_{+} \setminus \left\{ 1 \right\} }
			\left[
				h(i) = \rho \left( h|_{\left\{ 1,\cdots,i-1 \right\}} \right)
				\right]
			\right]
	\end{eqnarray*}
	and the whole statement is still true even if \( \rho \) is absent and the hypothesis is false.
	(Remember that statement \( A \Rightarrow B \) is true if \( A \) is false.)
	
	\fbox{(b)}
	For
	\( f \in \func{\left\{1, \cdots, m \right\}}{\mathbb{R}} \),
	define
	\begin{equation*}
		\rho(f):=\begin{cases}
			\left( f(m) - 1 \right)^{1/2} & \;\mathrm{:}\; f(m)>1     \\
			5                             & \;\mathrm{:}\; f(m)\le 1.
		\end{cases}
	\end{equation*}
	Then,
	\( \rho(f)>0 \)
	for all
	\( f \).
	The result follows from Theorem 8.4.
	\qed\end{sol}

\begin{lem}\label{lem:recursive}
	Let
	\( A \)
	be a set, and let
	\( a_0 \in A \).
	Suppose \( \rho \) is a function from
	\( \bigcup_{m\in \mathbb{Z}_{+}}\func{\left\{1, \cdots, m \right\}}{A} \)
	to
	\( A \).
	Then for all \( n \in \mathbb{Z}_{+} \)
	there exists a unique function
	\( h_n: \left\{1, \cdots, n \right\} \to A\)
	such that
	\begin{equation}\label{eq:recursive}
		\begin{split}
			h_n(1)&=a_0\\
			h_n(i)&=\rho \left( h_n |_{\left\{1, \cdots, i-1 \right\}} \right)
		\end{split}
	\end{equation}
	for all
	\( 1<i\le n \).
\end{lem}

\begin{proof}
	Let \( X \) be the set of all
	\( n \in \mathbb{Z}_{+} \)
	for which the lemma holds.
	Suppose \( \left\{1, \cdots, m \right\} \in X\),
	and let \( h_{n-1} \) satisfy (\ref{eq:recursive})
	for all \( i \) in its domain.
	Since
	\( h_{n-1} \in \func{\left\{1, \cdots, n-1 \right\}}{A} \),
	we can define a function
	\( h_n:\left\{1, \cdots, n \right\} \to A \)
	via
	\begin{equation*}
		h_n(i):=\begin{cases}
			h_{n-1}(i)                                                 & \;\mathrm{:}\; i \in \left\{ 1,\cdots,n-1 \right\} \\
			\rho \left( h_n |_{\left\{1, \cdots, i-1 \right\}} \right) & \;\mathrm{:}\; i=n
		\end{cases}
	\end{equation*}
	It is clear that \( h_n \) satisfies (\ref{eq:recursive}).
	
	To show the uniqueness of \( h_n \),
	suppose
	\( g: \left\{1, \cdots, n \right\} \to A\)
	also satisfies (\ref{eq:recursive}) for all \( i \) in its domain.
	Let \( Y \) be a subset of \( \mathbb{Z}_{+} \) consisting of all \( m \) such that we have
	\( g(i)=h_n(i) \)
	for all
	\( i \in \left\{1, \cdots, n \right\}\cap \left\{1, \cdots, m \right\} \).
	Suppose
	\( \left\{1, \cdots, k-1 \right\} \in Y \).
	We may assume
	\( k-1 \le n \).
	We have
	\( g|_{\left\{1, \cdots, k-1 \right\}} = h_n|_{\left\{1, \cdots, k-1 \right\}} \)
	by assumption, and hence
	\begin{equation*}
		g(k)
		= \rho \left( g|_{\left\{1, \cdots, k-1 \right\}} \right)
		= \rho \left( h_n|_{\left\{1, \cdots, k-1 \right\}} \right)
		= h_n(k),
	\end{equation*}
	which yields \( k \in Y \).
	Thus, strong induction principle gives \( Y = \mathbb{Z}_{+} \),
	establishing the uniqueness of \(  h_n \).
	We the conclude that \( n \in X \) and that \( X = \mathbb{Z}_{+} \).
\end{proof}

\begin{exe}
	Prove Theorem 8.4.
\end{exe}
\begin{sol}
	Define a function \( h:\mathbb{Z}_{+} \to A \)
	by setting its rule to be the union of the function \( h_n \),
	we have constructed in \refer{Lemma}{lem:recursive}.
	\( h \)
	is proved to be a function by exactly the same argument as Theorem 8.3
	since there we have exploited nothing of the property of the range of \( f_n \).
	
	We show that \( h \) satisfies
	\begin{equation}\label{eq:recursive2}
		\begin{split}
			h(1)&=a_0\\
			h(i)&=\rho \left( h |_{\left\{1, \cdots, i-1 \right\}} \right)
		\end{split}
	\end{equation}
	for all \( i >1 \).
	Let \( W \) be the set of all \( i \in \mathbb{Z}_{+} \) for which
	(\refeq{eq:recursive2}) is true.
	It is easy to see \( W= \mathbb{Z}_{+} \) by induction.
	
	To prove the uniqueness of \( h \),
	argue as we have done in \refer{Lemma}{lem:recursive}.
	\qed\end{sol}

\begin{exe}
	Verify the following version of the principle of recursive definition:
	Let \( A \) be a set.
	Let \( \rho \) be a function assigning,
	to every function \( f \) mapping a section \( S_n \) of \( \mathbb{Z}_{+} \) into \( A \),
	a element \( \rho(f) \) of \( A \).
	Then there is a unique function \( h:\mathbb{Z}_{+} \to A \) such that
	\( h(n)=\rho(h|_{S_n}) \)
	for each \( n \in \mathbb{Z}_{+} \).
\end{exe}
\begin{sol}
	Note that \( S_1 = \emptyset \), and that \( h|_{S_1} \) is well-defined as a unique empty function, denoted by \( \emptyset \).(see \refer{NOTE}{}).
	Set \( a_0:=\rho (\emptyset) \)
	and apply Theorem 8.4.
	\qed\end{sol}

\end{document}