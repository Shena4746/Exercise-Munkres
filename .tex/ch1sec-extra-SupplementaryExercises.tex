\documentclass[a4paper,12pt]{article}
\usepackage{mystyle}
\usepackage{commands}
\mathtoolsset{showonlyrefs=true}

\begin{document}
\section*{Supplementary Exercises:\;Well-Ordering}
\setcounter{exe}{0}
\begin{exe}[General principle of recursive definition]
	Let \( J \) be a well-ordered set;
	let \( C \) be a set.
	Let \( \mathcal{F} \) be the set of all functions mapping sections of \( J \) into \( C \).
	Given a function \( \rho : \mathcal{F} \to C\),
	there exists a unique function \( h:J \to C \)
	such that
	\( h(\alpha) = \rho(h | S_{\alpha}) \)
	for each \( \alpha \in J \).
\end{exe}
\begin{sol}
	Copy the argument for \S10 Exercise 10.
	\qed\end{sol}

\begin{rem}[Uniqueness of an order-preserving function]\label{note:unique_order_preserving}
	We establish here an useful result,
	which is a consequence of general principle of recursive definition (Exercise 1) and 
	Exercise 2(a), as follows:\leavevmode \par
	\textit{There exists at most one order-preserving function from a well-ordered set to another, whose image is the range of the function or a section of it.}
	
	For, if we suppose that there are two such functions, then Exercise 2(a) tells us that they both satisfy the formula indicated in Exercise 2(a)(ii),
	recursive formulas for themselves.
	Hence, general principle of recursive definition implies that 
	these two function equal to one another.
\end{rem}

\begin{exe}\leavevmode \par
	\begin{enumerate}
		\item
		      Let \( J \) and \( E \) be well-ordered sets;
		      let \( h: J \to E \).
		      Show the following two statements are equivalent:
		      \begin{enumerate}
			      \item
			            \( h \) is order preserving and its image is \( E \) or a section of \( E \).
			            
			      \item
			            \( h(\alpha) = \min{\left[ E \setminus h(S_{\alpha}) \right]} \)
			            for all \( \alpha \).
		      \end{enumerate}
		\item
		      If \( E \) is a well-ordered set,
		      show that no section of \( E \) has the order type of \( E \),
		      nor do two different sections of \( E \) have the same order type.
	\end{enumerate}
\end{exe}
\begin{sol}\leavevmode \par
	\fbox{(a)}
	We first show that (i) implies (ii).
	So, suppose (i) holds.
	Without loss of generality, we may assume that \( h:J\to E \) is an order isomorphism.
	Then it follows that
	\( h(S_{\alpha}) = S'_{h(\alpha)} \) for all \( \alpha \in J \),
	where \( S'_{y} \) denotes the section of \( E \) by \( y \),
	and that
	\begin{equation*}
		\min{\left[ E \setminus h(S_{\alpha}) \right]}
		=
		\min{\left[ E \setminus S'_{h(\alpha)} \right]}
		=
		\sup{S'_{h(\alpha)}}
	\end{equation*}
	for all \( \alpha \in J \),
	where we note that \( E \setminus h(S_{\alpha}) \)
	is nonempty by the assumption of (i).
	Now the conclusion follows from the obvious fact 
	\begin{equation*}
		h(\alpha) = \sup{S'_{h(\alpha)}}
	\end{equation*}
	for all \( \alpha \in J \).
	
	To prove the converse,
	suppose (ii) holds.
	It follows from (ii) that
	\( E \setminus h(S_{\alpha}) \)
	is the set of all upper bounds for \( h(\alpha) \)
	for every \( \alpha \in J \),
	from which we deduce that
	\begin{equation}\label{eq:preserve_section}
		S'_{h(\alpha)} = h(S_{\alpha})
	\end{equation}
	for every \( \alpha \in J \).
	Moreover, \refeq{eq:preserve_section} lets us see \( h \) preserve order.
	Indeed, if \( \alpha < \beta \), or equivalently, if \( \alpha \in S_{\beta} \),
	then \( h(\alpha) \in h(S_{\beta}) = S'_{h(\beta)}\),
	that is,
	\( h(\alpha) < h(\beta) \).
	
	Assume \( h(J) \subsetneq E \).
	We explicitly construct a section of \( E \) that equals \( h(J) \).
	
	If \( J \) has a largest element, denoted by \( m \),
	then we have
	\begin{equation*}
		h(J)
		=
		h(S_m) \cup \left\{ h(m) \right\}
		=
		S'_{h(m)}\cup \left\{ h(m) \right\}
		\subsetneq
		E.
	\end{equation*}
	This means \( h(m) \) is not a largest element of \( E \),
	and hence, \( h(m) \) admits an immediate successor \( n \) in \( E \)
	by \S10 Exercise 2.
	Then, \( h(J)=S'_n \).
	
	If not, neither does \( h(J) \).
	Note that we can write \( J=\bigcup_{\alpha \in J}S_{\alpha} \),
	and
	it follows from the fact
	\begin{equation*}
		h(J)
		=
		\bigcup_{\alpha \in J}h(S_{\alpha})
		=
		\bigcup_{\alpha \in J}S'_{h(\alpha)}
		\subsetneq
		E
	\end{equation*}
	that \( h(J) \) is bounded above.
	Then \S10 Exercise 1 tells us that \( h(J) \) has a supremum \( s \) in \( E \), more precisely, in \( E \setminus h(J) \).
	It is straightforward to deduce from the property of supremum that
	\( h(J) = S'_s\).
	
	\fbox{(b)}
	Let \( \alpha \in E \).
	It is obvious that the inclusion function
	\begin{equation*}
		i:S_{\alpha} \ni x \mapsto x \in E
	\end{equation*}
	preserves order, and its image \( i(S_{\alpha}) \) equals \( S_{\alpha} \),
	a section of \( E \),
	and that \( i \) is not surjective. 
	We deduce from \refer{Note}{note:unique_order_preserving} that no order preserving function from
	\( S_{\alpha} \) to \( E \) could be surjective, nor could be an order isomorphism.
	Thus, no section of \( E \) has the order type of \( E \).
	
	The second implication is a direct consequence of the first;
	Replace \( E \) with a section of \( E \) and use the innocent fact that
	\textit{a section of a section of \( E \) is a section of \( E \).}
	\qed\end{sol}

\begin{exe}
	Let \( J \) and \( E \) be well-ordered sets;
	suppose there is an order preserving map
	\( K: J\to E \).
	Using Exercises 1 and 2,
	show that \( J \) has the order type of \( E \) or a section of \( E \).
\end{exe}
\begin{sol}
	Choose \( e_0:=\min{E} \).
	Define a function \( h:J\to E\) by the recursive formula
	\begin{equation*}
		h(\alpha):=\begin{cases}
			\min{\left[ E \setminus h(S_{\alpha}) \right]}
			    & \;\mathrm{:}\;h(S_{\alpha}) \neq E \\
			e_0 & \;\mathrm{:}\;h(S_{\alpha}) = E.
		\end{cases}
	\end{equation*}
	We show by transfinite induction that for every \( \alpha \in J \)
	\begin{equation*}
		h(\alpha) \le k(\alpha)
	\end{equation*}
	holds.
	Let \( X \) be a subset of \( J \) consisting of all \( \alpha \) for which the above inequality holds.
	Let \( \alpha \in J \) and suppose \( S_{\alpha} \subset X \).
	We may assume \( h(S_{\alpha}) \neq E \)
	thanks to the choice of \( e_0 \),
	and so \( h(\alpha) = \min{\left[ E \setminus h(S_{\alpha}) \right]} \).
	On the other hand, it follows that \( k(\alpha) \) is greater than every element of
	\( k(S_{\alpha}) \) and \( h(S_{\alpha}) \) by assumption,
	which leads to \( k({\alpha}) \in E \setminus h(S_{\alpha}) \).
	Hence, \( h(\alpha) \le k(\alpha) \),
	completing the induction.
	
	It follows from the just proved inequality that
	\( h(S_{\alpha}) \subset S'_{k(\alpha)} \),
	and hence
	\( h(S_{\alpha}) \neq E \)
	for all \( \alpha \in J \),
	where
	\( S'_{k(\alpha)} \) denotes the section of \( E \) by \( k(\alpha) \).
	So, \( h \) satisfies 
	\begin{equation*}
		h(\alpha) = \min{\left[ E \setminus h(S_{\alpha}) \right]}
	\end{equation*}
	for all \( \alpha \in J \).
	Then Exercise 2(a) tells us that \( h \) is order preserving and its image is \( E \)
	or a section of \( E \),
	from which we conclude the result.
	\qed\end{sol}

\begin{exe}
	Use Exercise 1-3 to prove the following: 
	\begin{enumerate}
		\item
		      If \( A \) and \( B \) are well-ordered sets, then exactly one of the following three conditions holds:
		      \( A \) and \( B \) have the same order type, or \( A \) has the order type of a section of \( B \),
		      or \( B \) has the order type of a section of \( A \).
		      
		\item
		      Suppose that \( A \) and \( B \) are well-ordered sets that are uncountable,
		      such that every section of \( A \) and of \( B \) is countable.
		      Show \( A \) and \( B \) have the same order type.
	\end{enumerate}
\end{exe}
\begin{sol}\leavevmode \par
	\fbox{(a)}
	We first show that at least one of the three conditions is the case.
	Let \( (A,<_{A}) \) and \( (B,<_{B}) \) be well-ordered sets,
	and let \( (A\cup B,<) \) be the well-ordered set constructed in the way
	indicated in \S10 Exercise 8.
	Let \( S_{x}(A), S_{x}(B)\), and \(S_{x} \)
	denote the section of \( A, B\), and \( A\cup B \) by \( x \), respectively.
	We may assume \( A \neq \emptyset \) and \( B \neq \emptyset \).
	Consider an inclusion function \( \iota \) given by
	\begin{equation*}
		\iota : A \ni x \mapsto x \in A\cup B.
	\end{equation*}
	It is obvious that \( \iota \) preserves order and its image is \( A \),
	a section of \( A\cup B \) by a smallest element
	\( b_0 \) of \( B \) in \( (A\cup B,<) \).
	Hence, Exercise 3 tells us that we have either \( A\simeq A \cup B \) or
	\begin{equation}\label{secsupE4eq1}
		\unique{ y_A \in A\cup B}
		\left[ A \simeq S_{y_A} \right].
	\end{equation}
	But there cannot hold \( A\simeq A \cup B \) since \( \iota \) is not a surjection, neither is an order isomorphism. 
	%(Note \( y_A \notin A \) by Exercise 2.) 
	Using Exercise 3, we similarly deduce that
	\begin{equation}\label{secsupE4eq2}
		B \simeq A \cup B
		\vee
		\unique{ y_B \in A \cup B }
		\left[ B \simeq S_{y_B} \right]
	\end{equation}
	but not both.
	Note that we cannot reject \( B \simeq A \cup B \) this time,
	from which we deduce that
	\begin{equation}\label{secsupE4eq3}
		S_{f^{-1}(b_0)}(B) \simeq A,
	\end{equation}
	where \( f \) is an order isomorphism from \( B \) to \( A\cup B \).
	
	Suppose \( y_A \neq y_B \) and \( y_B \in B \).
	If \( y_B = b_0 \),
	then we deduce from \refeq{secsupE4eq2} that
	\begin{equation}\label{secsupE4eq4}
		B \simeq A.
	\end{equation}
	If not, we have
	\begin{equation}\label{secsupE4eq5}
		S_{g^{-1}(b_0)}(B) \simeq A
	\end{equation}
	where \( g \) is an order isomorphism from \( B \) to \( S_{y_B} \).
	If \( y_A \neq y_B \) and \( y_B \in A \),
	then we have
	\begin{equation}\label{secsupE4eq6}
		B \simeq S_{y_B}(A).
	\end{equation}
	At last, if \( y_A = y_B \) happens to be the case,
	then (\refeq{secsupE4eq1}),(\refeq{secsupE4eq2}) yield
	\begin{equation}\label{secsupE4eq7}
		A \simeq B.
	\end{equation}
	Thus, at least one of the three does hold.
	Note that every term in \refeq{secsupE4eq3} - \refeq{secsupE4eq7}
	is a section of \( (A\cup B,<) \).
	Hence, Exercise 2(b) applied to \( (A\cup B,<) \) establishes that
	none of two holds at the same time.
	Thus, we have proved that exactly one of the three holds.
	
	\fbox{(b)}
	Definition of uncountable set requires the absence of bijective correspondence of  \( A \) and \( B \) with a countable set.
	Hence, Neither \( A \) nor \( B \) have the order type of a section of the other. 
	Thus, we conclude from (a) that \( A \simeq B\).
	\qed\end{sol}


\begin{exe}
	Let \( X \) be a set;
	let \( \mathcal{A} \) be a collection of all pairs \( (A,<) \),
	where \( A \) is a subset of \( X \) and \( < \) is a well-ordering of \( A \).
	Define
	\begin{equation*}
		(A,<) \prec (A',<')
	\end{equation*}
	if \( (A,<) \) \textit{equals} a section of \( (A',<') \).
	\begin{enumerate}
		\item
		      Show that \( \prec \) is a strict partial order on \( \mathcal{A} \).
		      
		\item
		      Let \( \mathcal{B} \) be a subcollection of \( \mathcal{A} \) that is simply ordered by \( \prec \).
		      Define \( B' \) to be the union of the sets \( B \),
		      for all \( (B,<)\in \mathcal{B} \);
		      and define \( <' \) to be the union of the relations \( < \),
		      for all \( (B,<)\in \mathcal{B} \).
		      Show that \( (B',<') \) is a well-ordered set.
	\end{enumerate}
\end{exe}
\begin{sol}\leavevmode \par
	\fbox{(a)}
	Nonreflexivity follows from Exercise 2(b).
	To see transitivity,
	suppose \( (A,<_{A}) \prec (B,<_{B})\) 
	and 
	\( (B,<_{B})  \prec (C,<_{C})\).
	By definition of \( \prec \),
	there exist \( b\in B \) and \(  c\in C \) such that
	\begin{equation*}
		A \simeq S_{b}(B),\;\;B \simeq S_{c}(C).
	\end{equation*}
	It is easy by considering an order isomorphism to see
	\begin{equation*}
		S_{b}(B) \simeq S_{c'}(C)
	\end{equation*}
	for some \( c' \in C \).
	Then we have \( A \simeq S_{c'}(C)\),
	from which we conclude \( (A,<_{A}) \prec (C,<_{C}) \),
	as desired.
	
	\fbox{(b)}
	We first establish that \( <' \) is a simple order on \( B' \).
	Nonreflexivity is obviously satisfied.
	To see \( <' \) satisfy comparability,
	let \( b_1 \) and \( b_2 \) be two distinct points in \( B' \).
	There exist \( (B_1,<_1), (B_2,<_2) \in \mathcal{B} \)
	with \( (B_1,<_1) \prec  (B_2,<_2) \)
	such that \( b_1 \in B_1 \) and \( b_2 \in B_2 \).
	Then \( b_1, b_2 \in B_2 \) and these points are comparable under \( <_2 \),
	and so are under \( <' \).
	To prove transitivity,
	suppose that \( b_1 <' b_2 \) and \( b_2 <' b_3 \).
	We deduce, as we do to see comparability, that
	there exists \( (B,<) \in \mathcal{B} \) such that
	\( b_i \in B \)
	and
	\( b_{i} < b_{i+1} \)
	for all \( i \).
	Then transitivity of \( (B,<) \) gives \( b_1 < b_3 \).
	Hence, \( b_1 <' b_3 \).
	Thus, \( <' \) is a simple order on \( B' \).
	
	We show that \( (B',<') \) is a well-ordered set.
	Let \( W \) be a nonempty subset of \( B' \).
	The fact \( \mathcal{B} \) is simply ordered by \( \prec \)
	gives \( (B,<) \in \mathcal{B} \) with \( W \subset B \).
	Then \( W \) has a smallest element \( m \) in \( (B,<) \),
	for which we have
	\( m<b \)
	and hence
	\( m<'b \)
	for all \( b\in W \setminus \left\{ m \right\} \).
	Thus, \( W \) has a smallest element \( m \) in \( (B',<') \).
	\qed\end{sol}

\begin{exe}
	Use Exercise 1 and 5 to prove the following:\leavevmode \par
	\noindent \text{Theorem.}\;\;
	The maximum principle is equivalent to the well-ordering theorem.
\end{exe}
\begin{sol}
	We have already seen the well-ordering theorem imply the (Hausdorff) maximum principle.
	Here we establish the converse.
	
	Let \( X \) be a set,
	and let \( \mathcal{A} \) and \( \prec \) be as at Exercise 5,
	say, let \( \mathcal{A} \) be the collection of all pairs \( (A,<) \),
	where \( A \) is a subset of \( X \)
	and \( < \) is a well-ordering of \( A \);
	let \( (A,<) \prec (A',<') \)
	if \( (A,<) \) equals a section of \( (A',<') \).
	Note that \( \mathcal{A} \) is nonempty since \( (\emptyset,\emptyset) \) is vacuously well-ordered,
	and that if \( (A,<) \in \mathcal{A} \) and \( x\in X \setminus A \),
	then we can define a new well-ordered set
	\( (A\cup \left\{ x \right\}, <_{+x})\in \mathcal{A} \),
	where \( <_{+x} \) is defined to be equal to \( < \) on \( A \)
	and \( a <_{+x} x \) for all \( a\in A \).
	Observe that Exercise 5 implies that
	\( \mathcal{A} \) satisfies the hypothesis of Zorn's lemma,
	which, as we have seen in \S11, is equivalent to the maximum principle.
	Hence, \( \mathcal{A} \) has a maximal element.
	It is easy to verify that \( A^{\ast}= X \).
	\qed\end{sol}

\begin{exe}
	Use Exercise 1-5 to prove the following:\leavevmode \par
	\noindent \text{Theorem.}\;\;
	The choice axiom is equivalent to the well-ordering theorem.\\
	\textit{Proof.}\;
	Let \( X \) be a set;
	let \( c \) be a fixed choice function for the nonempty subsets of \( X \).
	If \( T \) is a subset of \( X \) and \( < \) is a relation on \( T \),
	we say that \( (T,<) \) is a \textbf{\textit{tower}}
	in \( X \) if \( < \) is a well-ordering of \( T \) and if for each \(  x \in T \),
	\begin{equation}\label{eq:def_tower}
		x=c \left( X \setminus S_x(T) \right),
	\end{equation}
	where \( S_x(T) \) is the section of \( T \) by \( x \).
	\begin{enumerate}
		\item
		      Let \( (T_1,<_1) \) and \( (T_2,<_2) \) be two towers in \( X \).
		      Show that either two ordered sets are the same,
		      or one equals a section of the other.
		      
		\item
		      If \( (T,<) \) is a tower in \( X \) and \( T \neq X \),
		      show there is a tower in \( X \) of which \( (T,<) \) is a section.
		      
		\item
		      Let \( \left\{ (T_k,<_k) \pipe  k \in K \right\} \)
		      be the collection of all tower in \( X \).
		      Let
		      \begin{equation*}
			      T:=\bigcup_{k \in K} T_k,\;\;<:=\bigcup_{k \in K}\left( <_k \right).
		      \end{equation*}
		      Show that \( (T,<) \) is a tower in \( X \).
		      Conclude that \( T=X \).
	\end{enumerate}
\end{exe}
\begin{sol}\leavevmode \par
	\fbox{(a)}
	By Exercise 5,
	we may assume that there exists a function \( h:T_1 \to T_2 \)
	that is order-preserving and \( h(T_1) \) equals
	either \( T_2 \) or a section of \( T_2 \).
	Exercise 2 tells us that \( h \) satisfies
	\begin{equation*}
		h(x) = \min{\left[ X \setminus h(S_x(T_1)) \right]}
	\end{equation*}
	and
	\begin{equation}\label{eq:secsuppE7}
		h \left( S_{x}(T_1) \right) = S_{h(x)}(T_2) 
	\end{equation}
	for all \(  x \in T_1 \).
	We show by transfinite induction that \( h \) is an identity function on \( T_1 \),
	that is,
	\begin{equation*}
		h(x) = x
	\end{equation*}
	for all \(  x \in T_1 \).
	Let \( I \) be a subset of \( T_1 \) consisting of all \( x \) for which
	the above equality holds.
	Let \(  x \in T_1 \) and suppose \( S_x(T_1) \subset I\).
	We then deduce from (\refeq{eq:secsuppE7}) that
	\begin{equation*}
		h(x)
		=
		c(X \setminus S_{h(x)}(T_2))
		=
		c(X \setminus h(S_{x}(T_1)))
		=
		c(X \setminus S_{x}(T_1))
		=x,
	\end{equation*}
	and hence \(  x \in I \),
	completing the induction.
	Thus, \( T_1 \) coincides with its image under \( h \),
	namely, with either \( T_2 \) or a section of \( T_2 \).
	
	\fbox{(b)}
	Let \( p:=c(X \setminus T) \).
	As in Exercise 6,
	we can define a new well-ordered set
	\( (T \cup \left\{ p \right\},<_{+p}) \).
	Observe that there holds \( S_p(T \cup \left\{ p \right\})=T \)
	and so
	\begin{equation*}
		p = c(X \setminus T) = c(X \setminus S_p(T \cup \left\{ p \right\})),
	\end{equation*}
	from which we conclude that
	\( (T \cup \left\{ p \right\},<_{+p}) \)
	is a tower, as required.
	(Note that \( (\emptyset, \emptyset) \) is a trivial tower,
	and this procedure yields the existence of a non-trivial tower in \( X \)
	that contains arbitrary element of \( X \).)
	
	\fbox{(c)}
	Define
	\begin{equation*}
		(T_k,<_k) \prec (T_{\ell}, <_{\ell})
	\end{equation*}
	if \( (T_k,<_k) \) equals a section of \( (T_{\ell}, <_{\ell}) \).
	(a) and Exercise 5 imply that \( \prec \) is a simple order on
	\( \left\{ (T_k,<_k) \right\} \).
	Hence, Exercise 5 tells us that \( (T,<) \) is a well-ordered set.
	It remains to show that \( (T,<) \) satisfies (\refeq{eq:def_tower})
	since (b) then gives \( T=X \).
	
	Let \(  x \in T \).
	There exists \( k_1 \in K \) for which \(  x \in T_{k_1} \).
	Then there holds
	\begin{equation*}
		x\in c(X \setminus S_{x}(T_{k_1})).
	\end{equation*}
	We claim
	\begin{equation*}
		S_{x}(T_{k_1}) = S_x(T).
	\end{equation*}
	To this end, we only have to show that \leavevmode \par
	\noindent\textit{If \(  x \in T \) and if \(  x \in T_k \) for some \( k\in K \),
		then \( S_{x}(T_{k}) = S_{x}(T_{k'}) \) for all \( k' \in K \)
		for which \(  x \in T_{k'} \)},\leavevmode \par\noindent
	since we then deduce that, under the assumption \(  x \in T_{k_1} \),
	there holds
	\begin{equation*}
		t\in S_{x}(T)
		\equiv
		\Exists{ k\in K }\left[ t\in S_{x}(T_{k}) \right]\\
		\equiv
		t\in S_{x}(T_{k_1}).
	\end{equation*}
	To prove the italic statement, let \(  x \in T \) and \(  x \in T_{k} \cap T_{k'} \).
	(a) implies that we may assume that there exists
	\( s \in T_{k'} \setminus T_{k} \)
	such that \( T_{k}=S_{s}(T_{k'}) \).
	This leads to
	\begin{eqnarray*}
		S_x(T_{k})
		&=&
		\left\{ t\in T_k \pipe t<_{k}x \right\}\\
		&=&
		\left\{ t\in S_{s}(T_{k'}) \pipe t<_{k'}x \right\}\\
		&=&
		\left\{ t\in T_{k'} \pipe t<_{k'}x \right\}\\
		&=&
		S_{x}(T_{k'}).
	\end{eqnarray*}
	
	We have shown so far that a choice function for
	\( \mathcal{P}(X)\setminus \emptyset \)
	yields the set of non-trivial towers in \( X \),
	which gives a tower \( (X,<) \).
	In other words,
	we have seen the choice axiom induce a well-ordering (even with the tower property ).
	Thus, the well-ordering theorem follows from the choice axiom.
	\qed\end{sol}

\begin{exe}
	Using the Exercise 1-4, construct an uncountable well-ordered set, as follows.
	Let \( \mathcal{A} \) be the collection of all pairs \( (A,<) \),
	where \( A \) is a subset of \( \mathbb{Z}_{+} \) and \( < \) is a well-ordering of \( A \).
	(We allow \( A \) to be empty.)
	Define \( (A,<) \simeq (A',<') \)
	if \( (A,<) \) and \( (A',<') \) have the same order type.
	It is trivial to show this is an equivalence relation.
	Let \( \left[ (A,<) \right] \) denote the equivalence class of \( (A,<) \);
	let \( E \) denote the collection of there classes.
	Define
	\begin{equation}\label{secsupE8order}
		\left[ (A,<) \right] \ll \left[ (A',<') \right]
	\end{equation}
	if \( (A,<) \) has the order type of a section of \( (A',<') \).
	\begin{enumerate}
		\item
		      Show that the relation \( \ll \) is well defined and is a simple order on \( E \).
		      Note that the equivalence class \( \left[ (\emptyset,\emptyset) \right] \)
		      is the smallest element of \( E \).
		      
		\item
		      Show that if \( \alpha = \left[ (A,<) \right] \) is an element of \( E \),
		      then \( (A,<) \) has the same order type as the section \( S_{\alpha}(E) \)
		      of \( E \) by \( \alpha \).
		      
		\item
		      Conclude that \( E \) is well-ordered by \( \ll \).
		      
		\item
		      Show that \( E \) is uncountable.
	\end{enumerate}
\end{exe}
\begin{sol}\leavevmode \par
	\fbox{(a)}
	It suffices to show that \( \ll \) is well-defined
	since it then follows from Exercise 2 and 3 that \( \ll \) is a simply ordered.
	Note that Exercise 4 tells us that no member of \( \left[ (A,<) \right] \)
	has the order type of a section of a member of \( \left[ (A,<) \right] \).
	Suppose \( (A,<_A) \) has the order type of a section
	\( S_b(B) \) of \( (B,<_B) \),
	which has the same order type as \( (C,<C) \).
	Then \( (A,<_A) \) has the order type of a section \( (C,<C) \).
	This means that the relation \refeq{secsupE8order} is determined
	independent of the choice of \( (A',<_{A'}) \).
	On the other hand, if \( (D,<_{D}) \) has the same order type as \( (A,<_A) \)
	which has the order type of \( (B,<_B) \),
	then it is obvious that
	we have \( (D,<_{D}) \) also has the order type of a section of \( (B,<_B) \).
	Hence, the relation \refeq{secsupE8order} is determined
	independent of the choice of \( (A,<_{A}) \) and \( (A,<_{A'}) \).
	In other words, the relation \( \ll \) is well-defined.
	
	\fbox{(b)}
	Define a function \( f:A \to E \) via
	\begin{equation*}
		f(x):=\left[( S_x(A),<_{S_{x}(A)}) \right],
	\end{equation*}
	where \( <_{S_{x}(A)} \) is the restriction of \( < \) onto \( S_{x}(A) \).
	\( f \) preserves order by Exercise 2 and 4,
	and so it suffices to show \( f(A)=S_{\alpha}(E) \).
	Clearly, \( f(A) \subset S_{\alpha}(E) \).
	Let \( e \in S_{\alpha}(E) \).
	We have
	\( e = \left[ (A',<_{A'}) \right] \)
	for some \( (A',<_{A'}) \)
	satisfying \( \left[  (A',<_{A'}) \right] \ll \alpha \),
	or equivalently,
	for some \( (A',<_{A'}) \) having the order type of a section of \( (A,<_A) \).
	This means \( e\in f(A) \),
	and hence \( S_{\alpha}(E) \subset f(A) \).
	
	\fbox{(c)}
	Let \( \mathcal{E} \) be a nonempty subset of \( E \),
	and let \( \alpha \in \mathcal{E} \).
	We may assume \( S_{\alpha}(E) \cap \mathcal{E} \neq \emptyset \),
	and it suffices to show that \( S_{\alpha}(E) \cap \mathcal{E}\)
	has a smallest element.
	If \( \alpha = \left[ (A,<_A) \right] \),
	then (b) gives
	\begin{equation*}
		(A,<_A) \simeq S_{\alpha}(E),
	\end{equation*}
	from which we deduce that
	\( S_{\alpha}(E) \cap \mathcal{E}\)
	has the order type of \( (A,<_A) \) or a section of \( (A,<_A) \).
	Since every nonempty subset of \( (A,<_A) \) admits a smallest element,
	so does \( S_{\alpha}(E) \cap \mathcal{E}\).
	
	\fbox{(d)}
	We ignore the advice by Munkres to argue by contradiction,
	and instead show that no injective function from
	\( \mathbb{Z}_{+} \) to \( E \) is surjective.
	Let \( k:\mathbb{Z}_{+} \to E \) be an injection.
	It is obvious that such injection does exist;
	consider, for instance, a function
	\( \mathbb{Z}_{+} \ni n \mapsto \left[ (S_n,<)\right] \in E\),
	where \( < \) is the restriction of the usual order of \( \mathbb{Z}_{+} \).
	Define a simple order \( \prec_{k} \) on \( \mathbb{Z}_{+} \) by setting
	\( x \prec_{k} y \) if \( k(x) \ll k(y) \).
	It is easy to check that \( \prec_{k} \) is well-defined and \( k \) is an order-preserving function from \( (\mathbb{Z}_{+}, \prec_{k}) \) to \( (E ,\ll) \),
	from which we deduce that
	\( \prec_{k} \) is a well-ordering of \( \mathbb{Z}_{+} \).
	Then it turns out that \( k \) is not surjective.
	In fact, we deduce from the fact \( (\mathbb{Z}_{+},\prec_{k}) \) has the order type of a section of \( E \) that
	%exploiting Exercise 2(b),
	the image of \( k \) does not coincide with \( E \).
	Hence, \( k \) is not surjective.
	
	The fact
	\( \inj{\mathbb{Z}_{+}}{E}\neq \emptyset \)
	and none of its member is surjective
	tells us that \( E \) is infinite but is not in bijective correspondence with \( \mathbb{Z}_{+} \).
	Thus,
	\( E \) is uncountable.
	\qed\end{sol}

\end{document}