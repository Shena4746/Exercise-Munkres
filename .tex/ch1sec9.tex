\documentclass[a4paper,12pt]{article}
\usepackage{mystyle}
\usepackage{commands}
\mathtoolsset{showonlyrefs=true}

\begin{document}
\section{Infinite Sets and the Axiom of Choice}
\setcounter{exe}{0}

\begin{rem}[Equivalence of the axiom of choice and choice function]\leavevmode \par
	We have seen
	the axiom of choice imply Lemma 9.2, the existence of choice function.
	But the converse is also true.
	Indeed, if \( \mathcal{A} \) is a collection of disjoint nonempty sets,
	and if \( c \) is a choice function for \( \mathcal{A} \),
	then the set \( C:=\bigcup_{ A \in \mathcal{A}}c(A) \) has the required property.
	So, we from now on identify the choice axiom and Lemma 9.2.
\end{rem}

\begin{exe}
	Define a injective map \( f:\mathbb{Z}_{+} \to X^{\omega} \),
	where \( X \) is the two element set \( \left\{ 0,1 \right\} \),
	without using the choice axiom.
\end{exe}

\begin{sol}
	Letting
	\( \varphi \in \bij{\mathcal{P}(\mathbb{Z}_{+})}{\left\{ 0,1 \right\}^{\omega}} \),
	define a function
	\( f: \mathbb{Z}_{+} \to X^{\omega} \)
	via
	\begin{equation*}
		f(n):=\varphi \circ i(n),
	\end{equation*}
	where \( i: \mathbb{Z}_{+} \to \mathcal{P}(\mathbb{Z}_{+})\)
	is given by
	\( i(x):=\{x\} \),
	which is obviously injective.
	\qed\end{sol}

\begin{exe}
	Find if possible a choice function for each of the following collections,
	without using the choice axiom:
	\begin{enumerate}
		\item
		      The collection \( \mathcal{A} \) of nonempty subsets of \( \mathbb{Z}_{+} \).
		      
		\item
		      The collection \( \mathcal{B} \) of nonempty subsets of \( \mathbb{Z}\).
		      
		\item
		      The collection \( \mathcal{C} \) of nonempty subsets of the rational numbers \( \mathbb{Q}\).
		      
		\item
		      The collection \( \mathcal{D} \) of nonempty subsets of \( X^{\omega} \),
		      where \( X=\left\{ 0,1 \right\} \).
	\end{enumerate}
\end{exe}
\begin{sol}\leavevmode \par
	\fbox{(a)}
	It is easy to check that 
	\( \bigcup_{A\in\mathcal{A}}(A,x_A) \)
	qualifies as the rule for a required choice function,
	where \( x_A \) denotes the smallest element of \( A \).
	
	\fbox{(b)}
	Let \( f:\mathbb{Z} \to \mathbb{Z}_{+} \) be a function defined at \S7 Example 1,
	which is proved to be bijective at \S7 Exercise 2.
	If we set \( c \) to be a choice function for \( \mathcal{A} \) we have found in (a),
	\( f ^{-1} \circ c \circ f \) is a choice function for \( \mathcal{B} \).
	
	\fbox{(c)}
	Note that \( \mathbb{Q} \) is shown to be countable without the choice axiom.(See Exercise 4 if you need Theorem 7.5 to establish the countability.)
	Hence, there exists a bijective function \( g:\mathbb{Q} \to \mathbb{Z}_{+} \).
	Then 
	\( g ^{-1} \circ c \circ g \) is a requested choice function.
	
	\fbox{(d)}
	We cannot find it without the choice axiom,
	and we need some knowledge about well-ordering theorem to explain why.
	(so, you can skip here for the first reading.)
	As Exercise 8 shows, \( X^{\omega} \) and \( \mathbb{R} \) have the same cardinality,
	which implies, as we have seen in (b) and (c),
	that the existence of a choice function for \( \mathcal{D} \) is
	equivalent to that for \( \mathcal{P}(\mathbb{R})\setminus \emptyset \).
	However, Exercise 7 in Supplementary Exercise tells us that
	this is true if and only if the well-ordering theorem is valid on \( \mathbb{R} \),
	which, as a well-known result in set theory says,
	is true only under the choice axiom.
	\qed\end{sol}

\begin{exe}
	Suppose that \( A \) is a set and \( \left\{ f_n \right\}_{n \in \mathbb{Z}_{+}} \)
	is a given indexed family of injective functions
	\begin{equation*}
		f_n:\left\{ 1,\cdots,n \right\} \to A.
	\end{equation*}
	Show that \( A \) is infinite.
	Can you define an injective function \( f:\mathbb{Z}_{+} \to A \)
	without using the choice axiom?
\end{exe}
\begin{sol}
	For every
	\( n \in \mathbb{Z}_{+} \),
	let
	\( A_n:=f_n(\left\{ 1,\cdots,n \right\}) \),
	and let a function
	\( g_n: \left\{ 1,\cdots,n \right\} \to A_n\)
	be given by
	\( g_n(i):=f_n(i) \).
	It is clear that every
	\( g_n \)
	is bijective.
	As we have seen in Exercise 2(a),
	there exists, without the choice axiom, a choice function, denoted by \( c \), for
	\( \mathcal{P}(\mathbb{Z}_{+})\setminus \emptyset \),
	and then, the argument in Exercise 2(d) shows that 
	\( c_n:=g_n \circ c \circ g_n ^{-1} \)
	is a choice function for \( \mathcal{P}(A_n) \setminus \emptyset \)
	for every \( n \).
	
	Now we are ready to construct the required function.
	Consider the following formula for \( f:\mathbb{Z}_{+} \to A \):
	\begin{equation}\label{recursive_formula:injection_f}
		\begin{split}
			f(1)&=A_1\\
			f(n)&=c_n \left( A_n \setminus f(\left\{ 1,\cdots,n-1 \right\}) \right)
		\end{split}
	\end{equation}
	for every \( n >1 \).
	In order to show that (\ref{recursive_formula:injection_f}) is what
	principle of recursive definition applies,
	we claim that the formula designates for \( f \) the unique element of \( A \) for every \( n\in \mathbb{Z}_{+} \).
	We do this by induction.
	Assume our claim holds for up to \( i-1 \).
	We deduce from the fact
	\( A_n \) has exactly \( n \) elements
	while
	\( f(\left\{ 1,\cdots,n-1 \right\}) \) does at most \( n-1 \)
	that 
	\( A_n \setminus f(\left\{ 1,\cdots,n-1 \right\} \neq \emptyset \),
	of which choice function \( c_n \) extracts one element.
	This means that the claim holds for \( i \).
	Then strong induction principle gives that
	our claim holds for all \( n \in \mathbb{Z}_{+} \).
	Hence, (\refeq{recursive_formula:injection_f}) defines unique
	\( f:\mathbb{Z}_{+} \to A \)
	that satisfies the formula.
	
	We at last insists that \( f \) is injective.
	In fact,
	if
	\( n <m \),
	or equivalently
	\( n \le m-1 \),
	then the fact that \( c_n \) is a choice function for
	\( \mathcal{P}(A_n) \setminus \emptyset\) gives
	\begin{equation*}
		f(m)
		\in
		A_m \setminus f \left( \left\{ 1,\cdots,n-1,n,n+1,\cdots,m-1 \right\} \right),
	\end{equation*}
	which yields
	\( f(n) \neq f(m) \).
	Thus,
	\( f \)
	is injective,
	and we complete the construction of \( f \) without the choice axiom.
	
	Note that the implication of (1) \( \Rightarrow \) (3) in Theorem 9.1 is valid without the choice axiom.
	Thus, the fact \( A \) is infinite can also be proved without it.
	\qed\end{sol}

\begin{exe}
	There was a theorem in \S7 whose proof involved an infinite number of arbitrary choices.
	Which one was it?
	Rewrite the proof so as to make explicit the use of the choice axiom.
\end{exe}
\begin{sol}
	That is where Theorem 7.5 says
	''Because each \( A_n \) is countable,
	we can choose, for each \( n \),
	a surjective function \( f_n:\mathbb{Z}_{+} \to A \)''.
	Here the set \( A_n \) is indexed by the set \( J \),
	which might be infinite.
	If it is infinite, we need the choice axiom (or a choice function) in order to extract
	\( f_n \) from each of nonempty set \( \surj{\mathbb{Z}_{+}}{A_n} \).
	If otherwise, say, if \( J \) is finite, of course we do not need it
	since the finite axiom of choice suffices.
	
	Now it is obvious how we should rewrite it.
	\qed\end{sol}

\begin{exe}\leavevmode \par
	\begin{enumerate}
		\item
		      Use the choice axiom to show that if \( f:A \to B \) is surjective,
		      then \( f \) has a right inverse \( h:B\to A \).
		      
		\item
		      Show that if \( f:A \to B \) is injective and \( A \) is not empty,
		      then \( f \) has a left inverse.
		      Is the axiom of choice needed?
	\end{enumerate}
\end{exe}
\begin{sol}
	This Exercise is the converse of \S2 Exercise 5(a),
	which insists that,
	
	\noindent \textit{if a function admits a right inverse, the function is surjective;
		if a function admits a left inverse, the function is injective.}
	
	\fbox{(a)}
	Consider the indexed family \( \left\{ f ^{-1}(b) \right\}_{b\in B} \) of sets.
	Each \( f ^{-1}(b) \) is nonempty since \( f \) is surjective.
	The choice axiom allows us to have a choice function
	\( c:B \to \bigcup_{b \in B}f ^{-1}(b) \).
	Then we see that for every \( b\in B \)
	\begin{equation*}
		f \circ c(b) = f(c(b)) = b
	\end{equation*}
	since \( c(b) \in f ^{-1}(b) \).
	Seeing \( c \) as a function from \( B \) to A,
	we conclude that  \( f \) has a right inverse \( c \).
	
	\fbox{(b)}
	The axiom choice is not needed this time.
	Indeed,
	if \( f \in \inj{A}{B} \)
	and \( A \neq \emptyset \),
	then a function \( \varphi:A \to f(A) \)
	given by
	\( \varphi(a)=f(a) \)
	is bijective,
	and hence we have
	\begin{equation*}
		\varphi ^{-1} \circ f(a) = a
	\end{equation*}
	for all \( a \in A \).
	(Extend \( \varphi ^{-1} \) onto \( B \) if necessary by setting arbitrary value outside \( f(A) \).)
	\qed\end{sol}

\begin{rem}
	Here are several technical remark about empty function.
	
	In Exercise 5(b), we cannot drop the assumption that ''\( A \) is not empty'';
	\( A = \emptyset \) means that
	\( f \) and therefore \( \varphi \) are an empty function,
	which is born injective but not surjective whatever the range is.
	Hence, \( \varphi \) by no means has the inverse.
	
	Also, note that in Exercise 5(a), the assumption \( f \) is surjective
	rejects the possibility of 
	\( f \) being an empty function,
	and consequently guarantees that \( A \) and \( B \) are not empty.
	We use this fact implicitly in the above proof.  
\end{rem}

\begin{exe}
	Most of famous paradoxes of naive set theory are associated in some way or other with the concept of the ''set of all sets.''
	None of the rules we have given for forming sets allows us to consider such a set.
	And for good reason---the concept itself is self-contradictory.
	For suppose that \( \mathcal{A} \) denotes the ''set of all sets.''
	\begin{enumerate}
		\item
		      Show that \( \mathcal{P}(\mathcal{A}) \subset \mathcal{A} \);
		      derive a contradiction.
		      
		\item
		      (\textit{Russell's paradox.})
		      Let \( \mathcal{B} \) be the subset of \( \mathcal{A} \) consisting of all sets that are not element of themselves;
		      \begin{equation*}
			      \mathcal{B} = \left\{ A \pipe A \in \mathcal{A}\wedge A \notin A \right\}
		      \end{equation*}
	\end{enumerate}
	(Of course, there may be \textit{no} set \( A \) such that \( A\in A \);
	if such is the case, then \( \mathcal{B} = \mathcal{A}\).)
	Is \( \mathcal{B} \) an element of itself or not?
\end{exe}
\begin{sol}\leavevmode \par
	\fbox{(a)}
	If
	\( a \in \mathcal{P}(A) \),
	then \( a \) itself is a set by the definition of \( \mathcal{A} \).
	Hence, \( a \in \mathcal{A} \).
	Thus, \( \mathcal{P}(\mathcal{A}) \subset \mathcal{A} \).
	But
	\( \mathcal{P}(A) \subset A\)
	leads
	\( \mathcal{P}(A) \hookrightarrow A\),
	contradicting to Theorem 7.8.
	
	\fbox{(b)}
	BOTH !
	
	If we assume \( \mathcal{B} \notin \mathcal{B} \),
	then the definition of
	\( \mathcal{B} \)
	leads to
	\( \mathcal{B}\in \mathcal{B} \).
	On the other hand, if we suppose \( \mathcal{B} \in \mathcal{B} \),
	then we are led to \( \mathcal{B} \notin \mathcal{B} \).
	\qed\end{sol}

\begin{exe}
	Let \( A \) and \( B \) be two nonempty sets.
	If there is an injection of \( B \) into \( A \),
	but no injection of \( A \) into \( B \),
	we say that \( A \) has \textbf{\textit{greater cardinality}} than \( B \).
	\begin{enumerate}
		\item
		      Conclude from Theorem 9.1 that every uncountable set has greater cardinality than \( \mathbb{Z}_{+} \).
		      
		\item
		      Show that if \( A \) has greater cardinality than \( B \),
		      and if \( B \) has greater cardinality than \( C \),
		      then \( A \) has greater cardinality than \( C \).
		      
		\item
		      Find a sequence \( A_1,A_2,\cdots \) of infinite sets, such that for each
		      \( n \in \mathbb{Z}_{+} \),
		      the set \( A_{n+1} \) has greater cardinality than \( A_n \).
		      
		\item
		      Find a set that for every \( n \) has greater cardinality than \( A_n \).
	\end{enumerate}
\end{exe}
\begin{sol}\leavevmode \par
	\fbox{(a)}
	Let \( A \) be a uncountable set, and \( B \) a countable set.
	Theorem 9.1 tells us that \( B \hookrightarrow A \)
	while definition of uncountable set implies
	\( \bij{A}{B} = \emptyset \).
	Thus, contrapositive of Schroeder-Bernstein theorem gives
	\( \inj{A}{B} = \emptyset \),
	from which we conclude the result.
	
	\fbox{(b)}
	\S2 Exercise 4 gives \( C \hookrightarrow A \),
	and contrapositive of that Exercise also implies
	\( \inj{A}{C} = \emptyset \).
	Thus, the result follows.
	
	\fbox{(c)}
	Define \( A_n \) by the formula below:
	\begin{eqnarray*}
		A_1&=&A,\\
		A_n&=&\mathcal{P}(A_{n-1})
	\end{eqnarray*}
	for all \( n >1 \).
	It is easy to check that principle of recursion definition applies for this formula,
	and hence \( A_n \) is unique defined.
	
	\( A_{n+1} \) has greater cardinality than \( A_n \)
	since we see
	\( A_n \hookrightarrow \mathcal{P}(A_n) = A_{n+1} \)
	while there holds 
	\( \inj{\mathcal{P}(A_n)}{A_n} = \emptyset \)
	by Theorem 7.8.
	
	\fbox{(d)}
	Define
	\( A^{\ast}:=\mathcal{P}(\bigcup_{n\in \mathbb{Z}_{+}}A_n)\),
	and repeat the same argument.
	\qed\end{sol}

\begin{exe}
	Show that \( \mathcal{P}(\mathcal{\mathbb{Z}_{+}}) \) and \( \mathbb{R} \) have the same cardinality.
\end{exe}
\begin{sol}
	It is easy to see, by considering decimal expansion for real numbers, that
	\begin{equation}\label{preeq:RandPZ}
		[0,1)
		\hookrightarrow
		\left\{ 0,1,2,\cdots,8,9 \right\}^{\omega}
		\sim
		\left\{ 0,1 \right\}^{\omega}
		\hookrightarrow
		[0,1).
	\end{equation}
	For instance, using the fact that any member \( x \) of \( [0,1) \) can be uniquely written, if we limit the expression to those not having infinite 9's tail, as
	\begin{equation*}
		x = 0. x_1 x_2 x_2 \cdots,
	\end{equation*}
	where \( x_i \in \left\{ 0,1,2,\cdots,8,9 \right\} \)
	for every \( i \in \mathbb{Z}_{+} \),
	we can define a injection
	\( \iota:[0,1) \to \left\{ 0,1,2,\cdots,8,9 \right\}^{\omega}\)
	via
	\begin{equation*}
		\iota (x) := (x_i)_{ i \in \mathbb{Z}_{+}}.
	\end{equation*}
	Since we can deduce from Schroeder-Bernstein theorem that
	all the sets at (\refeq{preeq:RandPZ}) have the same cardinality,
	it remains to show
	\( [0,1) \sim \mathbb{R} \),
	and 
	\( \left\{ 0,1,2,\cdots,8,9 \right\}^{\omega}
	\sim
	\left\{ 0,1 \right\}^{\omega}
	\sim
	\mathcal{P}(\mathbb{Z}_{+}) \).
	
	We first work on the former.
	We claim
	\begin{equation*}
		\mathbb{R}
		\sim
		(-1,1)
		\sim
		(0,1)
		\sim
		[0,1).
	\end{equation*}
	For the first equivalence, see \S3 Exercise 10;
	for  the second consider a function \( f:(-1,1)\to(0,1) \) given by
	\begin{equation*}
		f(x):=\frac{1}{2}x +1;
	\end{equation*}
	for the third, verify that a function
	\( g:[0,1)\to (0,1) \) 
	given by
	\begin{equation*}
		g(x):=\begin{cases}
			\frac{1}{2} & \;\mathrm{:}\; x= 0                                                       \\
			\frac{x}{2} & \;\mathrm{:}\; \Exists{n\in \mathbb{Z}_{+}}\left[ x=\frac{1}{2^n} \right] \\
			x           & \;\mathrm{:}\; \text{otherwise}
		\end{cases}
	\end{equation*}
	is bijective.
	
	For the latter, we have, as we have  repeatedly seen,
	\( \mathcal{P}(\mathbb{Z}_{+})
	\sim
	\mathbb{Z}_{+}^{\omega}
	\sim
	\left\{ 0,1 \right\}^{\omega}
	\)
	and
	\begin{equation*}
		\left\{ 0,1 \right\}^{\omega}
		\hookrightarrow
		\left\{ 0,1,2,\cdots,8,9 \right\}^{\omega}
		\hookrightarrow
		\mathbb{Z}_{+}^{\omega},
	\end{equation*}
	which yield, by Schroeder-Bernstein theorem, that
	\begin{equation*}
		\left\{ 0,1,2,\cdots,8,9 \right\}^{\omega}
		\sim
		\mathbb{Z}_{+}^{\omega}
		\sim
		\mathcal{P}(\mathbb{Z}_{+}).
	\end{equation*}
	
	Thus, we conclude
	\( \mathbb{R} \sim \mathcal{P}(\mathbb{Z}_{+}) \).
	\qed\end{sol}

\end{document}