\documentclass[a4paper,12pt]{article}
\usepackage{mystyle}
\usepackage{commands}
\mathtoolsset{showonlyrefs=true}
\begin{document}
\section{Cartesian Product}
\setcounter{exe}{0}

Note that \S2 Exercise 5 serves as an useful criterion to show that 
a given function is injective or surjective, or bijective.
We exploit it throughout this section,
and rarely prove a function is, for instance, bijective by chasing definition.

\begin{exe}
	Show there is a bijective correspondence of \( A\times B \)
	and \( B \times A \).
\end{exe}\begin{sol}
	Observe each of the following functions
	\begin{eqnarray*}
		&&f:A\times B \ni (a,b)\mapsto (b,a)\in B \times A\\
		&&g:B\times A \ni (b,a)\mapsto (a,b)\in A \times B
	\end{eqnarray*}
	is the inverse of the other, and hence is bijective.
	\qed\end{sol}

\begin{exe}\leavevmode \par
	\begin{enumerate}
		\item
		      Show that if \( n>1 \)
		      there is bijective correspondence of
		      \begin{equation*}
			      A_1 \times \cdots \times A_n
		      \end{equation*}
		      with
		      \begin{equation*}
			      (A_1 \times \cdots \times A_{n-1})\times A_{n}
		      \end{equation*}
		      
		\item
		      Given the indexed family \( \left\{ A_1, A_2\cdots \right\} \),
		      let \( B_i=A_{2i-1}\times A_{2i} \)
		      for each positive integer \( i \).
		      Show there is bijective correspondence of \( A_1 \times A_2 \times \cdots \)
		      with \( B_1 \times B_2 \times \cdots \).
	\end{enumerate}
\end{exe}\begin{sol}\leavevmode \par
	\fbox{(a)}
	Check that each of the following functions is the inverse of the other:
	\begin{eqnarray*}
		&&	f:\prod_{i =1}^n A_i \ni (a_1,\cdots,a_n)
		\mapsto ((a_1,\cdots,a_{n-1}),a_n)
		\in \left( \prod_{i =1}^{n-1} A_i \right) \times A_n\\
		&&	g:\left( \prod_{i =1}^{n-1} A_i \right) \times A_n
		\ni ((a_1,\cdots,a_{n-1}),a_n) \mapsto (a_1,\cdots,a_n)
		\in \prod_{i =1}^n A_i.
	\end{eqnarray*}
	
	\fbox{(b)}
	Apply the same argument to:
	\begin{eqnarray*}
		&&	f:\prod_{i \in \mathbb{Z}_{+}} A_i \ni (a_i)_{i \in \mathbb{Z}_{+}}
		\mapsto ((a_{2i - 1},a_{2i}))_{i \in \mathbb{Z}_{+}}
		\in \prod_{i \in \mathbb{Z}_{+}} B_i \\
		&&	g:\prod_{i \in \mathbb{Z}_{+}} B_i
		\ni ((a_{2i - 1},a_{2i}))_{i \in \mathbb{Z}_{+}}
		\mapsto (a_i)_{i \in \mathbb{Z}_{+}}
		\in \prod_{i =1}^n A_i.
	\end{eqnarray*}
	\qed\end{sol}

\begin{exe}
	Let \( A=A_1 \times A_2 \times \cdots \)
	and
	\( B = B_1 \times B_2 \times \cdots \).
	\begin{enumerate}
		\item
		      Show that if \( B_i \subset A_i \) for all \( i \),
		      then \( B \subset A \).
		      (Strictly speaking,
		      if we are given a function mapping the index set \( \mathbb{Z}_{+} \) into the union of the sets \( B_i \),
		      we must change its range before it can be considered as a function mapping
		      \( \mathbb{Z}_{+} \) into the union of the sets \( A_i \).
		      We shall ignore this technicality when dealing with cartesian product.)
		      
		\item
		      Show the converse of (a) holds if \( B\) is nonempty.
		      
		\item
		      Show that if \( A \) is nonempty, each \( A_i \) is nonempty.
		      Does the converse hold?
		      
		\item
		      What is the relation between the set \( A\cup B \)
		      and the cartesian product of the sets \( A_i \cup B_i \)?
		      What is the relation between the set \( A\cap B \) and the cartesian product of the sets \( A_i \cap B_i \)?
	\end{enumerate}
\end{exe}\begin{sol}\leavevmode \par
	\fbox{(a)}
	Let 
	\( \left( x_i \right)_{i \in \mathbb{Z}_{+}}\in B \).
	Hypothesis gives 
	\( x_i \in B_i \subset A_i \)
	for every
	\( i \),
	which means
	\( \left( x_i \right)_{i \in \mathbb{Z}_{+}}\in A \).
	
	\fbox{(b)}
	Fix \( i_0 \in \mathbb{Z}_{+} \)
	and
	\( b_0 \in B_{i_0} \).
	Hypothesis allows us to have
	\( \left( b_i' \right)_{i \in \mathbb{Z}_{+}} \in B\).
	Then, define a new \( \omega \)-tuple 
	\( (b_i)_{i \in \mathbb{Z}_{+}}\) by setting
	\begin{equation*}
		b_{i} := 
		\begin{cases}
			b_0				
			 & \mathrm{\colon\;} i = i_0 \\
			b_{i}'
			 & \mathrm{\colon\;}
			i \in \mathbb{Z}_{+} \setminus \left\{ i_0 \right\},
		\end{cases}
	\end{equation*}
	from which it follow that
	\( (b_i)_{i \in \mathbb{Z}_{+}} \in B \subset A\).
	Hence, we have, in particular,
	\( b_0 \in A_{i_0} \).
	
	\fbox{(c)}
	If \( (a_i)_{i \in \mathbb{Z}_{+}} \in A  \),
	then for every \( i \) we have
	\( a_i \in A_i \)
	and hence, \( A_i \neq \emptyset \).
	
	Converse does not hold in general.
	We cannot conclude from the assumption that existence of \( \omega \)-tuple which qualifies as a element of Cartesian product \( A \).
	
	\fbox{(d)}
	For intersection,
	confirm the following equivalence:
	\begin{eqnarray*}
		\left( x_i \right)_{i \in \mathbb{Z}_{+}}
		\in \left( \prod_{i \in \mathbb{Z}_{+}} A_i \right)
		\cap
		\left( \prod_{i \in \mathbb{Z}_{+}} B_i \right)
		&\equiv&
		\Forall{ i \in \mathbb{Z}_{+} }\left[ x_i \in A_i \right]
		\wedge
		\Forall{ i \in \mathbb{Z}_{+} }\left[ x_i \in B_i \right]\\
		&\equiv&
		\Forall{ i \in \mathbb{Z}_{+} }\left[ x_i \in A_i \cap B_i \right]\\
		&\equiv&
		\left( x_i \right)_{i \in \mathbb{Z}_{+}} \in
		\prod_{i \in \mathbb{Z}_{+}} \left( A_i \cap B_i \right).
	\end{eqnarray*}
	For union, however, we can only insist, as we have seen in \S1 Exercise 2(m), that
	\begin{equation*}
		\left( \prod_{i \in \mathbb{Z}_{+}} A_i \right)
		\cup
		\left( \prod_{i \in \mathbb{Z}_{+}} B_i \right)
		\subset
		\prod_{i \in \mathbb{Z}_{+}} \left( A_i \cup B_i \right).
	\end{equation*}
	Proof and counterexample can be derived by trivial modification of those found there.
	\qed\end{sol}

\begin{exe}
	Let \( m,n \in \mathbb{Z}_{+} \).
	Let \( X \neq \emptyset \).
	\begin{enumerate}
		\item
		      If \( m \le n \), find an injective map \( f:X^m \to X^n \).
		      
		\item
		      Find a bijective map \( g:X^m \times X^n \to X^{m+n} \).
		      
		\item
		      Find a bijective map \( h:X^n \to X^{\omega} \).
		      
		\item
		      Find a bijective map \( k:X^n \times X^{\omega} \to X^{\omega} \).
		      
		\item
		      Find a bijective map \( \ell:X^{\omega} \times X^{\omega} \to X^{\omega} \).
		      
		\item
		      If \( A \subset B \), find an injective map \( f:X^A \to X^B \).
	\end{enumerate}
\end{exe}\begin{sol}\leavevmode \par
	\fbox{(a)}
	Let
	\( m \le n \).
	Fix
	\( x_0 \in X \)
	and consider a function \( f \) given by
	\begin{equation*}
		f:X^m \ni (x_1,\cdots,x_m)
		\mapsto
		(x_1,\cdots,x_m,\underbrace{x_0,x_0,\cdots,x_0}_{n-m}) \in X^n,
	\end{equation*}
	and note that \( f \) has a left inverse
	\begin{equation*}
		f_{\ell}:X^n \ni (x_1,\cdots,x_n)
		\mapsto
		(x_1,\cdots,x_m) \in X^m.
	\end{equation*}
	Thus, \S2 Exercise 5 implies
	\( f \)
	is injective.
	
	Note that \( f \) is a trivial ''inclusion'' function from \( X^m \) into \( X^n \)
	while \( f_{\ell} \) is a ''cut-off'' function of \( X^n \) to \( X^m \).
	
	\fbox{(b)}
	Confirm that
	\begin{equation*}
		g:X^m \times X^n \ni
		((x_1,\cdots,x_m),(y_1,\cdots,y_n))
		\mapsto
		(x_1,\cdots,x_m,y_1,\cdots,y_n)
		\in X^{m+n}
	\end{equation*}
	is bijective since it admits the inverse
	\begin{equation*}
		g_1:X^{m+n} \ni
		(x_1,\cdots,x_{n+m})
		\mapsto
		((x_1,\cdots,x_m),(x_{m+1},\cdots,x_{n+m}))
		\in X^m \times X^n.
	\end{equation*}
	
	\fbox{(c)}
	Similar argument for (a) works here.
	Fix
	\( x_0 \in X \)
	and define a ''inclusion'' function
	\begin{equation*}
		h:X^m \ni (x_1,\cdots,x_m)
		\mapsto
		(x_1,\cdots,x_m,x_0,x_0,\cdots) \in X^{\omega}.
	\end{equation*}
	\( h \)
	admits a left inverse, which is given as a''cut-off'' function by the rule
	\begin{equation*}
		h_{\ell}:X^{\omega} \ni (x_i)_{i \in \mathbb{Z}_{+}}
		\mapsto
		(x_1,\cdots,x_n) \in X^n.
	\end{equation*}
	So, \( f \) is injective.
	
	\fbox{(d)}
	Make sure that a function given by
	\begin{equation*}
		k:X^n \times X^{\omega} \ni
		((x_1,\cdots,x_n),(y_1,y_2,\cdots))
		\mapsto
		(x_1,\cdots,x_n,y_1,y_2,\cdots)
		\in X^{\omega}
	\end{equation*}
	is bijective since it has the inverse
	\begin{equation*}
		k_1:X^{\omega} \ni
		(x_i)_{i \in \mathbb{Z}_{+}}
		\mapsto
		((x_1,\cdots,x_n),(x_{n+1},x_{n+2},\cdots))
		\in X^n \times X^{\omega}.
	\end{equation*}
	
	\fbox{(e)}
	Do the same thing as (d) to the following:
	\begin{eqnarray*}
		&&\ell : X^{\omega} \times X^{\omega} \ni
		\left( (x_i)_{i \in \mathbb{Z}_{+}}, (y_i)_{i \in \mathbb{Z}_{+}} \right)
		\mapsto
		(x_1,y_1,x_2,y_2,\cdots)
		\in X^{\omega}\\
		&&\ell_1 : X^{\omega} \ni
		\left( x_i \right)_{i \in \mathbb{Z}_{+}}
		\mapsto
		\left(	\left( x_{2i-1} \right)_{i \in \mathbb{Z}_{+}},
		\left( x_{2i} \right)_{i \in \mathbb{Z}_{+}}
		\right)
		\in X^{\omega} \times X^{\omega}.
	\end{eqnarray*}
	
	\fbox{(f)}
	Successive application of (e) implies a function
	\begin{equation*}
		\ell_0:	\left( A^{\omega} \right)^n
		\ni 	\left( (a_i ^{(1)})_{i \in \mathbb{Z}_{+}},
		\cdots,
		(a_i ^{(n)})_{i \in \mathbb{Z}_{+}}
		\right)
		\mapsto
		\left( a_1 ^{(1)}, a_1 ^{(2)},\cdots,a_1 ^{(n)},
		a_2 ^{(1)}, a_2 ^{(2)},\cdots
		\right)
		\in A^{\omega}
	\end{equation*}
	is bijective.
	So,
	\begin{equation*}
		m:=i\circ \ell_0
	\end{equation*}
	is injective,
	where
	\begin{equation*}
		i:A^{\omega} \ni (a_i)_{i \in \mathbb{Z}_{+}}
		\mapsto
		(a_i)_{i \in \mathbb{Z}_{+}} \in B^{\omega},
	\end{equation*}
	which is obviously injective.
	\qed\end{sol}

\begin{exe}
	Which of the following subsets of \( \mathbb{R}^{\omega} \)
	can be expressed as the cartesian product of subsets of \( \mathbb{R} \)?
	\begin{enumerate}
		\item
		      \( \left\{ \bm{x} \pipe x_i \;\text{is an integer for all \( i \)} \right\} \).
		      
		\item
		      \( \left\{ \bm{x} \pipe x_i \ge i \;\text{for all \( i \)} \right\} \).
		      
		\item
		      \( \left\{ \bm{x} \pipe x_i \;\text{is an integer for all \( i\ge 100 \)} \right\} \).
		      
		\item
		      \( \left\{ \bm{x} \pipe x_2=x_3 \right\} \).
	\end{enumerate}
\end{exe}\begin{sol}\leavevmode \par
	\fbox{(a)}
	\( \mathbb{Z}^{\omega} \).
	\fbox{(b)}
	\( [1,\infty) \times [2,\infty) \times [3,\infty)\times \cdots \).
	\fbox{(c)}
	\(
	\underbrace{\mathbb{R}\times \cdots \times \mathbb{R}}_{99}
	\times \mathbb{Z} \times \mathbb{Z} \times \cdots
	\).
	
	\fbox{(d)}
	Let \( X:=\left\{ x \pipe x_2=x_3 \right\} \).
	\( X \) is not the cartesian product of subsets of \( \mathbb{R} \).
	Observe
	\( (0,0,\cdots),(1,1,\cdots) \in X \),
	but
	\( (0,1,0,\cdots) \notin X \),
	which violates an obvious necessary condition for \( X \) to be the cartesian product.
	\qed\end{sol}

We end this section by introducing an extension of the proposition we have proved in \S1, which can be used to work on Exercise 5(d).
\begin{prp}[Equivalent condition for cartesian product]
	Let
	\( X_i \)
	be a set for every
	\( i \in \mathbb{Z}_{+} \),
	and let
	\( A \)
	be a subset of
	\( \prod_{i \in \mathbb{Z}_{+}} X_i \).
	A necessary and sufficient condition for \( A \) to be the cartesian product of subsets of \( X_i \) is that there holds
	\begin{equation*}
		\left( x_i^{(1)} \right)_{i \in \mathbb{Z}_{+}},
		\left( x_i^{(2)} \right)_{i \in \mathbb{Z}_{+}},\cdots \in A 
		\Rightarrow
		(z_i)_{i \in \mathbb{Z}_{+}} \in A,
	\end{equation*}
	where \( z_i=x_i^{(j)} \)
	for some
	\( j \in \mathbb{Z}_{+}\).
\end{prp}
\begin{prf}
	Sufficiency is proved as before.
	We show necessity part.
	Suppose that \( A \) itself is the cartesian product,
	and let
	\( \left( x_i^{(1)} \right)_{i \in \mathbb{Z}_{+}}
	,\left( x_i^{(2)} \right)_{i \in \mathbb{Z}_{+}},\cdots \in A \).
	Let \( I \) be a subset of \( \mathbb{Z}_{+} \) such that we have
	\( (z_i)_{i \in \mathbb{Z}_{+}} \in A \),
	where each \( z_i \) is given by
	\begin{equation*}
		z_i:=\begin{cases}
			x_i^{(j)} & \mathrm{:}\;i \in I    \\
			x_i^{(1)} & \mathrm{:}\;i \notin I
		\end{cases}
	\end{equation*}
	for arbitrary chosen
	\( j \in \mathbb{Z}_{+}\).
	Since \( A \) is assumed to be a cartesian product,
	it is easy to see that \( I \) is inductive, and so \( I=\mathbb{Z}_{+} \).
	This proves the necessity.
\end{prf}
\end{document}