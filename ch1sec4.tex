\documentclass[a4paper,12pt]{article}
\usepackage{mystyle}
\usepackage{commands}
\mathtoolsset{showonlyrefs=true}

\begin{document}
\section{The Integers and Real Numbers}
\setcounter{exe}{0}
\begin{rem}[Direct proof of Strong induction principle]
	Here we provide another proof of Theorem 4.2, Strong induction principle, 
	that includes no apagogical argument.
	You can safely skip here if you are not interested.
	This note is just an illustration of a possible way of verifying the result,
	and nothing given here is needed later.
	
	Logical rule allows us to deduce that following statements are all equivalent to Theorem 4.2:
	\begin{eqnarray*}
		&&
		A\subset \mathbb{Z}_{+}
		\Rightarrow
		\left[
			\Forall{ n \in \mathbb{Z}_{+}}
			\left[
				S_n \subset A \Rightarrow n\in A
				\right]
			\Rightarrow
			A=\mathbb{Z}_{+}
			\right]\\
		&\equiv&
		\left[
			A\subset \mathbb{Z}_{+}
			\Rightarrow 
			\left[
				A \subsetneq \mathbb{Z}_{+}
				\Rightarrow
				\neg
				\Forall{ n \in \mathbb{Z}_{+}}
				\left[
					S_n \subset A \Rightarrow n\in A
					\right]
				\right]
			\right]\\
		&\equiv&
		\left[
			A\subset \mathbb{Z}_{+}
			\Rightarrow
			\left[
				A \subsetneq \mathbb{Z}_{+}
				\Rightarrow
				\neg \Forall{ n \in \mathbb{Z}_{+}}
				\left[
					\neg \left( S_n \subset A \right)  \vee n\in A
					\right]
				\right]
			\right]\\
		&\equiv&
		\left[
			A\subset \mathbb{Z}_{+}
			\Rightarrow
			\left[
				A \subsetneq \mathbb{Z}_{+}
				\Rightarrow
				\Exists{ n \in \mathbb{Z}_{+}}
				\left[
					S_n \subset A \wedge n\notin A
					\right]
				\right]
			\right].
	\end{eqnarray*}
	We, of course, prove the last one.
	Suppose
	\( A \subsetneq \mathbb{Z}_{+} \).
	Then, Theorem 4.1 lets us have a smallest element
	\( n \) of
	\( \mathbb{Z}_{+} \setminus A \).
	It turns out that so derived \( n \) has the required property.
	In fact, smallest property implies 
	\( S_n \subset A \),
	and
	\( n \in \mathbb{Z}_{+} \setminus A \)
	does
	\( n\notin A \).
	Thus, Theorem 4.2 follows.
\end{rem}

\begin{rem}[Direct proof of Archimedean ordering property]\label{Archimedean}
	We continue to apply an analogous argument.
	This time, however,
	the proof establishes some useful fact that is actually exploited later.(see Exrceise 8 and 9, for instance.)
	
	It is clear that following statements are equivalent to Archimedean ordering property:
	
	\begin{eqnarray*}
		\neg
		\Exists{ b\in \mathbb{R} }
		\Forall{ n\in \mathbb{Z}_{+} }
		\left[ n \le b \right]
		&\equiv&
		\Forall{ b\in \mathbb{R} }
		\neg
		\Forall{ n\in \mathbb{Z}_{+} }
		\left[ n \le b \right]\\
		&\equiv&
		\Forall{ b\in \mathbb{R} }
		\Exists{ n\in \mathbb{Z}_{+} }
		\left[ n > b \right].
	\end{eqnarray*}
	Let
	\( A \) be a subset of \( \mathbb{Z}_{+} \) such that 
	\begin{equation*}
		\Forall{ b\in \mathbb{R} }
		\Exists{ n\in A }
		\left[ n > b \right]
	\end{equation*}
	holds.
	It is clear that 
	\( A \)
	satisfies the Hypothesis of Theorem 4.2,
	from which we conclude \( A = \mathbb{Z}_{+}\).
	This establishes the result.
	
	It is convenient to use Archimedean ordering property in the form we have just given the proof of, rather than the original stated in the text.
	This equivalent statement of Archimedean ordering property can be verbalized as:\\
	\textit{For any real number, there exists a positive integer larger than the real number.}
\end{rem}

\begin{exe}
	Prove the following ''laws of algebra'' for \( \mathbb{R} \),
	using only axiom (1)--(5):
	\begin{enumerate}
		\item
		      If \( x+y=x \),
		      then \( y=0 \).
		      
		\item
		      \( 0\cdot x =0 \).
		      
		\item
		      \( -0=0 \).
		      
		\item
		      \( -(-x)=x \).
		      
		\item
		      \( x(-y)=-(xy)=(-x)y \).
		      
		\item
		      \( (-1)x = -x\).
		      
		\item
		      \( x(y-z) = xy - xz \).
		      
		\item
		      \( -(x-y)=-x-y;\,-(x-y)=-x+y \).
		      
		\item
		      If
		      \( x \neq 0 \)
		      and
		      \( x \cdot y =x \)
		      then \( y=1 \).
		      
		\item
		      \( x/x = 1 \)
		      if \( x \neq 0 \).
		      
		\item
		      \( x/1=x \).
		      
		\item
		      \( x\neq 0 \)
		      and
		      \( y\neq 0 \Rightarrow xy \neq 0\).
		      
		\item
		      \( (1/y)(1/z)=1/(yz) \)
		      if \( y,z\neq 0 \).
		      
		\item
		      \( (x/y)(w/z)=(xw)/(yz) \)
		      if \( y,z\neq 0 \).
		      
		\item
		      \( (x/y)+(w/z)=(xz+wy)/(yz) \)
		      if \( y,z\neq 0 \).
		      
		\item
		      \( x \neq 0 \Rightarrow 1/x \neq 0 \).
		      
		\item
		      \( 1/(w/z) = z/w\)
		      if
		      \( w,z, \neq0 \).
		      
		\item
		      \( (x/y)/(w/z)=(xz)/(yw) \)
		      if
		      \( y,w,z\neq 0 \).
		      
		\item
		      \( (ax)/y=a(x/y) \)
		      if
		      \( y \neq 0 \).
		      
		\item
		      \( (-x)/y=x/(-y)=-(x/y) \)
		      if
		      \( y \neq 0\).
	\end{enumerate}
\end{exe}\begin{sol}\leavevmode \par
	\fbox{(a)}
	The result is verified by
	\begin{eqnarray*}
		0
		&\equalby{(4)}&
		x +(-x)
		=
		(x+y)+(-x)
		\equalby{(1)}
		x+\left( y+(-x) \right)\\
		&\equalby{(2)}&
		x+\left( (-x)+y \right)
		\equalby{(1)}
		\left( x+(-x) \right)+y
		\equalby{(4)}
		0+y
		\equalby{(3)}
		y.
	\end{eqnarray*}
	
	\fbox{(b)}
	It is easy to see that
	\begin{eqnarray*}
		x\cdot x+0\cdot x
		\equalby{(2)}
		x\cdot x+x\cdot 0
		\equalby{(5)}
		x\cdot(x+0)
		\equalby{(3)}
		x\cdot x.
	\end{eqnarray*}
	Thus, (a) implies the result.
	
	\fbox{(c)}
	(3) yields
	\( 0+0=0 \).
	Hence,
	\( -0=0 \)
	by (4).
	
	\fbox{(d)}
	Observe
	\( x+(-x)=0 \)
	and use (4).
	
	\fbox{(e)}
	It follows that
	\begin{equation*}
		0
		\equalby{(b)}
		x 0
		\equalby{(4)}
		x(y+(-y))
		\equalby{(5)}
		xy+x(-y).
	\end{equation*}
	So, by (4), we have
	\( x(-y)=-(xy) \).
	Similar argument shows
	the other equality.
	
	\fbox{(f)}
	We deduce that 
	\begin{equation*}
		0
		\equalby{(b)}
		\left( 1+(-1) \right)x
		\equalby{(1)}
		x+(-1)x,
	\end{equation*}
	and that
	\( (-1)x=-x \)
	by (4).
	
	\fbox{(g)}
	Check that
	\begin{equation*}
		x(y-z)
		=
		x(y+(-z))
		\equalby{(5)}
		xy+x(-z)
		\equalby{(e)}
		xy-xz.
	\end{equation*}
	
	\fbox{(h)}
	Observe
	\begin{eqnarray*}
		(x+y) -x-y
		=
		(y+x)-x-y
		=
		y+(x+(-x))-y
		=
		y-y
		=0,
	\end{eqnarray*}
	which implies
	\( -(x+y)=-x-y \).
	Similar argument proves
	the other equality.
	
	\fbox{(i)}
	Just use (4).
	
	\fbox{(j)}
	This is an immediate consequence from the the definition of reciprocal and quotient.
	
	\fbox{(k)}
	\( 1\cdot 1=1 \)
	implies
	\( 1/1 =1 \),
	from which it follows that
	\begin{equation*}
		\frac{x}{1}
		=
		x\cdot \frac{1}{1}
		=
		x \cdot 1
		=
		x.
	\end{equation*}
	
	\fbox{(\(\ell\))}
	Note that the statement is equivalent to each of the following:
	\begin{eqnarray*}
		&&
		\Forall{ x \in \mathbb{R} }
		\Forall{ y \in \mathbb{R}}
		\left[
			x\neq 0
			\wedge
			y \neq	0
			\Rightarrow
			xy \neq 0
			\right]\\
		&\equiv&
		\Forall{ x \in \mathbb{R}}
		\Forall{ y \in \mathbb{R}}
		\left[
			xy = 0
			\Rightarrow
			x= 0
			\vee
			y = 0
			\right]\\
		&\equiv&
		\Forall{ x \in \mathbb{R}}
		\Forall{ y \in \mathbb{R}}
		\left[
			xy = 0
			\Rightarrow
			\left(
			x \neq 0
			\Rightarrow
			y=0
			\right)
			\right].
	\end{eqnarray*}
	Of course, we establish the last statement.
	If
	\( xy=0 \)
	and
	\( x\neq 0 \),
	then we have
	\begin{equation*}
		y
		=
		\left( x\cdot \frac{1}{x} \right)y
		=
		xy \cdot \frac{1}{x}
		=0 \cdot \frac{1}{x}
		=0.
	\end{equation*}
	
	\fbox{(m)}
	Observe
	\begin{equation*}
		yz \cdot \left( \frac{1}{y} \right) \left( \frac{1}{z} \right)
		=
		y
		\left( z \cdot \frac{1}{z} \right) \frac{1}{y}
		=
		y \cdot \frac{1}{y}
		=
		1,
	\end{equation*}
	which implies
	\begin{equation*}
		\frac{1}{yz}
		=
		\left( \frac{1}{y} \right) \left( \frac{1}{z} \right).
	\end{equation*}
	
	\fbox{(n)}
	Use (m).
	
	\fbox{(o)}
	Letting, in (n),
	\( w=z \)
	and
	\( x=y \)
	yields
	\begin{equation*}
		\frac{xz}{yz}
		=
		\frac{x}{y}\cdot \frac{w}{w}
		=
		\frac{x}{y}
	\end{equation*}
	and
	\begin{equation*}
		\frac{wy}{yz}
		=
		\frac{w}{z}
	\end{equation*}
	respectively,
	from which the result follows.
	
	\fbox{(p)}
	Note that
	\begin{equation*}
		\Forall{ x\in \mathbb{R} }
		\left[
			x\neq 0
			\Rightarrow
			\frac{1}{x}\neq 0
			\right]
		\equiv
		\Forall{ x\in \mathbb{R} }
		\left[
			\frac{1}{x} = 0
			\Rightarrow
			x= 0
			\right],
	\end{equation*}
	and that
	the latter is vacuously true.
	
	\fbox{(q)}
	Observe
	\begin{eqnarray*}
		\frac{w}{z}\cdot \frac{z}{w}
		=
		w\cdot \frac{1}{z} \cdot z \frac{1}{w}
		=
		w \cdot \frac{1}{w}
		=
		1,
	\end{eqnarray*}
	from which the result follows.
	
	\fbox{(r)}
	For
	\( x\neq 0 \),
	replacing, in (m), 
	\( y \)
	with
	\( y/x \)
	and
	\( z \)
	with
	\( w/z \)
	allows us to have, along with (q),
	\begin{equation*}
		\frac{x/y}{w/z}
		=
		\left( \frac{1}{y/x} \right)\left( \frac{1}{w/z} \right)
		=
		\frac{1}{\frac{yw}{xz}}
		=
		\frac{xz}{yw}.
	\end{equation*}
	This is also valid for
	\( x=0 \).
	
	\fbox{(s)}
	For
	\( a \neq 0 \),
	setting
	\( w=1 \)
	and
	\( z=a \)
	in (r),
	lets us to obtain
	\begin{equation*}
		\frac{x/y}{1/a}
		=
		\frac{ax}{y}.
	\end{equation*}
	Then, (q) implies
	\begin{equation*}
		\frac{(ax)}{y}
		=
		a \left( \frac{x}{y} \right),
	\end{equation*}
	which is also valid for 
	\( a=0 \).
	
	\fbox{(t)}
	In (o),
	choosing
	\( z=y \)
	and
	\( w=-x \)
	gives
	\begin{equation*}
		\frac{x}{y}
		+
		\frac{(-x)}{y}
		=
		\frac{xy+(-x)y}{y^2}
		=
		0,
	\end{equation*}
	from which it follows that
	\begin{equation*}
		-\left( \frac{x}{y} \right)
		=
		\frac{(-x)}{y}.
	\end{equation*}
	Similar argument establishes the other equality.
	\qed
\end{sol}

\begin{exe}
	Prove the following ''laws of inequalities'' for \( \mathbb{R} \),
	using only axiom (1)--(6)
	along with the result of Exercise 1:
\end{exe}
\begin{enumerate}
	\item
	      \( x>y \)
	      and
	      \( w>z \Rightarrow x+w>y+z\).
	      
	\item
	      \( x>0 \)
	      and
	      \( y>0 \Rightarrow x+y>0\)
	      and
	      \( x \cdot y >0 \).
	      
	\item
	      \(  x>0 \Leftrightarrow -x <0\).
	      
	\item
	      \( x>y \Leftrightarrow -x < -y \).
	      
	\item
	      \( x>y \)
	      and
	      \( z<0 \Rightarrow xz<yz\).
	      
	\item
	      \( x \neq 0 \Rightarrow x^2 >0 \),
	      where \( x^2=x \cdot x \).
	      
	\item
	      \( -1 < 0< 1 \).
	      
	\item
	      \( xy>0 \Leftrightarrow \) \( x \) and \( y \) are both
	      positive or both negative. 
	      
	\item
	      \( x>0 \Rightarrow 1/x >0 \).
	      
	\item
	      \( x>y>0 \Rightarrow 1/x < 1/y \).
	      
	\item
	      \( x<y \Rightarrow x<(x+y)/2<y \).
\end{enumerate}
\begin{sol}\leavevmode \par
	\fbox{(a)}
	Use (6) successively to gain
	\( x+y > y+w > y+z \).
	
	\fbox{(b)}
	The statement immediately follows from (a) and (6) respectively.
	
	\fbox{(c)}
	First, suppose
	\( x>0 \).
	Letting in (6)
	\( y=0 \)
	and
	\( z=-x \)
	yields
	\begin{equation*}
		0 = x+(-x) > -x.
	\end{equation*}
	
	Suppose
	\( -x<0 \).
	It follows trivially from (6) that
	\begin{equation*}
		0 > y 
		\Rightarrow
		z> y+z.
	\end{equation*}
	Taking
	\( y=-x \)
	and
	\( z=x \)
	here proves the remaining implication.
	
	\fbox{(d)}
	Use (6),(c),(h) of the previous exercise, and (6) in order,
	to deduce
	\begin{eqnarray*}
		x>y
		\equiv
		x-y >0
		\equiv
		-(x-y)<0
		\equiv
		-x+y<0
		\equiv
		-x<-y.
	\end{eqnarray*}
	
	\fbox{(e)}
	Since
	\( -z>0 \)
	by (c),
	it follows from (6) that
	\begin{equation*}
		x(-z) > y(-z)
		\equiv
		xz< yz.
	\end{equation*}
	
	\fbox{(f)}
	If
	\( x>0 \),
	then (6) implies 
	\( x \cdot x > 0\).
	If otherwise, say,
	\( x<0 \),
	then
	\( -x>0 \).
	So, it follows from (2),(6), and (d) of the previous exercise that
	\( x \cdot x>0 \).
	
	\fbox{(g)}
	Note that
	\( 1 \neq 0 \).
	So, it follows from (f) that
	\( 1=1 \cdot 1 >0 \).
	Then, (c) implies
	\( -1 <0 \).
	
	\fbox{(h)}
	We claim that
	RHS on the statement of (h) is equivalent to:
	\begin{eqnarray*}
		\left( x>0 \wedge y>0 \right)
		\vee
		\left( x<0 \wedge y<0 \right)
		&\equiv&
		\left[
			\left( x \le 0 \vee y \le 0 \right)
			\Rightarrow
			\left( x<0 \wedge y< 0 \right)
			\right]\\
		&\equiv&
		xy>0
	\end{eqnarray*}
	The first equivalence is a result of logic.
	We need to prove the second.
	But, (\( \Leftarrow \)) part is obvious,
	so we establish (\( \Rightarrow \)) part.
	If
	\( x \le 0 \vee y \le 0 \)
	holds,
	then hypothesis yields 
	\( x<0 \wedge y< 0 \),
	which implies, with the help of (b),
	\begin{equation*}
		xy= (-x)(-y)>0
	\end{equation*}
	as required.
	On the other hand, If 
	\( x \le 0 \vee y \le 0 \)
	does not hold, that is, if
	\( x>0 \wedge y>0 \)
	is the case,
	then (b) again gives the required reslut.
	
	\fbox{(i)}
	Noting \( x>0 \) and
	\begin{equation*}
		x \cdot \frac{1}{x}=1>0,
	\end{equation*}
	apply (h) to get
	\( 1/x>0 \).
	
	\fbox{(j)}
	Observe that
	\( xy>0 \)
	by assumption and (h), and that
	\( 1/(xy) >0 \)
	by (i).
	Then, exploit (o) of the previous exercise and (h) to gain
	\begin{equation*}
		\frac{1}{y} - \frac{1}{x}
		=
		\frac{x-y}{xy}>0.
	\end{equation*}
	
	\fbox{(k)}
	Use (6) successively to obtain
	\( 2x < x+y < 2y \).
	Since
	\( 0<1 \)
	by (g),
	we have that
	\( 2>0 \)
	by (a),
	and that, by (i),
	\( 1/2 >0\),
	which, combined with the first inequality, implies the result.
	\qed\end{sol}

\begin{exe}\leavevmode \par
	\begin{enumerate}
		\item
		      Show that if \( \mathcal{A} \) is a collection of inductive sets,
		      then the intersection of the elements of \( \mathcal{A} \) is a inductive set.
		      
		\item
		      Prove the basic properties of (1) and (2) of \( \mathbb{Z}_{+} \).
	\end{enumerate}
\end{exe}\begin{sol}\leavevmode \par
	\fbox{(a)}
	Obviously
	\( 1 \in \cap \mathcal{A} \).
	the proof then completes by seeing
	\begin{equation*}
		x\in \cap \mathcal{A}
		\equiv
		\Forall{ A\in \mathcal{A}}
		\left[  x \in A \right]
		\Rightarrow
		\Forall{ A\in \mathcal{A}}
		\left[  x+1 \in A \right]
		\equiv
		x+1 \in \cap \mathcal{A}.
	\end{equation*}
	\fbox{(b)}
	\( 1\in \mathbb{Z}_{+} \)
	is a direct consequence of (a).
	To prove principle of induction,
	note that
	\( A \subset \mathbb{Z}_{+}\),
	by definition,
	and that
	\(  \mathbb{Z}_{+} = \cap \mathcal{A} \subset A \)
	since
	\( A \)
	itself is inductive.
	Thus,
	\( A = \mathbb{Z}_{+} \).
	\qed\end{sol}

\begin{exe}\leavevmode \par
	\begin{enumerate}
		\item
		      Prove by induction that given \( n \in \mathbb{Z}_{+} \), 
		      every nonempty subset of \( \left\{ 1,\cdots, n \right\} \) has a largest element.
		      
		\item
		      Explain why you cannot conclude from (a) that
		      every nonempty subset of \( \mathbb{Z}_{+} \)
		      has a largest element.
	\end{enumerate}
\end{exe}\begin{sol}\leavevmode \par
	\fbox{(a)}
	Let 
	\( A \)
	be a subset of
	\( \mathbb{Z}_{+} \)
	such that
	given
	\( n \in A \),
	every nonempty subset of
	\( \left\{ 1,\cdots,n  \right\}\)
	has a largest element.
	It is clear that
	\( 1\in A \).
	Assume
	\(  x \in A \)
	and let
	\( B \subset \left\{ 1,\cdots,x+1 \right\} \).
	We prove
	\(  x+1 \in A \).
	
	If
	\( x+1 \notin B \),
	then
	\( B \)
	possesses a largest element
	since there hold
	\( B \subset \left\{ 1,\cdots,x \right\} \)
	and
	\( x\in A \).
	So, suppose
	\( x+1 \in B \).
	Let
	\( B_0:=B\setminus \left\{ x+1 \right\} \),
	and let
	\( k \) be a largest element of 
	\( B_0 \).
	Then, the existence of a largest element of 
	\( B \)
	is reduced to that of
	\( \left\{ k,x+1 \right\}\).
	But the set admits a largest element since comparability of
	\( \mathbb{Z}_{+} \)
	allows us to have either
	\( k> x+1 \)
	or
	\( k< x+1 \).
	This means 
	\( x+1 \in A \)
	and so, 
	\( A \)
	is inductive.
	Thus,
	\( A = \mathbb{Z}_{+} \).
	
	\fbox{(b)}
	(a) does not deal with the sets like
	\( \mathbb{Z}_{+} \)
	and
	\( \mathbb{Z}_{+} \setminus \left\{ 1,\cdots,n \right\}\).
	\qed\end{sol}

\begin{exe}
	Prove the following properties of \( \mathbb{Z} \) and \( \mathbb{Z}_{+} \):
	\begin{enumerate}
		\item
		      \( a,b \in \mathbb{Z}_{+} \Rightarrow a+b \in \mathbb{Z}_{+}\).
		      
		\item
		      \( a,b \in \mathbb{Z}_{+} \Rightarrow a \cdot b \in \mathbb{Z}_{+}\).
		      
		\item
		      Show that
		      \( a\in \mathbb{Z}_{+} \Rightarrow a-1 \in \mathbb{Z}_{+} \cup \left\{ 0 \right\} \).
		      
		\item
		      \( c,d \in \mathbb{Z} \Rightarrow c+d \in \mathbb{Z} \)
		      and
		      \( c-d \in \mathbb{Z} \).
		      
		\item
		      \( c,d \in \mathbb{Z} \Rightarrow c \cdot d \in \mathbb{Z} \).
	\end{enumerate}
\end{exe}\begin{sol}\leavevmode \par
	\fbox{(a)}
	Given
	\( a \in \mathbb{Z}_{+} \),
	let
	\( X:=\left\{x \in \mathbb{R} \pipe a+x \in \mathbb{Z}_{+} \right\} \).
	The fact that 
	\( \mathbb{Z}_{+} \)
	is inductive implies
	\( a+1 \in \mathbb{Z}_{+} \)
	and so
	\( 1\in X \).
	Exactly the same argument shows that
	if
	\( y \in X \),
	then
	\( y+1 \in X \).
	Hence,
	\( X \)
	is inductive,
	which yields
	\begin{equation*}
		\Forall{ a }
		\Forall{ b }
		\left[
			a\in \mathbb{Z}_{+} \wedge b \in \mathbb{R}
			\Rightarrow
			a+b \in \mathbb{Z}_{+}
			\right].
	\end{equation*}
	Now it is easy to deduce (a) from this.
	
	\fbox{(b)}
	Given
	\( a \in \mathbb{Z}_{+} \),
	let
	\( X:=\left\{x \in \mathbb{Z}_{+} \pipe
	a \cdot	x \in \mathbb{Z}_{+} \right\} \).
	Assume
	\( S_n \subset X \)
	for every
	\( n \in \mathbb{Z}_{+} \)
	where
	\( S_n \)
	is a section of
	\( n \).
	Then,
	we have
	\( n \in X\)
	since
	\( a(n-1)\in \mathbb{Z}_{+} \wedge a \in \mathbb{Z}_{+} \)
	and (a) give
	\( an \in \mathbb{Z}_{+} \).
	Thus,
	\( X \)
	is inductive, which completes the proof.
	
	\fbox{(c)}
	We show the set
	\( X:=	\left\{  x \in \mathbb{Z}_{+} \pipe
	x-1 \in \mathbb{Z}_{+} \cup \left\{ 0 \right\}
	\right\} \)
	is inductive.
	It is clear that
	\( 1\in X \)
	and 
	\( \mathbb{Z}_{+} \cup \left\{ 0 \right\} \)
	is inductive.
	It follows from the latter fact that if
	\( x\in X \),
	then 
	\( x+1 \in X \).
	Thus,
	\( X \)
	is inductive.
	
	\fbox{(d)}
	It suffices to prove that
	\begin{equation}
		c\in \mathbb{Z} \wedge d \in \mathbb{Z}_{+}
		\Rightarrow
		c+d \in \mathbb{Z} \wedge c-d \in \mathbb{Z}.
		\label{eq:sec4E5d}
	\end{equation}
	Note that (a) and (c) establishes the following special case for this:
	\begin{equation*}
		c\in \mathbb{Z}_{+} \wedge d=1
		\Rightarrow
		c+d \in \mathbb{Z}_{+} \wedge c-d \in \mathbb{Z}_{+}\cup \left\{ 0 \right\},
	\end{equation*}
	which yields, by the definition of \( \mathbb{Z} \),
	\begin{equation*}
		c\in \mathbb{Z} \wedge d=1
		\Rightarrow
		c+d \in \mathbb{Z} \wedge c-d \in \mathbb{Z}.
	\end{equation*}
	This implies that if
	\( z\in \mathbb{Z} \),
	then
	\( z+1\in \mathbb{Z} \wedge z-1\in \mathbb{Z} \).
	Hence,
	(\refeq{eq:sec4E5d})
	follows from the induction on
	\( d\).
	
	\fbox{(e)}
	We already know (b), one special case for (e), and the fact
	\( 0\cdot 0 =0 \in \mathbb{Z} \).
	With this in mind, we start with considering another special case:
	\begin{equation*}
		c,d \in \mathbb{Z} \setminus
		\left\{ \mathbb{Z}_{+}\cup \left\{ 0 \right\} \right\}
		\Rightarrow
		c \cdot d \in \mathbb{Z}.
	\end{equation*}
	The proof is easy as it is reduced to the case where (b) works.
	So, it remains to show that
	\begin{equation}
		c \in \mathbb{Z}_{+}
		\wedge
		d\in \mathbb{Z} \setminus
		\left\{ \mathbb{Z}_{+}\cup \left\{ 0 \right\} \right\}
		\Rightarrow
		c \cdot d \in \mathbb{Z}.
		\label{eq:Sec4E5e}
	\end{equation}
	We prove this by induction on \( c \).
	(\refeq{eq:Sec4E5e}) is obviously valid for
	\( c = 1\).
	Now assume (\refeq{eq:Sec4E5e}) holds for \( c = n\).
	We have
	\( d, dn \in \mathbb{Z} \)
	and then (d) gives
	\( d(n+1) = d+dn \in \mathbb{Z} \).
	This means (\refeq{eq:Sec4E5e}) holds for \( c=n+1 \),
	from which we conclude (\refeq{eq:Sec4E5e}).
	\qed\end{sol}

\begin{exe}
	Let \( a \in \mathbb{R} \).
	Define inductively
	\begin{eqnarray*}
		a^1 &=& a,\\
		a^{n+1} &=& a^n \cdot a
	\end{eqnarray*}
	for \( n \in \mathbb{Z}_{+} \).
	Show that for every \( n,m \in \mathbb{Z}_{+} \) and 
	\( a,b \in \mathbb{Z}_{+} \),
	\begin{eqnarray}
		a^{n}a^{m} &=& a^{n+m}\label{eq1:sec4E6},\\
		(a^{n})^{m} &=& a^{nm} \label{eq2:sec4E6},\\
		a^{m}b^{m}&=& (ab)^{m}\label{eq3:sec4E6}.
	\end{eqnarray}
	These are called the \textbf{\textit{laws of exponents}}.
\end{exe}\begin{sol}
	Note that every term that appears in (\ref{eq1:sec4E6})--(\ref{eq3:sec4E6}) these equality is well-defined by induction.
	(We need principle of recursive definition to define it rigorously. See \S7, \S8 for the principle on the set of positive integers.)
	
	First, we prove (\ref{eq1:sec4E6}).
	Given 
	\( a\in \mathbb{R} \)
	and
	\( n \in \mathbb{Z}_{+} \),
	let
	\begin{equation*}
		X:=\left\{ x\in \mathbb{Z}_{+} \pipe a^{n}a^{x}=a^{n+x}\right\}.
	\end{equation*}
	It follows from the definition of
	\( a^{n} \)
	that
	\( X \)
	is inductive.
	Thus, (\ref{eq1:sec4E6}) follows.
	
	We proceed to verifying (\ref{eq2:sec4E6}).
	For 
	\( a\in \mathbb{R} \)
	and
	\( n \in \mathbb{Z}_{+} \),
	set
	\begin{equation*}
		Y:=\left\{ y\in \mathbb{Z}_{+} \pipe (a^{n})^{y}=a^{ny} \right\}.
	\end{equation*}
	It is easy to check that
	\( Y \)
	is inductive, from which we conclude (\ref{eq2:sec4E6}).
	
	We establish (\ref{eq3:sec4E6}).
	Let
	\(  a,b \in \mathbb{R} \)
	and let
	\begin{equation*}
		W:=\left\{ m\in \mathbb{Z}_{+}
		\pipe a^{m}b^{m}=(ab)^{m} \right\}.
	\end{equation*}
	It suffices to show that 
	\( W \) is inductive.
	Observe
	\( 1\in W \)
	and assume
	\( m \in W \),
	from which we deduce that
	\begin{equation*}
		a^{m+1}b^{m+1}
		=
		a (a^{m}b^{m})b
		=
		(ab)^{m}(ab)
		=
		(ab)^{m+1},
	\end{equation*}
	that is,
	\( m+1 \in W \).
	Hence, \( W \) is inductive.
	\qed\end{sol}

\begin{exe}
	Let \( a \in \mathbb{R} \) and \( a\neq 0 \).
	Define \( a^0 = 1 \),
	and for \( n\in \mathbb{Z}_{+} \),
	\( a^{-n}= 1/{a^n} \).
	Show that the laws of exponents hold for \( a,b\neq 0 \) and \( n,m \in \mathbb{Z}_{+} \).
\end{exe}\begin{sol}
	Let
	\( a,b \in \mathbb{R} \)
	and
	\(a \neq 0, b \neq 0 \)
	throughout this exercise,
	and let us simply mention (\ref{eq1:sec4E6}), for instance, 
	to mean the corresponding statement we are concerned here.
	
	Note that, for \( n \in \mathbb{Z}_{+} \),
	we have 
	\begin{equation*}
		a^{-n} = \frac{1}{a^n} = \left( \frac{1}{a} \right)^n.
	\end{equation*}
	
	We first prove (\ref{eq3:sec4E6}) for convenience.
	It is obvious that (\ref{eq3:sec4E6}) holds for
	\( m =0 \).
	For
	\( -m \in \mathbb{Z}_{+}\),
	previous exercise gives
	\begin{equation*}
		a^{m}b^{m}
		=
		\left( \frac{1}{a} \right)^{-m} \left( \frac{1}{b} \right)^{-m}
		=
		\left( \frac{1}{ab} \right)^{-m}
		=
		(ab)^{m}.
	\end{equation*}
	Thus, we complete the proof of (\ref{eq3:sec4E6}).
	
	Next, we show (\ref{eq2:sec4E6}).
	Since Exercise 6 yields
	\( 1^{\ell}=1 \)
	for all 
	\( \ell \in \mathbb{Z}_{+} \),
	we deduce that (\ref{eq2:sec4E6}) is valid for the case where
	either
	\( n =0 \)
	or
	\( m =0 \)
	holds.
	For
	\( -n, -m \in \mathbb{Z}_{+} \),
	it follows from Exercise 6 and Exercise 1(h) that
	\begin{equation*}
		(a^n)^m
		=
		\left( \frac{1}{\frac{1}{a^{-n}}} \right)^{-m}
		=
		\left( a^{-n} \right)^{-m}
		=
		a^{(-n)(-m)}
		=
		a^{nm}.
	\end{equation*}
	So, (\ref{eq2:sec4E6}) holds for \( -n, -m \in \mathbb{Z}_{+} \).
	One remaining case is where
	\( n \in \mathbb{Z}_{+} \)
	and
	\( -m \in \mathbb{Z}_{+} \).
	In this case,
	we have
	\begin{equation*}
		(a^n)^m
		=
		\left( \frac{1}{a^{n}} \right)^{-m}
		=
		\left( \left( \frac{1}{a} \right)^{n} \right)^{-m}
		=
		\left( \frac{1}{a} \right)^{n(-m)}
		=
		\left( \frac{1}{a} \right)^{-(nm)}
		=
		a^{nm},
	\end{equation*}
	as required.
	Confirm that similar argument applied to the case
	\( m \in \mathbb{Z}_{+} \),
	\( -n \in \mathbb{Z}_{+} \)
	also leads us to the required equation.
	Thus, (\ref{eq2:sec4E6}) is proved.
	
	Lastly, We turn to (\ref{eq1:sec4E6}).
	For
	\( n=0 \)
	and
	\( m=0 \),
	(\ref{eq1:sec4E6})
	is trivial.
	For
	\( -n, -m \in \mathbb{Z}_{+} \),
	it is easy to see that
	\begin{equation*}
		a^n a^m
		=
		\left( \frac{1}{a} \right)^{-n} \left( \frac{1}{a} \right)^{-m}
		=
		\left( \frac{1}{a} \right)^{-(n+m)}
		=
		a^{n+m}.
	\end{equation*}
	Thanks to the symmetry of 
	\( n \)
	and
	\( m \),
	we only have to consider the case
	\( n \in \mathbb{Z}_{+} \),
	\( -m \in \mathbb{Z}_{+} \).
	If 
	\( n\ge -m \),
	then
	\begin{equation*}
		a^n a^m
		=
		a^{n+m}a^{-m}a^{m}
		=
		a^{n+m}a^{0}
		=
		a^{n+m}.
	\end{equation*}
	It is now obvious how to prove if
	\( n < -m \).
	\qed\end{sol}

We briefly establish two useful inequalities we often use in real analysis.
We exploit these inequalities in Exercise 8 and
\refer{Proposition}{prop:supremum},
which is in turn used in a direct proof for existence of squared root (Exercise 10).
\begin{lem}\label{lem:ineq_epsilon}
	Let \( x \) and \( y \) be real numbers.
	We claim that there hold
	\begin{equation}\label{ineq:sec4ep1}
		\Forall{ \epsilon >0 }\left[ x< y+ \epsilon \right]
		\equiv
		x \le y,
	\end{equation}
	and 
	\begin{equation}\label{ineq:sec4ep2}
		x \ge 0
		\Rightarrow
		\left[
			\Forall{ \epsilon >0 }\left[ x <\epsilon \right]
			\equiv
			x=0
			\right].
	\end{equation}
\end{lem}
\begin{prf}
	First consider (\ref{ineq:sec4ep1}),
	which is equivalent to
	\begin{equation*}
		\Exists{ \epsilon >0 }\left[ x\ge y+ \epsilon \right]
		\equiv
		x > y.
	\end{equation*}
	This is true since LHS obviously implies RHS, and choosing \( \epsilon:=(x-y)/2 \) proves the converse.
	(\ref{ineq:sec4ep2}) follows by setting \( y:=0 \) at (\ref{ineq:sec4ep1}).
\end{prf}

\begin{exe}\leavevmode \par
	\begin{enumerate}
		\item
		      Show that \( \mathbb{R} \) has the greatest lower bound property.
		      
		\item
		      Show that \( \inf{\left\{ 1/n \pipe n \in \mathbb{Z}_{+} \right\}}=0 \).
		      
		\item
		      Show that given \( a \) with \( 0<a<1 \),
		      \( \inf{\left\{ a^n \pipe n \in \mathbb{Z}_{+} \right\}}=0 \).
	\end{enumerate}
\end{exe}\begin{sol}\leavevmode \par
	\fbox{(a)}
	Use (7) and \S3 Exercise 14.
	
	\fbox{(b)}
	Let
	\( N:= \left\{ 1/n \pipe n \in \mathbb{Z}_{+}  \right\} \).
	Note that \( 0 \) is a lower bound for \( N \),
	and so \( \inf{N} \ge 0\).
	In light of \refer{Lemma}{lem:ineq_epsilon}
	proof is reduced to
	\begin{equation*}
		\Forall{ \epsilon > 0  }\left[ \inf{N} < \epsilon \right].
	\end{equation*}
	To this end, it suffices to prove that
	\begin{equation*}
		\Forall{ \epsilon > 0 }
		\Exists{ n \in \mathbb{Z}_{+} }
		\left[ \frac{1}{n} < \epsilon \right].
	\end{equation*}
	Fix 
	\( \epsilon > 0  \).
	Archimedean ordering property(see \refer{Note}{Archimedean}) allows us to have
	\( n \in \mathbb{Z}_{+} \)
	such that
	\( 1/\epsilon < n \),
	that is,
	\( 1/n < \epsilon \).
	
	\fbox{(c)}
	It is easy to deduce from binomial expansion that
	\begin{equation*}
		(1+h)^{n} \ge 1+ nh
	\end{equation*}
	for every
	\( h>0 \)
	and
	\( n \in \mathbb{Z}_{+} \).
	Choosing
	\( h= (1-a)/a \)
	yields
	\begin{equation*}
		0 \le a^n \le \frac{a}{a + n - an} < \frac{1}{n}.
	\end{equation*}
	Since it is obvious that
	\( \inf{a_n} \ge 0 \),
	it remains to show 
	\( \inf{a_n} \le \inf{(1/n)} \),
	that is,
	monotonicity property of infimum.
	We establish this in \refer{Proposition}{prop:supremum}.
	\qed\end{sol}

\begin{prp}[Property of supremum and infimum]\label{prop:supremum}
	\leavevmode \par
	\begin{enumerate}
		\item \label{enu:sup_mono}
		      Let
		      \( A:=\left\{ a_n \in \mathbb{R} \pipe n \in \mathbb{Z}_{+} \right\} \)
		      and
		      \( B:=\left\{ b_n \in \mathbb{R} \pipe n \in \mathbb{Z}_{+} \right\} \).
		      Suppose that
		      \( \sup{A} \)
		      and
		      \( \inf{B} \)
		      and etc.\! exist,
		      and that
		      \( a_n \le b_n \)
		      for all \( n \in \mathbb{Z}_{+} \).
		      Then we have
		      \begin{equation*}
			      \sup{A} \le \sup{B},\;\; \inf{A} \le \inf{B}.
		      \end{equation*}
		\item \label{enu:sup:scale}
		      Let
		      \( c>0 \), and
		      let \( X \) and \( Y \) be nonempty subsets of \( \mathbb{R} \)
		      for which \( \sup{X} \) and \( \inf{Y} \) etc.\! exist.
		      Then we have
		      \begin{equation*}
			      \sup{cX}=c \cdot \sup{X},\;\;\inf{cY}=c \cdot \inf{Y},
		      \end{equation*}
		      where \( cX:=\left\{ c \cdot x \pipe x\in X \right\} \).
	\end{enumerate}
\end{prp}
\begin{prf}
	We show \ref{enu:sup_mono}.
	Note that, in general, there holds
	\( \inf{A} \le a_n \)
	for all \( n \in \mathbb{Z}_{+} \),
	which combined with assumption gives
	\( \inf{A} \le b_n \)
	for all
	\( n \in \mathbb{Z}_{+} \).
	This means
	\( \inf{A} \)
	is a lower bound for
	\( B \),
	from which we conclude that
	\( \inf{A} \le \inf {B} \).
	Similar argument verifies the other implication.
	
	Consider \ref{enu:sup:scale}.
	Observe
	\( cX \)
	is bounded above by
	\( c \cdot \sup{X} \).
	Hence,
	\( \sup{cX} \le c \cdot \sup{X} \).
	For every
	\( \epsilon>0 \)
	there exists 
	\( x\in X \)
	such that
	\( \sup{cX} - \epsilon < cx \),
	or equivalently,
	\( \left( \sup{cX} - \epsilon \right)/c <x \),
	from which it follows that
	\( \left( \sup{cX} - \epsilon \right)/c <\sup{X} \),
	that is,
	\( \sup{cX}<c \cdot \sup{X} + \epsilon\).
	This implies that
	\( \sup{cX} \le c \cdot \sup{X}\).
\end{prf}

\begin{exe}\leavevmode \par
	\begin{enumerate}
		\item
		      Show that every nonempty subset of \( \mathbb{Z} \) that is bounded above has a largest element.
		      
		\item
		      If \( x\notin \mathbb{Z} \),
		      show there is exactly one \( n \in \mathbb{Z} \) such that \( n<x<n+1 \).
		      
		\item
		      If \( x-y>1 \),
		      show there is at least one \( n \in \mathbb{Z} \) such that \( y<n<x \).
		      
		\item
		      If \( y<x \),
		      show there is a rational number \( z \) such that \( y<z<x \).
	\end{enumerate}
\end{exe}\begin{sol}\leavevmode \par
	\fbox{(a)}
	Let
	\( A \) be a nonempty subset of \( \mathbb{Z} \)
	that is bounded above by \( N \in \mathbb{Z}_{+} \).
	Assume first that \( A \cap \mathbb{Z}_{+} \neq \emptyset \).
	Exercise 4(a) allows us to choose a largest element \( m \) of
	\( \left\{ 1,\cdots, N \right\} \cap A \).
	It is obvious that \( m \) is a largest element of \( A \).
	
	For general \( A \),
	consider an order isomorphism
	\( f:\mathbb{Z} \ni n \to n +K \in \mathbb{Z} \)
	for sufficiently large \( K \in \mathbb{Z}_{+} \),
	so that \( f(A) \cap \mathbb{Z}_{+} \neq \emptyset \).
	Then the preceding argument guarantees
	the existence of a largest element of \( f(A) \),
	that is, that of \( A\).
	
	\fbox{(b)}
	Let 
	\( x \notin \mathbb{Z} \).
	The set 
	\( B:=\left\{ b \in \mathbb{Z} \pipe b <x \right\} \)
	is nonempty and bounded above.
	(a) implies that
	\( B \) possesses a largest element
	\( n \)
	of
	\( B \),
	for which we have
	\( n< x \le n+1 \).
	The required inequality follows from the fact
	\( x \notin \mathbb{Z} \).
	Uniqueness of such
	\( n \)
	follows from the property of 
	\( n \)
	as a largest element.
	
	\fbox{(c)}
	Let
	\( x-y>1 \).
	If
	\(  x \in \mathbb{Z} \)
	is the case,
	then
	\( n:=x-1 \)
	does the job.
	So, assume
	\(  x \notin \mathbb{Z} \).
	In that case, by (b),
	there exists
	\( n \)
	such that
	\( n<x<n+1 \).
	Since
	\( y+1<x<n+1 \),
	we deduce that
	\( y<n \).
	Thus,
	\( y<n<x \).
	
	\fbox{(d)}
	In light of (c),
	we may assume that
	\( 0 < x-y \le 1 \).
	Archimedean ordering property(see \refer{Note}{Archimedean}) lets us pick
	\( m\in \mathbb{Z}_{+} \)
	such that
	\( 1/(x-y) < m \),
	for which we have
	\( mx-my>1 \).
	Then, (c) gives
	\( n \in \mathbb{Z} \)
	such that 
	\( my< n< mx \),
	that is,
	\( y<n/m<x \).
	This completes the proof.
	\qed\end{sol}

\begin{exe}\leavevmode \par
	Show that every positive number \( a \) has exactly one positive square root,
	as follows:
	\begin{enumerate}
		\item
		      Show that if \( x>0 \)
		      and
		      \( 0 \le h <1 \),
		      then
		      \begin{eqnarray*}
			      (x+h)^2 &\le& x^2 +h(2x+1),\\
			      (x-h)^2 &\le& x^2 -h(2x).
		      \end{eqnarray*}
		      
		\item
		      Let \( x>0 \).
		      Show that if \( x^2<a \),
		      then
		      \( (x+h)^2<a \) for some \( h>0 \);
		      and if if \( x^2>a \),
		      then
		      \( (x-h)^2>a \) for some \( h>0 \).
		      
		\item
		      Given \( a>0 \),
		      let \( B \) be the set of all real numbers \( x \) such that \( x^2 <a \).
		      Show that \( B \) is bounded above and contains at least one positive number.
		      Let \( b=\sup{B} \);
		      show that \( b^2 = a \).
		      
		\item
		      Show that if \( b \) and \( c \) are positive and \( b^2 = c^2 \),
		      then \( b=c \).
	\end{enumerate}
\end{exe}\begin{sol}\leavevmode \par
	\fbox{(a)}
	Consider binomial expansion as we did in exercise 8(c).
	note that both inequalities are valid for \( h=1\).
	
	\fbox{(b)}
	Let
	\( x>0 \)
	and
	\( x^2<a \).
	If
	\( (x+1)^2 <a\),
	then
	\( h:=1 \)
	qualifies.
	If, on the other hand,
	\( (x+1)^2 \ge a\),
	then we have
	\begin{equation*}
		0 < \frac{a-x^2}{2x+1} \le 1.
	\end{equation*}
	In this case, exercise 9(d) allows us to have a rational number
	\( r \)
	such that
	\begin{equation*}
		0< r< \frac{a-x^2}{2x+1},
	\end{equation*}
	which leads us to
	\( (x+r)^2<a \).
	
	To show the other implication, check that
	\begin{equation*}
		\frac{x^2-a}{2x}<1
	\end{equation*}
	holds if
	\( (x-1)^2 \le a\),
	and choose a rational number
	\( q \) with
	\begin{equation*}
		\frac{x^2-a}{2x}< q <1.
	\end{equation*}
	
	\fbox{(c)}
	Let
	\( a>0\).
	Observe
	\( B:=\left\{  x \in \mathbb{R} \pipe x^2 <a \right\} \).
	Since
	\( 0^2=0<a \),
	it follows from (b) that
	\( h^2 <a \)
	for some
	\( h>0 \),
	which gives
	\( h \in B \),
	and
	\( B_0:=B \cap \mathbb{R}_{+} \neq \emptyset \).
	
	We claim that
	\( B_0 \)
	(and so \( B \))
	is bounded above.
	Let \(  x \in B_0 \).
	If
	\( a\le1 \),
	then
	\( x^2 < 1 \)
	and so
	\( x < 1 \).
	Next, if
	\( a>1 \),
	then
	we see that for every
	\( x > 1 \)
	we have
	\( x<x^2<a \).
	Thus, \( B_0 \)
	is bounded above.
	Note that we have also shown that  \( b>0 \).
	
	For every
	\( h>0 \),
	we have
	\( b+h \notin B_0 \)
	and so
	\( (b+h)^2 \ge a \),
	which yields, by (b),
	\( b^2 \ge a \).
	
	Suppose, to the contrary, that
	\( b^2 >a \)
	were the case,
	then (b) lets us to pick
	\( h>0 \)
	such that
	\( (b+h)^2 < a \).
	But this contradicts the fact that
	\( b \)
	is a supremum of
	\( B \).
	Thus,
	\( b^2 =a \).
	(We have not used here the second statement of (b).)
	
	We have shown \( b^2 \le a \) resorting to proof by contradiction since it is straightforward and depends on less information than direct proof.
	
	We provide two notes that proves the fact
	\( b^2 \le a \)
	by exploiting (b) or property of supremum.
	In the first approach, we have to look a bit more into (b) in order to gain stronger result of it.
	See \refer{Note}{note:root1}
	and
	\refer{Note}{note:root2}.
	
	\fbox{(d)}
	Hypothesis is equivalent to 
	\( (b-c)(b+c)=0 \).
	If
	\( b \)
	and
	\( c \)
	are positive,
	then
	\( b-c =0 \).
	\qed\end{sol}

\begin{rem}[First direct proof for existence of square root]\label{note:root1}
	As predicted, we begin with extracting more information from exercise 10(b) in order to construct short direct proof for exercise10(c), existence of square root.
	
	We provide a proposition, whose contraposition is used for our main proof:
	\begin{prp}\label{cnt:quadratic}
		Let
		\( x>0 \)
		and
		\( \epsilon > 0\).
		If \( x^2 < a\),
		then
		\( (x+h)^2 < a\)
		for some \( h \in (0,\epsilon) \);
		and if \( x^2 > a\),
		then
		\( (x-h)^2 > a\)
		for some \( h \in (0,\epsilon) \).
	\end{prp}
	\begin{prf}
		As a preparation, first observe the following very intuitive fact:\\
		\textit{The mapping
			\( f:\mathbb{R}_{+}\cup \{0\} \ni x \mapsto x^2 \in \mathbb{R}_{+}\cup \{ 0 \} \)
			preserves order.}\\
		Note that this is a direct consequence of exercise 2(h).
		
		Suppose
		\( x^2 < a\).
		Exercise 10(b) gives some
		\( d>0 \)
		such that
		\( (x+d)^2 < a\).
		Let
		\( \epsilon_0 := \min\{\epsilon, d\} \).
		It follows from the fact the mapping
		\( f \)
		defined above preserves the order that
		for every 
		\( h \in (0,\epsilon_0) \)
		we have
		\( f(x+h)<f(x+d) \),
		that is,
		\( (x+h)^2<(x+d)^2<a \).
		Similar argument works for verifying the other implication.
	\end{prf}
	Note that the proof says more than Proposition \ref{cnt:quadratic} states,
	but for our proof for exercise 10(c), we do not need further result.
	More general result can be found in theory of continuous function.
	
	Now we provide a predicted direct proof for existence of squared root.
	It suffices to show that 
	\( b^2 \le a \).
	Let
	\( \epsilon:=\min \left\{ b,1 \right\} \).
	For every
	\( h \in (0,\epsilon) \),
	there exists
	\( x \in B_0 \)
	such that
	\( 0< b-h<x \),
	which yields
	\( (b-h)^2<x^2<a \).
	Then Proposition \ref{cnt:quadratic} implies
	\( b^2 \le a \).
\end{rem}

\begin{rem}[Second direct proof for existence of square root]\label{note:root2}
	In the second approach,
	we do not need such finer information as Proposition \ref{cnt:quadratic}.
	We instead need an almost obvious property of supremum, which we have established in \refer{Proposition}{prop:supremum} and an obvious result of exercise 10(b) stated as follows:\leavevmode \par
	\noindent
	\textit{Let \( x>0 \).
		If \( x^2 > a\),
		then
		\( (x-h)^2 > a\)
		for some \( 0<h\le 1\).
	}
	
	We establish
	\( b^2 = a \).
	Suppose
	\( a>1 \).
	We already know that
	\( B_0 \) is bounded above by \( a \),
	which gives
	\( b\le a \).
	This implies that if
	\( b \le 1 \),
	then,
	\( b^2 \le b \le a \).
	If \( b>1 \),
	then for every
	\( 0<h\le1 \),
	there exists
	\(  x \in B_0 \)
	such that
	\( 0< b-h <x \),
	from which it follows that
	\( 0< (b-h)^2<a \).
	By the italic statement, this yields
	\( b^2 \le a \).
	Thus \( b^2 =a \) for this case.
	
	Next, Suppose
	\( a \le 1 \).
	In this case,
	Archimedean ordering property(see \refer{Note}{Archimedean}) allows us to choose
	\( k \in \mathbb{Z}_{+} \)
	such that 
	\( ak^2 >1 \).
	Define the sets
	\( B_k:=\left\{  x \in \mathbb{R} \pipe x^2 < ak^2 \right\} \)
	and
	\( kB:=\left\{ kx \pipe x \in B \right\} \).
	For a fixed \( k \in \mathbb{Z}_{+} \), we see 
	\( B_k = kB \),
	from which we deduce that
	\begin{equation*}
		k^2b^2
		=
		\left( k \cdot \sup{B} \right)^2
		=
		\left( \sup{kB} \right)^2
		=
		\left( \sup{B_k} \right)^2
		=
		ak^2.
	\end{equation*}
	Thus, \( b^2 =a \).
	\qed\end{rem}

\begin{exe}
	Given \( m \in \mathbb{Z} \),
	we say that \( m \) is \textbf{\textit{even}} if \( m/2 \in \mathbb{Z} \),
	and \( m \) is \textbf{\textit{odd}} if otherwise.
	
	\begin{enumerate}
		\item
		      Show that if \( m \) is odd,
		      \( m=2n+1 \) for some \( n\in \mathbb{Z} \).
		      
		\item
		      Show that if \( p \) and \( q \) are odd, so are \( p \cdot q \) and \( p^n \),
		      for any \( n \in \mathbb{Z}_{+} \).
		      
		\item
		      Show that if \( a>0 \)\ is rational,
		      then \( a=m/n \) for some \( m,n \in \mathbb{Z}_{+} \)
		      where not both \( m \) and \( n \) are even.
		      
		\item
		      \textit{Theorem.}\;
		      \( \sqrt{2} \) \textit{is irrational}.
	\end{enumerate}
\end{exe}\begin{sol}\leavevmode \par
	\fbox{(a)}
	If
	\( m/2 \notin \mathbb{Z} \),
	then Exercise 9(d) allows us to take
	\( n \) such that
	\( n < m/2 <n+1 \),
	that is,
	\( 2n < m <2n+2 \).
	Thus,
	\( m=2n+1 \).
	
	\fbox{(b)}
	Note that
	\begin{equation*}
		\frac{m}{2}\in \mathbb{Z}
		\equiv
		\Exists{ n \in \mathbb{Z} }
		\left[ m=2n \right]
	\end{equation*}
	and that this equivalence establishes the converse of (a).
	Let
	\( p \)
	and
	\( q \)
	are odd.
	Choose
	\( k,\ell \in \mathbb{Z}\)
	so that
	\( p=2k+1 \),
	\( q=2\ell +1 \).
	Then, we have
	\( p \cdot q = 2(2k \ell + k + \ell) +1 \).
	Since 
	\( 2k \ell + k + \ell \in \mathbb{Z} \),
	we conclude that
	\( p \cdot q \)
	is odd.
	
	Let \( A \) be the subset of \( \mathbb{Z}_{+} \) consisting of all \( n \)
	for which
	\( p^n \)
	is odd for every.
	It is obvious that
	\( 1 \in A \)
	and that if
	\( n \in A \),
	then
	\( p^{n+1}=p^n \cdot p \)
	is odd by what we have just verified.
	This means \( A \) is inductive, which completes the proof.
	
	\fbox{(c)}
	Let 
	\( a>0 \)
	be a rational.
	By definition,
	\( a=\ell /k \)
	for some
	\( k, \ell \in \mathbb{Z} \)
	with
	\( k \neq 0 \).
	If both
	\( k\)
	and
	\( \ell \)
	are even,
	then there exists 
	\( k',\ell' \in \mathbb{Z}\)
	with
	\( k'<k \),
	\( \ell' < \ell \)
	and
	\( a={\ell'}/{k'} \).
	
	Let \( n \) be a smallest element of the set
	\(	A:=\left\{ x\in \mathbb{Z}_{+} \pipe
	\Exists{ m\in \mathbb{Z}_{+} }
	\left[ a=\frac{n}{m} \right]
	\right\} \),
	and let
	\( a=n/m \).
	Because of the smallest property,
	we cannot have a smaller
	\( n' \)
	or
	\( m' \)
	than
	\( n \)
	or
	\( m \)
	respectively.
	Thus,
	both
	\( m \)
	and
	\( n \)
	are not even.
	
	\fbox{(d)}
	Observe the following equivalence:
	\begin{eqnarray}
		\sqrt{2}\in \mathbb{Q}
		&\equiv&
		\Exists{ n\in \mathbb{Z}_{+} }
		\Exists{ m \in \mathbb{Z}_{+} }
		\left[
			2nn=mm 
			\wedge
			\neg
			\left(
			\frac{n}{2} \in \mathbb{Z}_{+}
			\wedge
			\frac{m}{2} \in \mathbb{Z}_{+}
			\right)
			\right] \label{equiv1:sec4E11}\\
		&\equiv&
		\Exists{ n\in \mathbb{Z}_{+} }
		\Exists{ m \in \mathbb{Z}_{+} }
		\left[
			2nn=mm 
			\wedge
			\frac{n}{2} \notin \mathbb{Z}_{+}
			\wedge
			\frac{m}{2} \in \mathbb{Z}_{+}
			\right]\label{equiv2:sec4E11}
	\end{eqnarray}
	For second equivalence, note that,
	\( m \)
	needs to be even since
	\( 2nn \)
	are even, from which it follows that
	\( n \)
	are odd.
	Each of three statements turns out to be false since, in (\ref{equiv2:sec4E11}),
	we have
	\( 2nn/4 \notin \mathbb{Z}_{+} \)
	while
	\( mm/4 \in \mathbb{Z}_{+} \)
	and hence
	\( 2nn \neq mm \).
	Thus, the statement \( \sqrt{2}\in \mathbb{Q}\) is false.
	
	Put in a language of number theory,
	we can simply argue in (\ref{equiv1:sec4E11}) that
	\( 2nn \)
	has odd number of prime factors
	whereas
	\( mm \)
	has even,
	which implies \( 2nn\neq mm \).
	\qed\end{sol}
\end{document}