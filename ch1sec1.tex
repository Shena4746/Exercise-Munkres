\documentclass[a4paper,12pt]{article}
\usepackage{mystyle}
\usepackage{commands}
\mathtoolsset{showonlyrefs=true}
\begin{document}
\section{Fundamental Concepts}
%\setcounter{exe}{0}

\begin{exe}
	Check the distribution laws for \( \cup \) and \( \cap \) and
	DeMorgan's laws.
\end{exe}\begin{sol}%1
	These laws are simple consequence of the corresponding laws for \( \wedge,\vee \), as follows:
	\begin{eqnarray*}
		A \cap (B \cup C)
		&=&
		\left\{ x \pipe x\in A \wedge (x \in B \vee  x \in C) \right\}\\
		&=&
		\left\{ x \pipe (x \in A \wedge x \in B)
		\vee (x \in A \wedge x \in C)
		\right\}\\
		&=&
		(A \cap B) \cup (A \cap C),
	\end{eqnarray*}
	\begin{eqnarray*}
		A \cup (B \cap C)
		&=&
		\left\{ x \pipe x\in A \vee (x \in B \wedge  x \in C) \right\}\\
		&=&
		\left\{ x \pipe (x \in A \vee x \in B)
		\wedge (x \in A \vee x \in C)
		\right\}\\
		&=&
		(A \cup B) \cap (A \cup C),
	\end{eqnarray*}
	\begin{eqnarray*}
		A \setminus (B \cup C)
		&=&
		\left\{ x \pipe x\in A \wedge \neg(x \in B \vee x\in C) \right\}\\
		&=&
		\left\{ x \pipe x \in A \wedge (x \notin B \wedge x \notin C)
		\right\}\\
		&=&
		\left\{ x \pipe (x \in A \wedge x \notin B)
		\wedge
		(x \in A \wedge x \notin C)
		\right\}\\
		&=&
		(A \setminus B) \cap (A \setminus C),
	\end{eqnarray*}
	\begin{eqnarray*}
		A \setminus (B \cap C)
		&=&
		\left\{ x \pipe x\in A \wedge \neg(x \in B \wedge x\in C) \right\}\\
		&=&
		\left\{ x \pipe x \in A \wedge (x \notin B \vee x \notin C)
		\right\}\\
		&=&
		\left\{ x \pipe (x \in A \wedge x \notin B)
		\vee
		(x \in A \wedge x \notin C)
		\right\}\\
		&=&
		(A \setminus B) \cup (A \setminus C).
	\end{eqnarray*}
	\qed\end{sol}

\begin{prp}[Basic property of set operation]\label{prop:set_operation}
	\leavevmode \par \noindent
	Let \( A, B, C, D, \) etc be sets. We have the following fact.
	\begin{enumerate}
		\item \label{enu:set_basic_cup}
		      \( A \subset A \cup B \), \( B \subset A \cup B \), \( B \cup A = A \cup B \).
		      
		\item \label{enu:set_basic_cap}
		      \(A \cap B \subset A \), \(A \cap B \subset B \), \( B \cap A = A \cap B \).
		      
		\item \label{enu:set_basic_cup_bound}
		      \( A \subset C \wedge B \subset C \equiv A\cup B \subset C \).
		      
		\item \label{enu:set_basic_cap_bound}
		      \( D \subset A \wedge D \subset B \equiv D \subset A \cap B \).
		      
		\item \label{enu:set_basic_setminus_smaller}
		      \( A \setminus B \subset A \).
		      
		\item \label{enu:set_basic_trivial_cup}
		      \( A\cup A =A \).
		      
		\item \label{enu:set_basic_empty_cup}
		      \( A \cup \emptyset =A \).
		      
		\item \label{enu:set_basic_trivial_cap}
		      \( A \cap A =A \).
		      
		\item \label{enu:set_basic_empty_cap}
		      \( A \cap \emptyset = \emptyset\).
		      
		\item \label{enu:set_basic_setminus_itself}
		      \( A \setminus A = \emptyset \).
		      
		\item \label{enu:set_basic_itself_minus_empty}
		      \( A \setminus \emptyset = A \).
		      
		\item \label{enu:set_basic_empty_minus_itself}
		      \( \emptyset \setminus A = \emptyset \).
		      
		\item \label{enu:set_basic_decomposition}
		      \( A = (A \setminus B) \cup (A \cap B) \).
		      
		\item \label{enu:set_basic_cup_decomposition}
		      \( A \cup B = (A\setminus B) \cup B \).
		      
		\item \label{enu:set_basic_setminus_disjoint}
		      \( B \cap (A \setminus B) = \emptyset \).
		      
		\item \label{enu:set_basic_bigger_setminus}
		      \( A' \subset A \Rightarrow A' \setminus B \subset A \setminus B\).
		      
		\item \label{enu:set_basic_setminus_bigger}
		      \( B' \subset B \Rightarrow A \setminus B \subset A \setminus B'\).
		      
		\item \label{enu:set_basic_equiv_disjoint}
		      \( A \setminus B =A \equiv A\cap B = \emptyset \).
		      
		\item \label{enu:set_basic_equiv_inclusion}
		      (i)-(v) below are all equivalent to \( A\subset B \).
		      \begin{enumerate}
			      \item \( A\cup B =B \).
			      \item \( A\cap B =A \).
			      \item \( A \setminus B = \emptyset \).
			      \item \( A\cup (B\setminus A)=B \).
			      \item \( A= B \setminus(B \setminus A) \).
		      \end{enumerate}
		\item \label{enu:set_basic_cup_setminus}
		      \( (A\cup B)\setminus C = (A \setminus C) \cup (B \setminus C) \).
		      
		\item \label{enu:set_basic_cap_setminus}
		      \( (A\cap B)\setminus C = (A \setminus C) \cap (B \setminus C) \).
		      
		\item \label{enu:set_basic_times_cap}
		      \( (A\times B) \cap (C \times D) = (A \cap C) \times (B \cap D) \).
		      
		      %\item \( A \times (B \setminus C) = (A \times B) \setminus (A \times C) \).
		      
		\item \label{enu:set_basic_setminus_times}
		      \( (A \setminus B) \times  (C \setminus D) = (A \times C \setminus B \times C) \setminus A \times D \).
	\end{enumerate}
\end{prp}
\begin{prf}
	\ref{enu:set_basic_cup} and \ref{enu:set_basic_cap}
	are obvious from the property of \( \wedge,\vee \).
	
	\ref{enu:set_basic_cup_bound}
	(\( \Leftarrow \)) part is obvious by \ref{enu:set_basic_cup} .
	Conversely, if \( x\in A\cup B \), then \( x\in A \vee  x \in B \),
	but in any case we have \(  x \in C \).
	
	\ref{enu:set_basic_cap_bound}
	\( (\Leftarrow) \) part comes from \ref{enu:set_basic_cap}.
	Argue as \ref{enu:set_basic_cup_bound} to show the converse.
	
	\ref{enu:set_basic_setminus_smaller}
	Observe
	\( x\in A \wedge \neg( x \in B) \Rightarrow x\in A\).
	
	\ref{enu:set_basic_trivial_cup}
	\( A \subset A \Rightarrow A \subset A \cup A \subset A\)
	by \ref{enu:set_basic_cup} and \ref{enu:set_basic_cup_bound}.
	
	\ref{enu:set_basic_empty_cup}
	\( A \subset A \wedge \emptyset \subset A \Rightarrow A \subset A \cup \emptyset \subset A\).
	
	\ref{enu:set_basic_trivial_cap}
	\( A \subset A \Rightarrow A \subset A \cap A \subset A\)
	by \ref{enu:set_basic_cap} and \ref{enu:set_basic_cap_bound}.
	
	\ref{enu:set_basic_empty_cap}
	\( A \cap \emptyset \subset \emptyset \).
	
	\ref{enu:set_basic_setminus_itself}
	The statement \( x\in A\setminus A \Rightarrow  x \in \emptyset \)
	is vacuously true.
	
	\ref{enu:set_basic_itself_minus_empty}
	Note \( \Forall{ x}\left[ x \notin \emptyset \right] \),
	which implies
	\( A \subset A \setminus \emptyset \).
	
	\ref{enu:set_basic_empty_minus_itself}
	\( \emptyset \setminus A \subset \emptyset \).
	
	\ref{enu:set_basic_decomposition}
	Distribution law for \( \wedge,\vee \) gives
	\begin{eqnarray*}
		x \in A
		&\equiv&
		x \in A \wedge( x \in B \vee x \notin B)\\
		&\equiv&
		( x \in A \wedge x \in B) \vee ( x \in A \wedge x \notin B)\\
		&\equiv&
		x \in (A \cap B)\cup(A \setminus B).
	\end{eqnarray*}
	
	\ref{enu:set_basic_cup_decomposition}
	Distribution law for \( \wedge,\vee \) yields
	\begin{eqnarray*}
		x \in (A \setminus B) \cup B
		&\equiv&
		(x \in A \wedge  x \notin B) \vee  x \in B\\
		&\equiv&
		x \in A \vee  x \in B\\
		&\equiv&
		x \in A\cup B.
	\end{eqnarray*}
	
	\ref{enu:set_basic_setminus_disjoint}
	The statement \(  x \in  B \cap (A \setminus B) \Rightarrow x \in \emptyset \)
	is vacuously true.
	
	\ref{enu:set_basic_bigger_setminus}
	Just check the definition.
	
	\ref{enu:set_basic_setminus_bigger}
	Note
	\( B' \subset B
	\equiv
	\Forall{ x }\left[ x \in B' \Rightarrow  x \in B\right]
	\equiv
	\Forall{ x }\left[ x \notin B \Rightarrow  x \notin B'\right]
	\),
	and argue as \ref{enu:set_basic_bigger_setminus}.
	
	\ref{enu:set_basic_equiv_disjoint}
	Use \ref{enu:set_basic_decomposition} and \ref{enu:set_basic_setminus_disjoint}.
	
	\ref{enu:set_basic_equiv_inclusion}
	It is easy to see
	\( B \setminus(B \setminus A) = A \cap B \),
	and so (ii) and (v) are equivalent.
	We show
	\( \mathrm{(iii)} \Rightarrow \mathrm{(ii)} \Rightarrow \mathrm{(iv)}
	\Rightarrow (A \subset B) \Rightarrow \mathrm{(i)} \Rightarrow \mathrm{(iii)}\).
	
	\ref{enu:set_basic_decomposition} gives \( \mathrm{(iii)} \Rightarrow \mathrm{(ii)} \Rightarrow \mathrm{(iv)} \),
	and \( \mathrm{(iv)}\Rightarrow (A \subset B) \) is trivial.
	\( (A \subset B) \Rightarrow \mathrm{(i)} \) is verified by seeing
	\( A \subset B \Rightarrow B \subset A\cup B \subset B\).
	For the last implication, observe
	\( A\setminus B = A \setminus (A\cup B) \subset A \setminus A = \emptyset\).
	
	\ref{enu:set_basic_cup_setminus},
	\ref{enu:set_basic_cap_setminus},
	\ref{enu:set_basic_times_cap}
	and \ref{enu:set_basic_setminus_times} are verified by chasing definition.
\end{prf}

\begin{exe}
	Determine which of the following statements are true for all sets \( A,B,C \) and \( D \).
	If a double implication fails, determine whether one or the other of the possible implications holds.
	If an equality fails, determine whether the statement becomes true
	if the ''equals'' symbols is replaced by one or the other of the inclusion symbols
	\( \subset \) or \( \supset \).
	\begin{enumerate}
		\item
		      \( A\subset B \) and \( A\subset C \Leftrightarrow A \subset (B \cup C) \).
		      
		\item
		      \( A\subset B \) or \( A\subset C \Leftrightarrow A \subset (B \cup C) \).
		      
		\item
		      \( A\subset B \) and \( A\subset C \Leftrightarrow A \subset (B \cap C) \).
		      
		\item
		      \( A\subset B \) or \( A\subset C \Leftrightarrow A \subset (B \cap C) \).
		      
		\item
		      \( A \setminus (A \setminus B) = B\).
		      
		\item
		      \( A \setminus (B \setminus A) = A \setminus B\).
		      
		\item
		      \( A\cap (B\setminus C) = (A \cap B) \setminus (A \cap C) \).
		      
		\item
		      \( A\cup (B\setminus C) = (A \cup B) \setminus (A \cup C) \).
		      
		\item \( (A \cap B) \cup (A \setminus B) = A \).
		      
		\item
		      \( A\subset C \) and \( B \subset D \Rightarrow (A \times B) \subset (C \times D)\).
		      
		\item
		      The converse of (j).
		      
		\item
		      The converse of (j), assuming that \( A \) and \( B \) are nonempty.
		      
		\item
		      \( (A\times B) \cup (C \times D) = (A \cup C) \times (B \cup D) \).
		      
		\item
		      \( (A\times B) \cap (C \times D) = (A \cap C) \times (B \cap D) \).
		      
		\item
		      \( A \times (B \setminus C) = (A \times B) \setminus (A \times C) \).
		      
		\item
		      \( (A \setminus B) \times  (C \setminus D) = (A \times C \setminus B \times C) \setminus A \times D \).
		      
		\item
		      \( (A \times B) \setminus  (C \times D)
		      = (A \setminus C) \times (B \setminus D)\).
	\end{enumerate}
\end{exe}\begin{sol}\leavevmode \par
	\fbox{(a)}
	(\( \Rightarrow \)) part is true by \refer{Proposition}{prop:set_operation}
	and \( B \cap C \subset B \cup C \).\leavevmode \par
	(\( \Leftarrow \)) part fails for the case, for instance,
	\( A=\{2,3\},B=\{1,2\},C=\{2,3\} \).
	
	\fbox{(b)}
	(\( \Rightarrow \)) part is obviously true while the other part fails by the same counterexample given at (a).
	
	\fbox{(c)}
	This is true by \refer{Proposition}{prop:set_operation} (d).
	
	\fbox{(d)}
	(\( \Rightarrow \)) fails. A counterexample could be
	\( A=\{1\},B=\{1,2\},C=\emptyset\).
	The converse is true by \refer{Proposition}{prop:set_operation} (d).
	
	\fbox{(e)}
	\refer{Proposition}{prop:set_operation} (e) tells us that
	only (\( \subset \)) part holds.
	
	\fbox{(f)}
	\refer{Proposition}{prop:set_operation} (o) implies
	\( A \cap (B \setminus A) = \emptyset \),
	which is equivalent to
	\( A \setminus (B \setminus A) =A \)
	by \refer{Proposition}{prop:set_operation} (r).
	Note that \( A = A \setminus B \) if and only if \( A \cap B = \emptyset \).
	Thus, only (\( \supset \)) is valid in general.
	
	\fbox{(g)}
	Equality holds as we see that
	\begin{eqnarray*}
		(A\cap B)\setminus (A\cap C)
		&=&
		\left\{ x \pipe	( x \in A \wedge  x \in B)
		\wedge
		\neg( x \in A \wedge  x \in C)
		\right\}\\
		&=&
		\left\{ x \pipe	( x \in A \wedge  x \in B)
		\wedge
		( x \notin A \vee x \notin C)
		\right\}\\
		&=&
		\left\{ x \pipe	 x \in A \wedge  (x \in B \wedge x \notin C) \right\}\\
		&=&
		A \cap (B \setminus C).
	\end{eqnarray*}
	
	\fbox{(h)}
	Only (\( \supset \))holds.
	We deduce that
	\begin{equation*}
		A \cup (B \setminus C)
		=
		(A\cup B)\setminus (C\setminus A).
	\end{equation*}
	It follows from the fact
	\( C\setminus A \subset A \cup C \)
	and
	\refer{Proposition}{prop:set_operation} (q)
	that
	\( (A\cup B)\setminus (C\setminus A)
	\supset
	(A\cup B) \setminus (A\cup C)\).
	
	A counterexample for (\( \subset \)) could be \( A=B=C=\{1\} \).
	
	\fbox{(i)}
	Equality holds by \refer{Proposition}{prop:set_operation} (m).
	
	\fbox{(j)}
	This is almost obvious as follows:
	\begin{eqnarray*}
		A\times B
		&=&
		\left\{ (a,b) \pipe a \in A \wedge b \in B \right\}\\
		&\subset&
		\left\{ (a,b) \pipe a\in C \wedge b \in D \right\}\\
		&=&
		C \times D.
	\end{eqnarray*}
	
	\fbox{(k)}
	The converse of (j) is false.
	Consider the case where \( B \) and \( C \) are empty
	while \( A \) and \( D \) are nonempty.
	
	\fbox{(\(\ell\))}
	If \( a\in A \) and \( b \in B \),
	or equivalently, if
	\( (a,b)\in A\times B \),
	then, by assumption, there holds
	\( (a,b)\in C\times D \),
	from which we conclude \( a\in C \) and \( b \in D \).
	
	\fbox{(m)}
	Only (\( \subset \)) is the case.
	If \( (x,y) \in (A\times B)\cup (C \times D) \),
	or equivalently, if
	\begin{equation*}
		( x \in A \wedge y\in B) \vee ( x \in C \wedge y\in D),
	\end{equation*}
	then, we deduce that
	\( x\in A\cup C \) and \( y \in B\cup D \),
	that is,
	\( (x,y) \in (A\cup C)\times (B\cup D) \).
	
	A counterexample for the other part could be:
	\( A=B=\{1\},C=D= \{2\}\).
	In that case,
	\( (1,2)\notin (A\times B)\cup (C \times D)\)
	but
	\( (1,2)\in (A\cup C)\times (B\cup D)\).
	
	\fbox{(n)}
	This is true by \refer{Proposition}{prop:set_operation} (v).
	
	\fbox{(o)}
	This is true by \refer{Proposition}{prop:set_operation} (w).
	
	\fbox{(p)}
	This is true by \refer{Proposition}{prop:set_operation} (w).
	
	\fbox{(q)}
	Only (\( \supset \)) is valid.
	(p) gives
	\begin{eqnarray*}
		(A \setminus C) \times (B \setminus D)
		=
		\left( (A \times B)\setminus (A\times D) \right) \setminus (C \times D),
	\end{eqnarray*}
	from which the (\( \supset \)) part follows.
	
	As a counterexample, consider the case \( A=B=\{1,2\},C=D=\{1\}\),
	where we see
	\( (1,2)\in (A \times B)\setminus (C\times D) \),
	but
	\( (1,2)\notin (A \setminus C)\times (B \setminus D) \).
	\qed\end{sol}

\begin{exe}\leavevmode \par
	\begin{enumerate}
		\item
		      Write the contrapositive and converse of the following statement:
		      ''If \( x<0 \), then \( x^2 - x >0 \),''
		      and determine which (if any) of the three statements are true.
		      
		\item
		      Do the same thing for the statement ''If \( x>0 \), then \( x^2 - x >0 \).''
	\end{enumerate}
\end{exe}\begin{sol}\leavevmode \par
	\fbox{(a)}
	The contrapositive statement is
	''If \( x^2-x \le 0 \), then \( x \ge 0 \).''
	This is true, since we have \( x^2-x \le 0 \) if and only if \( 0 \le x \le 1 \).
	
	The converse is:
	''If \( x^2-x > 0 \), then \( x < 0 \).''
	This is false, since we see \( x^2-x > 0 \)
	if and only if \( x<0 \) or \( 1<x \).
	
	\fbox{(b)}
	The contrapositive statement is
	''If \( x^2-x \le 0 \), then \( x \le 0 \),''
	and the converse is:
	''If \( x^2-x > 0 \), then \( x > 0 \).''
	They are both false for the same reason as (a).
	\qed\end{sol}

\begin{exe}
	Let \( A \) and \( B \) be sets of real numbers.
	Write the negation of each of the following statements:
	\begin{enumerate}
		\item
		      For every \( a \in A \), it is true that \( a^2 \in B \).
		      
		\item
		      For at least one \( a \in A \), it is true that \( a^2 \in B \).
		      
		\item
		      For every \( a \in A \), it is true that \( a^2 \notin B \).
		      
		\item
		      For at least one \( a \notin A \), it is true that \( a^2 \in B \).
	\end{enumerate}
\end{exe}\begin{sol}\leavevmode \par
	\fbox{(a)}
	For at least one \( a \in A \), it is true that \( a^2 \notin B \).
	
	\fbox{(b)}
	For every \( a \in A \), it is true that \( a^2 \notin B \).
	
	\fbox{(c)}
	For at least one \( a \in A \), it is true that \( a^2 \in B \).
	
	\fbox{(d)}
	For every \( a \notin A \), it is true that \( a^2 \notin B \).
	\qed\end{sol}

\begin{exe}
	Let \( \mathcal{A} \) be a nonempty collection of sets.
	Determine the truth of each of the following statements and of their converses:
	\begin{enumerate}
		\item
		      \( x \in \bigcup_{A \in \mathcal{A}}A \Rightarrow  x \in A\) for at least one
		      \( A \in \mathcal{A} \).
		      
		\item
		      \( x \in \bigcup_{A \in \mathcal{A}}A \Rightarrow  x \in A\) for every
		      \( A \in \mathcal{A} \).
		      
		\item
		      \( x \in \bigcap_{A \in \mathcal{A}}A \Rightarrow  x \in A\) for at least one
		      \( A \in \mathcal{A} \).
		      
		\item
		      \( x \in \bigcap_{A \in \mathcal{A}} \Rightarrow  x \in A\) for every
		      \( A \in \mathcal{A} \).
	\end{enumerate}
\end{exe}\begin{sol}\leavevmode \par
	\fbox{(a)}
	The statement and its converse are both true by definition.
	
	\fbox{(b)}
	The converse is true by definition while the original statement is false in the case where \( x\in A \) for only one \( A\in \mathcal{A} \) and \( x \notin A \) for others.
	
	\fbox{(c)}
	The original is trivially true by definition while the converse fails for the same reason as (b).
	
	\fbox{(d)}
	The statement and its converse are both true by definition.
	\qed\end{sol}

\begin{exe}
	Write the contrapositive of each of the statements of Exercise 5.
\end{exe}\begin{sol}\leavevmode \par
	\fbox{(a)}
	\( x \notin A \)
	for every
	\( A \in \mathcal{A} \Rightarrow x \notin \bigcup_{A \in \mathcal{A}} A\).
	
	\fbox{(b)}
	\( x \notin A \)
	for at least one
	\( A \in \mathcal{A} \Rightarrow x \notin \bigcup_{A \in \mathcal{A}} A\).
	
	\fbox{(c)}
	\( x \notin A \)
	for every
	\( A \in \mathcal{A} \Rightarrow x \notin \bigcap_{A \in \mathcal{A}} A\).
	
	\fbox{(d)}
	\( x \notin A \)
	for at least one
	\( A \in \mathcal{A} \Rightarrow x \notin \bigcap_{A \in \mathcal{A}} A\).
	\qed\end{sol}

\begin{exe}
	Given sets \( A,B \), and \( C \), express each of the following sets in terms of
	\( A, B \), and \( C \), using the symbols \( \cup \), \(\cap \), and \( \setminus \).
	\begin{eqnarray*}
		D &=&
		\left\{ x \pipe x\in A\;\text{and}\;( x \in B \;\text{or}\; x \in C) \right\},\\
		E &=&
		\left\{ x \pipe (x \in A\;\text{and}\; x \in B) \;\text{or}\; x \in C \right\},\\
		F &=&
		\left\{ x \pipe x\in A\;\text{and}\;( x \in B \Rightarrow x \in C) \right\}.
	\end{eqnarray*}
\end{exe}\begin{sol}%7
	It is obvious that we have
	\begin{equation*}
		D=A \cap (B \cup C),
	\end{equation*}
	and
	\begin{equation*}
		E=(A\cap B)\cup C.
	\end{equation*}
	For \( F \), we deduce that
	\begin{eqnarray*}
		F
		&=&
		\left\{ x \pipe x\in A \wedge(\neg(x\in B) \vee x\in C) \right\}\\
		&=&
		\left\{ x \pipe (x\in A \wedge x\notin B)
		\vee
		(x \in A \wedge x\in C)
		\right\}\\
		&=&
		(A\setminus B)\cup (A\cap C).
	\end{eqnarray*}
	\qed\end{sol}

\begin{exe}
	If a set \( A \) has two elements, show that \( \mathcal{P}(A) \) has four elements.
	How many elements does \( \mathcal{P}(A) \) have if \( A \) has one element.
	Three elements?
	No elements?
	Why is \( \mathcal{P}(A) \) called the power set of \( A \)?
\end{exe}\begin{sol}
	It is easy to check that
	\( \mathcal{P}(\{ a \}) = \{ \emptyset, \{ a \} \} \),
	\( \mathcal{P}(\{ a,b \}) = \{ \emptyset, \{ a \},\{ b \},\{ a,b \} \} \).
	
	
	In general, we claim that
	if a set \( A \) has \( n \) elements,
	then \( \mathcal{P}(A) \) has \( 2^n \) elements.
	(This is why we call \( \mathcal{P}(A) \) the ''power'' set.)
	Given a set \( A \) that has \( n \) elements,
	there are
	\begin{equation*}
		\binom{n}{k}\left( :=\frac{n!}{(n-k+1)!k!} \right)
	\end{equation*}
	subsets of \( A \) that have \( k \) elements,
	from which we deduce that
	\( \mathcal{P}(A) \) has
	\begin{equation*}
		\sum_{k=1}^{n}\binom{n}{k}
		= (1+1)^n=2^n
	\end{equation*}
	elements.
	%We prove this by mathematical induction on \( n \).
	%We have already seen that the claim holds for \( n=1 \).
	%Assume it is valid for \( n-1 \).
	%Let
	%\( A:=\left\{ a_1,a_2,\cdots,a_n \right\} \)
	%and define
	%\( B:=A \setminus \left\{ a_n \right\} \)
	%Assumption implies
	%\( \mathcal{P}(B) \)
	%has \( 2^{n-1} \)
	%elements.
	%On the other hand, it is obvious that there holds
	%\begin{equation*}
	%	\left\{ A' \in \mathcal{P}(A) \pipe a_n \in A' \right\}
	%	=
	%	\left\{ \left\{ a_n \right\}\cup B' \pipe B'\in \mathcal{P}(B) \right\}.
	%\end{equation*}
	%Letting this collection of sets denoted by \( \mathcal{A}\),
	%it is easy to check that
	%\( \mathcal{P}(A)=\mathcal{P}(B) \cup \mathcal{A} \),
	%and that
	%\( \mathcal{P}(B) \cap \mathcal{A} = \emptyset\),
	%and that
	%\( \mathcal{A} \)
	%has the same number of elements as
	%\( \mathcal{P}(B) \).
	%Thus, we conclude that
	%\( \mathcal{P}(A) \)
	%has
	%\( 2^{n-1}+2^{n-1}=2^n \)
	%elements.
	\qed\end{sol}

\begin{exe}
	Formulate and prove DeMorgan's laws for arbitrary unions and intersections.
\end{exe}\begin{sol}
	Given a set \( X \) and a collection \( \mathcal{A}\) of subsets of \( X \),
	DeMorgan's laws are formulated as:
	\begin{equation*}
		X \setminus \bigcup_{A \in \mathcal{A}} A
		=
		\bigcap_{A \in \mathcal{A}}(X\setminus A),\;\;
		X \setminus \bigcap_{A \in \mathcal{A}} A
		=
		\bigcup_{A \in \mathcal{A}}(X\setminus A).
	\end{equation*}
	The proof is essentially the same as Exercise 1.
	\begin{eqnarray*}
		x\in X \setminus \bigcup_{A \in \mathcal{A}} A
		&\equiv&
		x \in X
		\wedge
		x \notin \bigcup_{A \in \mathcal{A}} A\\
		&\equiv&
		x \in X
		\wedge
		\neg \left( \Exists{A \in \mathcal{A}}\left[  x \in A \right] \right)\\
		&\equiv&
		x \in X
		\wedge
		\Forall{A \in \mathcal{A}}\left[  x \notin A \right]\\
		&\equiv&
		\Forall{A \in \mathcal{A}}\left[  x \in X \setminus A \right]\\
		&\equiv&
		x \in \bigcap_{A \in \mathcal{A}}(X\setminus A).
	\end{eqnarray*}
	Similar argument works for the other implication.
	\qed\end{sol}

\begin{exe}
	Let \( \mathbb{R} \) denote the set of real numbers .
	For each of the following subsets of \( \mathbb{R} \times \mathbb{R} \),
	determine whether it is equal to the cartesian product of two subsets of \( \mathbb{R} \).
	\begin{enumerate}
		\item
		      \( \left\{ (x,y) \pipe x \in \mathbb{Z} \right\} \).
		      
		\item
		      \( \left\{ (x,y) \pipe 0< y \le 1\right\} \).
		      
		\item
		      \( \left\{ (x,y) \pipe y>x\right\} \).
		      
		\item
		      \( \left\{ (x,y) \pipe  x \notin \mathbb{Z} \wedge y \in \mathbb{Z} \right\} \).
		      
		\item
		      \( \left\{ (x,y) \pipe x^2 + y^2 < 1\right\} \).
	\end{enumerate}
\end{exe}\begin{sol}\leavevmode \par%10
	\fbox{(a)}
	It is equal to
	\( \mathbb{Z} \times \mathbb{R} \).
	
	\fbox{(b)}
	It is equal to
	\( \mathbb{R} \times \left\{ x\in \mathbb{R} \pipe 0\le x \le 1 \right\} \).
	
	\fbox{(c)}
	It is not equal to any cartesian product.
	Observe \( (-1,0) \) and \( (0,1) \) belong to the given set, but \( (0,0) \) does not,
	which violates an obvious necessary condition for the set to be a cartesian product.
	
	\fbox{(d)}
	\( \left( \mathbb{R} \setminus \mathbb{Z} \right) \times \mathbb{Z}\).
	
	\fbox{(e)}
	Not equal to any cartesian product.
	\( (3/4,0) \)
	and
	\( (0,3/4) \)
	belong to the given set
	whereas
	\( (3/4,3/4) \)
	does not.
	\qed\end{sol}

The necessary condition we use at Exercise 10(c) and (e) turns out to be a sufficient condition as follows:
\begin{prp}\label{prop:cartesian}
	Let
	\( X \)
	and
	\( Y \)
	be sets, and let
	\( A \)
	be a subset of
	\( X \times Y \).
	A necessary and sufficient condition for \( A \) to be the cartesian product of a subset of \( X \) and that of \( Y \) is that there holds
	\begin{equation*}
		(x_1,y_1),(x_2,y_2) \in A \Rightarrow (x_i,y_j) \in A
	\end{equation*}
	for any \( i,j=1,2 \).
\end{prp}
\begin{prf}
	Necessity is obvious.
	To prove sufficiency, suppose the given condition holds.
	Define two sets \( X_1 \) and \( Y_1 \) by setting
	\begin{eqnarray*}
		X_1&:=&\left\{  x \in X \pipe \Exists{ y \in Y }\left[ (x,y)\in A \right] \right\},\\
		Y_1&:=&\left\{  y \in Y \pipe \Exists{ x \in X }\left[ (x,y)\in A \right] \right\}.
	\end{eqnarray*}
	We show
	\( A=X_1\times Y_1 \).
	It is easy to check that
	\( A \subset X_1\times Y_1 \).
	Let
	\((x,y)\in X_1\times Y_1 \),
	or equivalently, let
	\( x\in X_1 \)
	and
	\( y \in Y_1 \).
	By definition, there exist \( y_0 \in Y \) such that \( (x,y_0) \in A \),
	and \( x_0 \in X \) such that \( (x_0,y) \in A \).
	Then, it follows from the assumption that \( (x,y) \in A \),
	which implies \( A \supset X_1\times Y_1 \).
	Thus, \( A=X_1\times Y_1 \).
\end{prf}
\end{document}
