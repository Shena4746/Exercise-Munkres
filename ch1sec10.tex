\documentclass[a4paper,12pt]{article}
\usepackage{mystyle}
\usepackage{commands}
\mathtoolsset{showonlyrefs=true}
\begin{document}
\section{Well-Ordered Sets}
\setcounter{exe}{0}

\begin{exe}
	Show that every well-ordered set has the least upper bound property.
\end{exe}
\begin{sol}
	See \S3 Exercise 13,14, where we have established that
	an ordered set has the least upper bound property if and only if
	it has the greatest lower bound property.
	
	Let \( W \) be a well-ordered set,
	and let \( W_0 \) be its nonempty subset bounded below.
	Definition of well-ordered set gives a smallest element \( s \) of \( W_0 \).
	Since \( s \) is also an infimum of \( W_0 \),
	we conclude that \( W \) has the the greatest lower bound property.
	\qed\end{sol}

\begin{exe}\leavevmode
	\begin{enumerate}
		\item Show that in a well-ordered set,
		      every element except the largest (if exists) has an immediate successor.
		\item Find a set in which every element has an immediate successor that is not well-ordered.
	\end{enumerate}
\end{exe}
\begin{sol}\leavevmode \par
	\fbox{(a)}
	Let \( W \) be a well-ordered set,
	and let \( a \) be a non-largest element of \( W \).
	Set
	\( W_0:=\left\{ b\in W \pipe a<b \right\} \),
	which is nonempty by the choice of \( a \).
	Then \( W_0 \) admits a smallest element \( s \),
	for which we have
	\( (a,s)= \emptyset \).
	Thus, \( s \) is the immediate successor of \( a \).
	
	\fbox{(b)}
	\( \mathbb{Z} \).
	\qed\end{sol}

\begin{exe}
	Both \( \left\{ 1,2 \right\} \times \mathbb{Z}_{+} \)
	and
	\( \mathbb{Z}_{+}\times \left\{ 1,2 \right\} \)
	are well-ordered in the dictionary order.
	Do they have the same order type?
\end{exe}
\begin{sol}
	No, they have different order types.
	Note that every element of
	\( \left\{ 1,2 \right\} \times \mathbb{Z}_{+} \)
	admits the immediate successor
	while that of 
	\( \mathbb{Z}_{+} \times \left\{ 1,2 \right\} \)
	does not.
	(consider \( (a,1)\in \mathbb{Z}_{+} \times \left\{ 1,2 \right\} \),
	for instance.)
	Our claim follows from this fact and \refer{Proposition}{prop:order_type}.
	
	See also \S3 Exercise 12, where we discussed a similar question. 
	\qed\end{sol}

\begin{exe}\leavevmode \par
	\begin{enumerate}
		\item
		      Let \( \mathbb{Z}_{-} \) denote the set of negative integers in the usual order.
		      Show that a simply ordered set \( A \) fails to be well-ordered
		      if and only if it contains a subset having the same order type as \( \mathbb{Z}_{-} \).
		\item
		      Show that \( A \) is simply ordered and every countable subset of \( A \)
		      is well-ordered, then \( A \) is well-ordered.
	\end{enumerate}
\end{exe}
\begin{sol}\leavevmode \par
	\fbox{(a)}
	''if'' part is obvious since in that case that subset does not admits a smallest element,
	and hence \( A \) is not well-ordered.
	
	Conversely, suppose that \( A \) is not well-ordered.
	Let \( A_0 \) be a subset of \( A \) that does not admit a smallest element.
	Also, let \( a\in A_0 \) and let \( c \) be a choice function for \( A_0 \).
	Consider the following formula for \( f:\mathbb{Z}_{+} \to A_0 \):
	\begin{eqnarray*}
		f(-1)&:=&a,\\
		f(-i)&:=&c \left( S_{f(-i+1)} \right)
	\end{eqnarray*}
	for all \( i \in \mathbb{Z}_{+} \),
	where \( S_{\alpha} \) denotes the section of \( A_0 \) by \( \alpha \in A_0 \)
	given by
	\( S_{\alpha}:=\left\{ \alpha' \in A_0 \pipe \alpha' < \alpha \right\} \).
	It is easy to check that principle of recursive definition works for this formula,
	and that so derived \( f \) is injective and order-preserving.
	Thus, \( f(\mathbb{Z}_{-}) \) is a subset of \( A_0 \) which has the same order type as \( \mathbb{Z}_{-} \).
	
	\fbox{(b)}
	This is a direct consequence of contrapositive statement of (a).
	\qed\end{sol}

\begin{exe}
	Show the well-ordering theorem implies the choice axiom.
\end{exe}
\begin{sol}
	Let \( \mathcal{A} \) be a collection of nonempty sets.
	It suffices to show that there exists a choice function for \( \mathcal{A} \).
	Let \( A^{\ast}:=\bigcup_{A \in \mathcal{A}} \).
	Apply to \( \mathcal{A} \) the well-ordering theorem so that
	\( \mathcal{A} \) is well-ordered.
	Then, a function
	\( c:\mathcal{A} \to A^{\ast} \)
	that assigns, to each \( A \in \mathcal{A} \),
	a smallest element \( c(A) \) of \( A \) is a choice function for \( \mathcal{A} \).
	\qed\end{sol}

\begin{exe}
	Let \( S_{\Omega} \) be the minimal uncountable well-ordered set.
	\begin{enumerate}
		\item
		      Show that \( S_{\Omega} \) has no largest element.
		      
		\item
		      Show that for every \( \alpha \in S_{\Omega} \),
		      the set \( \left\{ x \pipe \alpha < x \right\} \)
		      is uncountable.
		      
		\item
		      Let \( X_0 \) be the subset of  \( S_{\Omega} \) consisting of all elements
		      \( x \) such that \( x \) has no immediate predecessor.
		      Show that \( X_0 \) is uncountable.
	\end{enumerate}
\end{exe}
\begin{sol}\leavevmode \par
	\fbox{(a)}
	It suffices to show that
	every elements of \( S_{\Omega} \) has the immediate successor in \( S_{\Omega} \).
	It is obvious that \( S_{\Omega}\cup \Omega \) has a unique largest element \( \Omega \).
	Let \( a\in S_{\Omega} \).
	Exercise 2(a) implies \( a \) has its immediate successor \( b \) in \( S_{\Omega}\cup \Omega \).
	Observe that 
	\( S_{b} = S_{a}\cup \left\{ a \right\} \) is countable
	while
	\( S_{\Omega} \)
	is uncountable,
	and hence 
	\( S_{b}\neq S_{\Omega} \).
	Thus, \( b \neq \Omega \).
	(Remember, in general, for a simply ordered set,
	\( S_{\alpha} = S_{\beta}\) if and only if \( \alpha = \beta \).)
	This means \( b \in S_{\Omega} \).
	
	We might argue as follows:
	suppose, to the contrary, that \( m \) is a largest element of \( S_{\Omega} \).
	Then we are led to \( S_{\Omega} = S_{m} \cup \left\{ m \right\}\).
	But this is impossible since \( S_{\Omega} \) is uncountable while 
	\( S_{m} \cup \left\{ m \right\}\) is countable.
	Thus, \( S_{\Omega} \) has no largest element.
	
	\fbox{(b)}
	Given \( \alpha \in S_{\Omega}\),
	it follows from (a) that
	the set \( \left\{ x \pipe \alpha < x \right\} \)
	is unbounded above in \( S_{\Omega} \).
	Thus, we deduce from Theorem 10.3 that
	\( \left\{ x \pipe \alpha < x \right\} \)
	is uncountable.
	
	\fbox{(c)}
	By Theorem 10.3,
	it suffices to show that \( X_0 \) is unbounded above.
	Let \( s: S_{\Omega} \to S_{\Omega} \) be a function that
	assigns, to each \( a\in S_{\Omega} \), the immediate successor of \( a \).
	It is easy to verify that \( s \) preserves order.
	Given \( \alpha \in S_{\Omega} \),
	let
	\( A:=\left\{ \alpha, s(\alpha), s^2(\alpha),\cdots \right\} \).
	Since \( A \) is countable,
	\( A \) admits a supremum \( \sup{A} \) in \( S_{\Omega} \)
	by Theorem 10.3 and Exercise 1.
	We claim \( \sup{A} \in X_0 \).
	In fact, for any \(  x \in S_{\Omega} \) with \( x< \sup{A} \),
	there exists \( y \in A \)
	such that \( x<y<S_{\Omega} \), that is,
	\( \left( x, \sup{A}\right) \neq \emptyset \).
	Hence, \( \sup{A}\in X_0 \).
	It follows from \( \alpha < \sup{A} \) that \( X_0 \) is unbounded above.
	\qed\end{sol}

\begin{exe}[The principle of transfinite induction]
	Let \( J \) be a well-ordered set.
	A subset \( J_0 \) of \( J \) is said to be \textbf{\textit{inductive}}
	if for every \( \alpha \in J \),
	\begin{equation*}
		(S_{\alpha} \subset J_0) \Rightarrow \alpha \in J_0.
	\end{equation*}
	Theorem.\;
	If \( J \) is a well-ordered set and \( J_0 \) is an inductive subset of \( J \),
	then \( J_0=J \).
\end{exe}
\begin{sol}
	Let \( J \) be a well-ordered set,
	and let \( J_0 \) be a subset of \( J \).
	Suppose \( J \setminus J_0  \neq \emptyset \).
	Then, there exists a smallest element \( x \) of \( J \setminus J_0 \).
	We see that
	\( S_x \subset J_0 \)
	and
	\( x \notin J_0 \).
	Thus, \( J_0 \) is not inductive.
	\qed\end{sol}

\begin{exe}\leavevmode
	\begin{enumerate}
		\item
		      Let \( (A_1,<_{1}) \) and \( (A_2,<_2) \) be disjoint well-ordered sets.
		      Define an order relation on \( A_1 \cup A_2 \) by letting
		      \( a<b \)
		      either if \( a,b \in A_1 \)
		      and \( a<_1 b \),
		      or if \( a,b \in A_2 \)
		      and \( a<_2 b \),
		      or if \( a \in A_1 \) and \( b \in A_2 \).
		      Show that this is a well-ordering.
		      
		\item
		      Generalize (a) to an arbitrary family of disjoint well-ordered sets, indexed by a well-ordered set.
	\end{enumerate}
	
	
\end{exe}
\begin{sol}\leavevmode \par
	\fbox{(a)}
	We first verify that the given relation is actually an order relation.
	Noting that \( A_1 \) and \( A_2 \) are disjoint,
	comparability and nonreflexivity are obvious.
	To show transitivity,
	suppose \( a<b \) and \( b<c \).
	If \( a,b,c \in A_i \) for some \( i \),
	then we have \( a<_{A_i} c \), and so \( a<c \).
	If \( a,b \in A_1 \) and \( c \in A_2 \),
	then definition of \( < \) gives \( a<c \).
	The same argument works for the remaining cases.
	Hence, the given relation is an order relation.
	
	We establish \( (A_1\cup A_2, <) \) is a well-ordered set.
	Let \( B \) is a nonempty subset of \( A_1\cup A_2 \).
	If \( B \) is included in \( A_i \) for some \( i \),
	then there exists a smallest element \( s \) of \( B \) in \( (A_i, <_i) \).
	It is obvious that \( s \) is also a smallest element of \( B \)
	in \( (A_1\cup A_2, <) \).
	If, on the other hand, \( B \cap A_1 \neq \emptyset \) and \( B \cap A_2 \neq \emptyset \),
	then a smallest element of \( B \cap A_1 \) in \( (A_i, <_i) \)
	turns out to be that of \( B \cap A_1 \) in \( (A_1\cup A_2, <) \)
	because every element of \( B \cap A_1 \) is smaller than that of \( B \cap A_2 \).
	
	\fbox{(b)}
	Let \( (W_i, <_i) \) be a family of disjoint well-ordered sets, indexed by a well-ordered set \( I \).
	Define \( W:=\bigcup_{i \in I}W_i \)
	and an order relation on \( W \)
	by setting 
	\( a<b \)
	either if
	\( a,b \in W_i \)
	and
	\( a<_i b \)
	for some \( i \),
	or if
	\( a \in W_i \)
	and
	\( b \in W_j \)
	for \( i<j \).
	
	The same argument as (a) shows that \( < \) is an order relation on \( W \).
	Let \( B \) be a subset of \( W \) and let \( i^{\ast} \) be a smallest element of the set \( \left\{ i \in I \pipe B \cap W_i \neq \emptyset \right\} \).
	It is then easy to show that a smallest element of \( B \cap W_{ i^{\ast}} \)
	is also that of \( (W,<) \) by the way \( < \) is introduced.
	\qed\end{sol}

\begin{rem}
	In Exercise 8,
	we require \( (A_1, <_1) \) and \( (A_2, <_2) \) to be disjoint.
	But this is not an essential assumption:
	
	Given two (not necessarily disjoint) well-ordered sets
	\( (B,<_B) \) and \( (C,<_C) \),
	define new disjoint well-ordered sets
	\( (\left\{ 1 \right\}\times B, <_{B'}) \)
	and
	\( (\left\{ 2 \right\}\times C, <_{C'}) \),
	where \( <_{B'} \) is defined to be \( (1,b_1) <_{B'} (1,b_2) \)
	if \( b_1 <_B b_2 \),
	and similarly for \( <_{C'} \).
	It is obvious that \( (B,<_B) \) and \( (C,<_C) \)
	have the same order types as
	\( (\left\{ 1 \right\}\times B, <_{B'}) \)
	and
	\( (\left\{ 2 \right\}\times C, <_{C'}) \),
	respectively.
	Thus, we may assume two well-ordered sets are disjoint if necessary.
\end{rem}

\begin{exe}
	Consider the subset \( A \) of \( (\mathbb{Z}_{+})^{\omega} \)
	consisting of all infinite sequence of positive integers
	\( \bm{x}=(x_1,x_2,\cdots) \)
	that end in an infinite string of 1's.
	Give \( A \) the following order:
	\( \bm{x} < \bm{y} \) if \( x_n < y_n \) and \( x_i=y_i \) for \( i>n \).
	\begin{enumerate}
		\item
		      Show that for every \( n \),
		      there is a section of \( A \) that has the same order type as
		      \( (\mathbb{Z}_{+})^{n} \) in the dictionary order.
		      
		\item
		      Show that \( A \) is well-ordered.
	\end{enumerate}
\end{exe}
\begin{sol}
	We first note that \( A \) is an ordered set with the given order.
	Given \( n\in \mathbb{Z}_{+} \),
	let
	\( I^{(n)}:=\{ \underbrace{1,1,\cdots,1}_{n},2,1,1,\cdots \} \).
	Let \( S(\bm{x}) \) denote the section of \( A \) by \( \bm{x} \).
	
	\fbox{(a)}
	We claim that for every \( n \in \mathbb{Z}_{+} \)
	there exists an order isomorphism from \( S \left( I^{(n)} \right) \)
	to \( \mathbb{Z}_{+}^{n} \).
	First observe
	\begin{equation*}
		S \left( I^{(n)} \right)
		=
		\left\{  \left\{ x_1,x_2,\cdots,x_n,1,1,\cdots \right\} \pipe x_i \in \mathbb{Z}_{+} \right\}
	\end{equation*}
	We prove our claim by induction on \( n \).
	It is easy to check that a function given by
	\begin{equation*}
		S \left( I^{(1)} \right) \ni \left\{ x,1,1,\cdots \right\}
		\mapsto x \in \mathbb{Z}_{+}
	\end{equation*}
	is an order isomorphism.
	Assume the claim holds for \( n-1 \),
	and let \( f \) be an order isomorphism from
	\( S \left( I^{(n-1)} \right) \) to \( \mathbb{Z}_{+}^{n-1} \).
	Define a function
	\( h: S \left( I^{(n)} \right) \to \mathbb{Z}_{+}\times \mathbb{Z}_{+}^{n-1}\)
	via
	\begin{equation*}
		h \left( \left\{ x_1,x_2,\cdots,x_{n-1},x_n,1,1,\cdots \right\} \right)
		\to
		\left( x_n, f(x_1,\cdots,x_{n-1},1,1,\cdots) \right)
	\end{equation*}
	It is clear that \( h \) is injective.
	Also, It is straightforward to verify that \( h \) preserves order.
	Since there exists an order isomorphism from
	\( \mathbb{Z}_{+}\times \mathbb{Z}_{+}^{n-1} \)
	to
	\( \mathbb{Z}_{+}^{n} \),
	there also exists an order isomorphism from
	\( S \left( I^{(n)} \right) \)
	to \( \mathbb{Z}_{+}^{n} \).
	
	\fbox{(b)}
	Let \( A_0 \) be a nonempty subset of \( A \).
	Let \(  \bm{x} \in A_0 \).
	We can write \( \bm{x}=(x_1,x_2,\cdots,x_n,1,1,\cdots) \)
	for some \( n \in \mathbb{Z}_{+} \).
	Observe \( \bm{x} \in S \left( I^{(n)} \right) \)
	and so \( A_0 \cap  S \left( I^{(n)} \right) \neq \emptyset\).
	Then, \( f \left( A_0 \cap S \left( I^{(n)} \right) \right)\)
	is a nonempty subset of \( \mathbb{Z}_{+}^{n} \),
	where \( f \) is an order isomorphism from
	\( S\left( I^{(n)} \right) \) to \( \mathbb{Z}_{+}^{n} \).
	Since \( \mathbb{Z}_{+}^{n} \) is well-ordered,
	\( f \left( A_0 \cap S \left( I^{(n)} \right) \right)\)
	admits a smallest element \( m \).
	So, \( f ^{-1}(m) \) is a smallest element of
	\( A_0 \cap  S \left( I^{(n)} \right) \).
	It is obvious that \( f ^{-1}(m) \) is also a smallest element of \( A_0 \).
	Thus, \( A \) is a well-ordered set. 
	\qed\end{sol}

\begin{exe}[pre-general principle of recursive definition]\leavevmode \par
	\noindent \text{Theorem.}\;\;
	Let \( J \) and \( C \) be well-ordered sets;
	assume that there is no surjective function from a section of \( J \) onto \( C \).
	Then there exists a unique function
	\( h: J \to C \)
	satisfying the equation
	\begin{equation}\label{eq:pre_recursive}
		h(x)=\min{\left[ C \setminus h(S_x) \right]}
	\end{equation}
	for each \( x \in J \),
	where \( S_x \) is the section of \( J \) by \( x \).
	
	\textit{Proof.}
	\begin{enumerate}
		\item
		      If \( h \) and \( k \) map sections of \( J \), or all of \( J \), into \( C \)
		      and satisfy (\refeq{eq:pre_recursive}) for all \( x \) in their respective domains,
		      show that \( h(x)=k(x) \) for all \( x \) in both domains.
		      
		\item
		      If there exists a function \( h:S_{\alpha}\to C\) satisfying (\ref{eq:pre_recursive}),
		      show that there exists a function
		      \( k :S_{\alpha}\cup \left\{ \alpha \right\}\to C \)
		      satisfying (\refeq{eq:pre_recursive}).
		      
		\item
		      If \( K \subset J \)
		      and for all \( \alpha \in K \) there exists a function
		      \( h_{\alpha}: S_{\alpha}\to C \)
		      satisfying (\refeq{eq:pre_recursive}),
		      show that there exists a function
		      \begin{equation*}
			      k: \bigcup_{\alpha \in K} S_{\alpha}\to C
		      \end{equation*}
		      satisfying (\refeq{eq:pre_recursive}).
		      
		\item
		      Show by transfinite induction that every \( \beta \in J \),
		      there exists a function
		      \( h_{\beta}: S_{\beta}\to C\)
		      satisfying (\refeq{eq:pre_recursive}).
		      
		\item
		      Prove the theorem.
	\end{enumerate}
\end{exe}
\begin{sol}\leavevmode \par
	\fbox{(a)}
	This is a simple consequence of the principle of transfinite induction.
	
	Let
	\( D_{h} \)
	and
	\( D_k \)
	be the domain of \( h \) and \( k \),
	respectively.
	Suppose \( h \) and \( k \) satisfy (\refeq{eq:pre_recursive}) for all \( x \)
	in their respective domains.
	Let \( D \) be a subset of \( D_h \cap D_k \) 
	consisting of all \(  x \) for which we have
	\begin{equation*}
		h(x) = k(x).
	\end{equation*}
	Note that
	\( h(S_{\min{J}}) = h(\emptyset) = \emptyset \)
	and so
	\( h(\min{J}) = \min{C} = k(\min{J}) \)
	is well-defined.
	We deduce from the fact
	\( D \)
	is inductive that
	\( D= D_h \cap D_k\).
	
	\fbox{(b)}
	Consider a function
	\( k :S_{\alpha}\cup \left\{ \alpha \right\}\to C \)
	given by
	
	\begin{equation*}
		k(x):=	 \begin{cases}
			h(x) & \;\mathrm{:}\; x \in S_{\alpha} \\
			\min{\left[ C \setminus h(S_{\alpha}) \right] }
			     & \;\mathrm{:}\;x=\alpha.
		\end{cases}
	\end{equation*}
	
	\fbox{(c)}
	Let \(  x \in \bigcup_{\alpha \in K} S_{\alpha}\).
	There exists \( \alpha \in K \) such that \(  x \in S_{\alpha} \).
	(a) implies that the value \( h_{\alpha}(x) \) is uniquely determined independent of \( \alpha \).
	Thus, a function	
	\( k: \bigcup_{\alpha \in K} S_{\alpha}\to C \)
	given by
	\begin{equation*}
		k(x):=h_{\alpha}(x)
	\end{equation*}
	is well-defined.
	
	Note that \( \bigcup_{\alpha \in K} S_{\alpha} \) is a subset of \( J \),
	a well-ordered set, and hence it is also a well-ordered set.
	Let \( X \) be a subset of \( \bigcup_{\alpha \in K} S_{\alpha} \)
	consisting of all \( x \) for which 
	\begin{equation*}
		k(x) = \min{\left[ C \setminus k(S_{\alpha}) \right]}
	\end{equation*}
	holds.
	Let \(  x \in \bigcup_{\alpha \in K} S_{\alpha} \)
	and suppose \( S_x \subset X \).
	Then, (b) gives \(  x \in X \),
	which means \( X \) is inductive.
	Thus, \( X = \bigcup_{\alpha \in K} S_{\alpha} \)
	by the principle of transfinite induction.
	
	\fbox{(d)}
	Let \( Y \) be a subset of \( J \)
	consisting of all \( \beta \) for which there exists a function
	\( h_{\beta}: S_{\beta}\to C\)
	satisfying (\ref{eq:pre_recursive}).
	We only have to show that \( Y \) is inductive so that
	the principle of transfinite induction yields the result.
	Let \( \beta \in J \) and suppose \( S_{\beta} \subset Y \).
	If \( \beta \) has an immediate successor \( \alpha \),
	then we have \( S_{\beta}=S_{\alpha} \cup \left\{ \alpha \right\} \)
	and so (b) implies \( \beta \in Y \).
	If not, 
	\( S_{\beta}=\bigcup_{\alpha < \beta} S_{\alpha} \),
	and hence (c) gives \( \beta \in Y \).
	Thus, \( Y \) is inductive.
	
	\fbox{(e)}
	Uniqueness of \( h \) (if exists) follows from (a).
	
	(d) and (c) imply that there exists a function
	\begin{equation*}
		k: \bigcup_{\alpha \in J} S_{\alpha}\to C
	\end{equation*}
	satisfying (\ref{eq:pre_recursive}).
	If \( J \) has a largest element \( m \),
	then we have
	\begin{equation*}
		\bigcup_{\alpha \in J} S_{\alpha}
		=S_m
		=J \setminus \left\{ m \right\}
	\end{equation*}
	and so, (b) yields a function \( h:J\to C \)
	satisfying (\ref{eq:pre_recursive}).
	If not, then \( J=\bigcup_{\alpha \in J} S_{\alpha} \)
	and we are done.
	\qed\end{sol}

\begin{exe}
	Let \( A \) and \( B \) be two sets.
	Using the well-ordering theorem, prove that either they have the same cardinality,
	or one has the greater cardinality greater than the other.
\end{exe}
\begin{sol}
	Apply, to \( A \) and \( B \) if necessary, the well-ordering theorem
	so that both \( A \) and \( B \) are well-ordered.
	Remember that, under the choice axiom, a consequence of the well-ordering theorem,
	\( \surj{A}{B} \neq \emptyset \) if and only if
	\( \inj{B}{A} \neq \emptyset \)
	by \S9 Exercise 5 and \S2 Exercise 5.
	
	So, if \( \inj{A}{B} = \emptyset \),
	or equivalently if \( \surj{B}{A} = \emptyset \),
	then Exercise 10 gives \( \inj{A}{B} \neq \emptyset \).
	This means that there cannot hold
	\( \inj{A}{B} = \emptyset \)
	and
	\( \inj{B}{A} = \emptyset \)
	at the same time,
	from which we conclude the result.
	\qed\end{sol}

\end{document}