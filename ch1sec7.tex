\documentclass[a4paper,12pt]{article}
\usepackage{mystyle}
\usepackage{commands}
\mathtoolsset{showonlyrefs=true}

\begin{document}
\section{Countable and Uncountable Sets}
\setcounter{exe}{0}
Let us introduce new notations "\( \sim \)" and "\( \hookrightarrow \)".

Because we study cardinality through bijection,
we are repeatedly in need to write "there exists a bijection from...\! to...".
Let us introduce a notation to omit this and thus simplify our description:
let \( X \) be a set; define a relation on \( \mathcal{P}(X) \) by setting
\( A \sim B \) if \( A \) and \( B \) are in bijective correspondence.
It is easy to check that this is well-defined and is an equivalence relation.
This observation invites us to the idea of classifying sets based on their cardinality.
This idea is developed as we proceed;
see \S7 Exercise 6, \S9 Exercise 7, \S10 Exercise 11.
Note that \( A \sim B \) if and only if \( \bij{A}{B} \neq \emptyset \).

Although the concept of cardinality is defined in terms of of bijection,
it is quite often hard to verify the existence of a desired bijective function between sets of concern.
That is why we have investigated some alternative conditions that allow us to say something about cardinality without explicit mention of bijection.
One of them is the one given in terms of the existence of injection, such as \S6 Corollary 6.7, \S Theorem 7.1.
However, it turns out that, as Schroeder-Bernstein theorem (Exercise 6) implies,
injection is indeed a very good substituting tool to explore with.
(Moreover, several concepts related to cardinality are formulated in terms of injection rather than bijection; see \S9 Exercise 7, \S10 Exercise).
So, it is natural to have a notation that expresses the existence of injection;
we write \( A \hookrightarrow B \) if there exists an injection from \( A \) to \( B \),
that is, if \( \inj{A}{B}\neq \emptyset \).

\begin{prp}[Cardinality of the set of functions]\label{prop:bij-card}
	\leavevmode \par \noindent
	Let \( A,B,C, \) etc.\! be sets. We claim that we have
	\begin{enumerate}
		\item \label{enu:power_card_func}
		      \( \mathcal{P}(A) \sim \func{A}{\left\{ 0,1 \right\}}\).
		      
		\item \label{enu:card_func_augment}
		      \( \func{A\times B}{C} \sim  \func{A}{\func{B}{C}}\).
		      
		\item \label{enu:func_card_func}
		      \( A \sim A' \wedge B \sim B' \Rightarrow \func{A}{B}\sim \func{A'}{B'} \).
		      
		\item \label{enu:power_card_power}
		      \( A \sim A' \Rightarrow \mathcal{P}(A) \sim \mathcal{P}(A') \).
	\end{enumerate}
\end{prp}
\begin{prf}
	Consider \ref{enu:power_card_func}.
	Let \( \tau \) be a function given by
	\begin{equation*}
		\tau: \mathcal{P}(A) \ni B \mapsto \in 1_{B} \in \func{A}{\left\{ 0,1 \right\}},
	\end{equation*}
	where \( 1_B \) is the characteristic function of \( B \) given by
	\begin{equation*}
		1_B(x):=\begin{cases}
			1 & \;\mathrm{:}\; x \in B              \\
			0 & \;\mathrm{:}\; x \in A \setminus B.
		\end{cases}
	\end{equation*}
	\( \tau \) is bijective since it admits the inverse
	\( \tau' \) defined as follows:
	\begin{equation*}
		\tau':\func{A}{\left\{ 0,1 \right\}} \ni \chi
		\mapsto
		\left\{  x \in A \pipe \chi(x)=1\right\} \in \mathcal{P}(A).
	\end{equation*}
	
	We show \ref{enu:card_func_augment}.
	Consider a function \( \phi \) given by
	\begin{equation*}
		\phi : \func{A\times B}{C}\times A \ni (f,a)
		\mapsto
		f(a,\cdot) \in \func{B}{C},
	\end{equation*}
	and define a function \( \Phi \) by setting
	\begin{equation*}
		\Phi : \func{A\times B}{C} \ni f
		\mapsto \phi(f,\cdot) \in \func{A}{\func{B}{C}}.
	\end{equation*}
	It is straightforward to check that \( \Phi \) is bijective.
	
	%Note that
	%\( \func{A}{\func{B}{C}} \subset A \times (B\times C)\),
	%\( \func{A\times B}{C} \subset (A\times B)\times C \),
	%and that \( A \times (B\times C) \sim (A\times B)\times C\)
	%under some bijective function \( \iota \).
	%We show
	%\( \func{A}{\func{B}{C}} \hookrightarrow \func{A\times B}{C}\)
	%and
	%\( \func{A\times B}{C} \hookrightarrow \func{A}{\func{B}{C}}\)
	%under \( \iota \) and \( \iota ^{-1}\) respectively.
	%Then, Schroeder-Bernstein theorem proves the result.
	%Observe the following equivalence:
	%\begin{eqnarray*}
	%&&
	%f\in \func{A}{\func{B}{C}}\\
	%&\Leftrightarrow&
	%\Forall{ a\in A }
	%\unique{ f(a) \subset B \times C }
	%\left[
	%\Forall{ b\in B }
	%\unique{ c\in C }
	%\left[ (b,c) \in f(a) \right]
	%\right]\\
	%&\Leftrightarrow&
	%\Forall{ a\in A }
	%\unique{ F(a) \subset A \times (B \times C) }
	%\left[
	%\Forall{ b\in B }
	%\unique{ c\in C }
	%\left[ (a,(b,c)) \in F(a) \right]
	%\right]\\
	%&\Leftrightarrow&
	%\left[
	%\Forall{ a\in A }
	%\unique{ F(a) \subset A \times (B \times C)}
	%\right]
	%\wedge
	%\Forall{ a\in A }
	%\Forall{ b\in B }
	%\unique{ c\in C }
	%\left[ (a,(b,c)) \in F(a) \right]\\
	%&\Leftrightarrow&
	%\bigcup_{a\in A}F(a) \subset A \times (B \times C)
	%\wedge
	%\Forall{ a\in A }
	%\Forall{ b\in B }
	%\unique{ c\in C }
	%\left[ (a,(b,c)) \in \bigcup_{a\in A}F(a) \right]\\
	%&\Leftrightarrow&
	%\iota \left( \bigcup_{a\in A}F(a) \right)
	%\subset (A \times B) \times C
	%\wedge
	%\Forall{ (a,b)\in A\times B }
	%\unique{ c\in C }
	%\left[ ((a,b),c) \in \iota \left( \bigcup_{a\in A}F(a) \right) \right]\\
	%&\Leftrightarrow&
	%\iota \left( \bigcup_{a\in A}F(a) \right) \in \func{A\times B}{C},
	%\end{eqnarray*}
	%where \( F(a):=\left\{ (a,(b,c)) \pipe (b,c)\in f(a)\wedge a\in A \right\} \).
	%Since \( \iota \) and \( \iota ^{-1}\) are both bijective, this implies the result.
	
	\ref{enu:func_card_func} is easy to prove;
	let \( \varphi \in \bij{A}{A'} \) and \( \psi \in \bij{B}{B'} \).
	A function given by
	\begin{equation*}
		\func{A}{B} \ni g \mapsto \psi \circ g \circ \varphi ^{-1} \in \func{A'}{B'}
	\end{equation*}
	is bijective since it admits the inverse
	\begin{equation*}
		\func{A'}{B'} \ni h \mapsto \psi^{-1} \circ h \circ \varphi \in \func{A}{B}.
	\end{equation*}
	
	\ref{enu:power_card_power} follows as 
	\( \mathcal{P}(A)
	\sim \func{A}{\left\{ 0,1 \right\}}
	\sim \func{A'}{\left\{ 0,1 \right\}}
	\sim \mathcal{P}(A')
	\).
\end{prf}

\begin{exe}
	Show that \( \mathbb{Q} \) is countably infinite.
\end{exe}\begin{sol}
	Note
	\( \mathbb{Q}=\mathbb{Q}_{-} \cup \left\{ 0 \right\} \cup \mathbb{Q}_{+} \)
	by definition,
	where
	\( \mathbb{Q}_{-}:=\left\{ q \pipe -q \in \mathbb{Q}_{+} \right\} \)
	which is countable.
	Apply Theorem 7.5.
	\qed\end{sol}

\begin{exe}
	Show that the maps \( f \) and \( g \) of Example 1 and 2 are bijections.
\end{exe}\begin{sol}
	We list these functions here for convenience as follows:
	The function \( f:\mathbb{Z} \to \mathbb{Z}_{+} \) in Example 1 is defined by
	\begin{equation*}
		f(n) = \begin{cases}
			2n    & \;\mathrm{:}\;n>0    \\
			-2n+1 & \;\mathrm{:}\;n\le0.
		\end{cases}
	\end{equation*}
	Let
	\( A:=\left\{ (x,y)\in \mathbb{Z}_{+} \times \mathbb{Z}_{+}
	\pipe y \le x\right\} \).
	The functions
	\( f:\mathbb{Z}_{+} \times \mathbb{Z}_{+} \to A\)
	and
	\( g:A \to \mathbb{Z}_{+} \)
	in Example 2 are defined by
	\begin{equation*}
		f(x,y) = (x+y-1,y),
	\end{equation*}
	and
	\begin{equation*}
		g(x,y)=\frac{1}{2}(x-1)x+y.
	\end{equation*}
	
	We first consider \( f \) in Example 1.
	Define a function \( h:\mathbb{Z}_{+} \to \mathbb{Z} \) by setting
	\begin{equation*}
		h(n):=\begin{cases}
			m  & \;\mathrm{:}\;
			\Exists{ m \in \mathbb{Z}_{+}}\left[ n=2m \right] \\
			-m & \;\mathrm{:}\;
			\Exists{ m \in \mathbb{Z}_{+}} \cup \left\{ 0 \right\}
			\left[ n=2m + 1 \right].
		\end{cases}
	\end{equation*}
	It is straightforward to check that \( h \) is well-defined and is the inverse of \( f \).
	
	We proceed to Example 2.
	The function
	\begin{equation*}
		\varphi:A \ni (x,y) \mapsto (x-y+1,y) \in \mathbb{Z}_{+} \times \mathbb{Z}_{+}
	\end{equation*}
	is the inverse of \( f \).
	So, both \( f \) and \( \varphi \) are bijective.
	
	Next, we show how \( g \) is constructed.
	Let
	\( s:\mathbb{Z}_{+}\cup\{0\} \to \mathbb{Z}_{+}\cup\{0\} \)
	be a function such that
	\( s(0):=0 \)
	and
	\( s(x):=\sum_{k=0}^{x}k \)
	for
	\( x \ge 1 \).
	Observe that there holds, for \( x \ge 1 \),
	\begin{equation*}
		s(x-1) = \frac{1}{2}x(x-1),
	\end{equation*}
	and that, given \( n\in \mathbb{Z}_{+} \), the set
	\begin{equation*}
		X:=\left\{  x \in \mathbb{Z}_{+} \pipe n \le s(x) \right\}
	\end{equation*}
	is a nonempty subset of \( \mathbb{Z}_{+} \).
	So, we can take the smallest element \( x^{\ast} \) of \( X \)
	with the following property:
	\begin{equation*}
		s(x^{\ast} - 1) < n \le  s(x^{\ast}).
	\end{equation*}
	Since we have \( s(x)-s(x-1)=x \) in general,
	there exists \( y \in \mathbb{Z}_{+} \) with \( 1 \le y \le x \)
	such that
	\begin{equation*}
		n=s(x^{\ast} - 1)+y.
	\end{equation*}
	Uniqueness of \( x^{\ast} \) and \( y \) is obvious by construction.
	Hence, we have shown that
	\begin{equation*}
		\Forall{ n \in \mathbb{Z}_{+} }
		\unique{ (x,y)\in A }
		\left[ n=\frac{1}{2}x(x-1)+y \right],
	\end{equation*}
	which implies
	\( g \)
	is bijective.
	\qed\end{sol}

\begin{exe}
	Let \( X \) be the two-element set \( \left\{ 0,1 \right\} \).
	Show there is a bijective correspondence between the set
	\( \mathcal{P}(\mathbb{Z}_{+}) \)
	and the cartesian product \( X^{\omega} \).
\end{exe}\begin{sol}
	Noting the fact \( X^{\omega} = \func{\mathbb{Z}_{+}}{X} \),
	it suffices to show that, for a set \( A \), we have
	\( \mathcal{P}(A) \sim \func{A}{\left\{ 0,1 \right\}}\).
	But, we have already proved this.
	See \refer{Proposition}{prop:bij-card}.
	\qed\end{sol}

\begin{exe}\leavevmode \par
	\begin{enumerate}
		\item
		      A real number \( x \) is said to be \textbf{\textit{algebraic}}(over the rationals)
		      if it satisfies some polynomial equation of positive degree
		      \begin{equation*}
			      x^n + a_{n-1}x^{n-1}+\cdots+a_1x + a_0 = 0
		      \end{equation*}
		      with rational coefficients \( a_i \).
		      Assuming that each polynomial equation has only finitely many roots,
		      show that the set of algebraic numbers is countable.
		      
		\item
		      A real number is said to be \textbf{\textit{transcendental}}
		      if it is not algebraic.
		      Assuming the reals are uncountable,
		      show that the transcendental numbers are uncountable.
	\end{enumerate}
\end{exe}\begin{sol}\leavevmode \par
	\fbox{(a)}
	Without loss of generality, we may assume each polynomial equation has integer coefficients.
	Let \( E \) represent the set of polynomial equations over integers,
	and, given \( f \in E \), let \( \mathrm{deg}(f) \) be the degree of \( f \).
	Let \( E_n \) be the set consisting of elements of \( E \) whose degree is \( n \).
	We define a function \( H : E \to \mathbb{Z}_{+}\) by setting,
	for
	\( E \ni f(x)=a_0 + a_1 x +\cdots +a_n x^n \),
	\begin{equation*}
		H(f):=n+\sum_{k=0}^{\mathrm{deg}(f)}|a_k|.
	\end{equation*}
	Observe that
	\( H(f) \ge 2\) for all \( f \in E \), 
	and that the set defined as
	\( E_n(h):=\left\{ f\in E_n \pipe H(f)=h \right\} \)
	is finite for every
	\( n, h \in \mathbb{Z}_{+} \),
	and that \( E_n=\bigcup_{h\in \mathbb{Z}_{+}}E_n(h) \)
	for every
	\( n \in \mathbb{Z}_{+} \),
	and that
	\( E=\bigcup_{n \in \mathbb{Z}_{+}}E_n \).
	
	Let \( g:E \to \mathcal{P}(\mathbb{R}) \) be a function that maps every \( f \in E \) to the set consisting of its roots.
	Each \( g(E_n(h)) \) is finite
	since each \( E_n(h) \) is finite and each of \( f \in E_n(h) \) has at most finitely many roots.
	Thus,
	the set of algebraic numbers
	\begin{equation*}
		g(E)
		=
		g \left( \bigcup_{n,h \in \mathbb{Z}_{+}}E_n(h) \right)
		=
		\bigcup_{n,h \in \mathbb{Z}_{+}}g(E_n(h))
	\end{equation*}
	is countable.
	
	\fbox{(b)}
	We prove that the set of transcendental number has the same cardinality as that of real numbers.
	Let \( A \) be the set of algebraic numbers.
	In order to establish the existence of transcendental numbers,
	we claim \( \mathbb{R}\setminus A \neq \emptyset \).
	In fact, we have in general
	\( \mathbb{R}\setminus A = \emptyset  \Leftrightarrow \mathbb{R} \subset A\),
	and RHS turns out to be false since a countable set cannot include an uncountable set.
	
	Let \( \theta \) be a transcendental number.
	It is obvious that \( n \theta \) is also transcendental number for all \( n \in \mathbb{Z}_{+} \).
	Consider a partition of \( \mathbb{R} \) given by
	\begin{equation*}
		\mathbb{R} = A \cup T_1 \cup T_2,
	\end{equation*}
	where \( T_1:=\left\{ n \theta \pipe n \in \mathbb{Z}_{+} \right\} \)
	and
	\( T_2:=\mathbb{R} \setminus (A \cup T_1) \),
	both combined constitutes the set of transcendental numbers.
	It is easy to see that
	\( A \) and \( T_1 \) are countable, and hence
	\( A \cup T_1 \sim T_1 \),
	which implies
	\( \mathbb{R} = (A \cup T_1) \cup T_2 \sim T_1 \cup T_2\).
	Thus, the proof is completed.
	\qed\end{sol}

\begin{exe}
	Determine, for each of the following sets,
	whether or not it is countable.
	Justify your answers.
	\begin{enumerate}
		\item
		      The set \( A \) of all functions
		      \( f:\left\{ 0,1 \right\} \to \mathbb{Z}_{+} \).
		      
		\item
		      The set \( B_n \) of all functions
		      \( f:\left\{ 1,\cdots,n \right\} \to \mathbb{Z}_{+} \).
		      
		\item
		      The set \( C=\bigcup_{n \in \mathbb{Z}_{+}}B_n \).
		      
		\item
		      The set \( D \) of all functions
		      \( f:\mathbb{Z}_{+} \to \mathbb{Z}_{+} \).
		      
		\item
		      The set \( E \) of all functions
		      \( f:\mathbb{Z}_{+} \to \left\{ 0,1 \right\} \).
		      
		\item
		      The set \( F \) of all functions
		      \( f:\mathbb{Z}_{+} \to \left\{ 0,1 \right\} \).
		      that are "eventually zero."
		      [We say that \( f \) is \textbf{\textit{eventually zero}} if there is a positive integer \( N \) such that \( f(n)=0 \) for all \( n \ge N \).]
		      
		\item
		      The set \( G \) of all functions
		      \( f:\mathbb{Z}_{+} \to \mathbb{Z}_{+} \).
		      that are eventually \( 1 \).
		      
		\item
		      The set \( H \) of all functions
		      \( f:\mathbb{Z}_{+} \to \mathbb{Z}_{+} \).
		      that are eventually constant.
		      
		\item
		      The set \( I \) of all two-element subsets of \( \mathbb{Z}_{+} \).
		      
		\item
		      The set \( J \) of all finite subsets of \( \mathbb{Z}_{+} \).
	\end{enumerate}
	
\end{exe}\begin{sol}
	Remember the concept of cartesian product is defined in terms of functions.
	
	\fbox{(a)}
	\( A=\func{\left\{ 0,1 \right\}}{\mathbb{Z}_{+}}
	\sim \mathbb{Z}_{+}\times \mathbb{Z}_{+} \)
	is countable by Theorem 7.6.
	
	\fbox{(b)}
	\( B_n = \func{\left\{ 1,2,\cdots,n \right\}}{\mathbb{Z}_{+}}
	\sim \mathbb{Z}_{+}^{n} \)
	is countable by Theorem 7.6.
	
	\fbox{(c)}
	Theorem 7.5 yields that \( C \) is countable.
	
	\fbox{(d)}
	Observe that we have
	\( D=\func{\mathbb{Z}_{+}}{\mathbb{Z}_{+}}\sim \mathbb{Z}_{+}^{\omega} \),
	and
	\( \left\{ 0,1 \right\}^{\omega} \sim \left\{ 1,2 \right\}^{\omega} \subset \mathbb{Z}_{+}^{\omega}\).
	Then, Theorem 7.7 and contrapositive of Theorem 7.3 implies
	\( \mathbb{Z}_{+}^{\omega} \) is uncountable.
	
	\fbox{(e)}
	\( E = \func{\mathbb{Z}_{+}}{\left\{ 0,1 \right\}}
	\sim \mathcal{P}(\mathbb{Z}_{+}) \)
	is uncountable by Theorem 7.8.
	
	\fbox{(f)}
	We provide a bit generalized proof for (f) here in order to illustrate the idea that works for (g) and (h).
	
	Consider a function
	\( N:F\to \mathbb{Z}_{+} \)
	that assigns, to every \( f\in F \), the smallest element \( N(f) \) of the set
	\begin{equation*}
		\left\{
		n \in \mathbb{Z}_{+} \pipe
		\Forall{ m \in \mathbb{Z}_{+} }
		\Forall{ k \in \mathbb{Z}_{+} }
		\left[
			m\ge n \wedge k \ge n \Rightarrow f(m)=f(k) \right]
		\right\},
	\end{equation*}
	which the definition of \( F \) guarantees is nonempty.
	In other words, \( N(f)\) designates exactly when \( f \) starts to be a constant.
	Setting \( F_n:=\left\{ f\in F \pipe N(f)=n \right\} \),
	we see that \( F= \bigcup_{n\in \mathbb{Z}_{+}}F_n \),
	and that each \( F_n \) is countable since a function \( \varphi \) defined by
	\begin{equation*}
		\varphi : F_n \ni f \mapsto f|_{S_{n+1}}
		\in \func{S_{n+1}}{\mathbb{Z}_{+}}
	\end{equation*}
	is injective, where
	\( f|_{S_{n+1}} \)
	represents the restriction of \( f \) onto \( S_{n+1} \).
	Thus, \( F \) is countable. 
	
	\fbox{(g)}\fbox{(h)}
	\( G \) and \( H \) can be proved to be countable by trivial modifications of (f).
	
	\fbox{(i)}
	Let \( I^{\ast}:=\left\{ (x_i)_{i=0,1}\in \mathbb{Z}_{+} \times \mathbb{Z}_{+} \pipe x_0 < x_1 \right\} \).
	It is obvious that \( I \sim I^{\ast} \).
	We then deduce from the fact 
	\( I^{\ast} \hookrightarrow \mathbb{Z}_{+} \times \mathbb{Z}_{+} \)
	that \( I \) is countable.
	
	\fbox{(j)}
	Let \( J_n:=\left\{ N \in J \pipe \card{N} = n \right\} \),
	and let
	\begin{equation*}
		J_n^{\ast}
		:=
		\left\{
		(x_i)_{i \in S_{n+1}}\in \mathbb{Z}_{+}^{n} \pipe
		\Forall{ i \in S_{n} }\left[ x_i < x_{i+1} \right]
		\right\}.
	\end{equation*}
	As before, we deduce from the fact
	\( J_n^{\ast} \hookrightarrow
	\func{S_{n+1}}{\mathbb{Z}_{+}}\sim \mathbb{Z}_{+}^n \)
	that \( J_n^{\ast} \) is countable,
	and so is \( J_n \) since \( J_n^{\ast} \sim J_n \).
	Thus, \( J=\bigcup_{n\in \mathbb{Z}_{+}}J_n \)
	is countable.
	\qed\end{sol}

\begin{exe}
	We say that two sets \( A \) and \( B \)
	\textbf{\textit{have the same cardinality}}
	if there is a bijection of \( A \) with \( B \).
	\begin{enumerate}
		\item
		      Show that if \( B \subset A \) and if there is an injection
		      \begin{equation*}
			      f:A \to B,
		      \end{equation*}
		      then \( A \) and \( B \) have the same cardinality.
		      
		\item
		      \text{Theorem\;(Schroeder-Bernstein theorem).}\;\;
		      If there are injections
		      \( f:A\to C \)
		      and
		      \( g:C \to A \),
		      then \( A \) and \( C \) have the same cardinality.
	\end{enumerate}
\end{exe}\begin{sol}\leavevmode \par
	\fbox{(a)}
	Let \( B \subset A \) and \( f\in \inj{A}{B} \).
	Define the sets \( A_n \)  and \( B_n \) recursively by the formula
	\begin{eqnarray*}
		A_1&:=&A,\\
		B_1&:=&B,\\
		A_n&:=&f(A_{n-1}),\\
		B_n&:=&f(B_{n-1}),
	\end{eqnarray*}
	for \( n>1 \).
	We have
	\( A_1 \supset B_1 \supset A_2 \supset B_2 \supset A_3 \supset \cdots \)
	by definition.
	Consider a function \( h:A\to B \) such that
	\( h(x):=f(x) \)
	if
	\( x\in A_n\setminus B_n \)
	for some \( n \), and
	\( h(x):=x \) if otherwise.
	
	Then, we claim \( h \) is bijective.
	We first show the injectivity.
	Let \( x,y \in A \)
	with
	\( x \neq y \).
	We may assume
	\( x\in A_n\setminus B_n \)
	for some \( n \) and
	\( y \notin A_n\setminus B_n \)
	for all
	\( n \in \mathbb{Z}_{+} \)
	since otherwise injectivity trivially follows.
	But, in this case,
	\( f \in \inj{A}{B} \) implies
	\( h(x)=f(x) \in f(A_n)\setminus f(B_n)=A_{n+1}\setminus B_{n+1} \),
	and hence,
	\begin{equation*}
		h(y)=y \neq h(x),
	\end{equation*}
	establishing \( h\in \inj{A}{B} \).
	
	We prove next that \( h \) is surjective.
	Let \( b \in B(=B_1) \).
	Seeing the definition of \( h \),
	we may assume
	\( b \in A_n \setminus B_n \)
	for some
	\( n \ge 2 \).
	Then,
	\( f \in \inj{A}{B} \) implies
	\begin{equation*}
		b \in f(A_{n-1})\setminus f(B_{n-1})=f(A_{n-1} \setminus B_{n-1}),
	\end{equation*}
	which completes the proof.
	
	\fbox{(b)}
	Let
	\( C_0:=f(A) \)
	and
	\( C_1:=f \circ g(C) \subset C \).
	It is obvious that
	\( A \sim C_0 \)
	and
	\( C \hookrightarrow C_1\).
	Then (a) yields \( c \sim C_1 \).
	On the other hand, it follows from
	\( g(C) \subset A \) and \( f(A) \subset C \)
	that
	\( f \circ g(C) \subset f(A) \subset C \),
	that is,
	\( C_1 \subset C_0 \subset C \),
	which gives \( C \hookrightarrow C_0 \hookrightarrow \).
	Hence, (a) establishes \( C_0 \sim C \sim C_1 \).
	Thus, \( A \sim C \).
	\qed\end{sol}

\begin{exe}
	Show that the sets \( D \) and \( E \) of Exercise 5
	have the same cardinality.
\end{exe}\begin{sol}
	Note that
	\( D=\func{\mathbb{Z}_{+}}{\mathbb{Z}_{+}}
	\sim \mathbb{Z}_{+}^{\omega} \),
	\( E=\func{\mathbb{Z}_{+}}{\left\{ 0,1 \right\}}
	\sim \left\{ 0,1 \right\}^{\omega} \),
	and that 
	\( \left\{ 0,1 \right\}^{\omega} \sim \mathcal{P}(\mathbb{Z}_{+})\).
	
	It is clear that there holds \( E \hookrightarrow D \).
	Conversely, we claim \( D \hookrightarrow E \).
	Indeed, the fact
	\( \mathbb{Z}_{+} \subset \mathcal{P}(\mathbb{Z}_{+}) \)
	and \refer{Proposotion}{prop:bij-card}
	allow us to have
	\begin{eqnarray*}
		\func{\mathbb{Z}_{+}}{\mathbb{Z}_{+}}
		&\hookrightarrow &
		\func{\mathbb{Z}_{+}}{\mathcal{P}(\mathbb{Z}_{+})}\\
		&\sim&
		\func{\mathbb{Z}_{+}}{\func{\mathbb{Z}_{+}}{\left\{ 0,1 \right\}}}\\
		&\sim&
		\func{\mathbb{Z}_{+}\times \mathbb{Z}_{+}}{\left\{ 0,1 \right\}}\\
		&\sim&
		\func{\mathbb{Z}_{+}}{\left\{ 0,1 \right\}}.
	\end{eqnarray*}
	Schroeder-Bernstein theorem then establishes \( D \sim E \).
	\qed\end{sol}

\begin{exe}
	Let \( X \) denote the two-element set \( \left\{ 0,1 \right\} \);
	let \( \mathcal{B} \) be the set of \textit{countable} subsets of \( X^{\omega} \).
	Show that \( X^{\omega} \) and \( \mathcal{B} \) have the same cardinality.
\end{exe}\begin{sol}
	A function
	\( \left\{ 0,1 \right\}^{\omega} \mapsto \left\{ x \right\} \in \mathcal{B}\)
	is obviously injective.
	In order to claim by Schroeder-Bernstein theorem that
	\( \mathcal{B} \sim \left\{ 0,1 \right\}^{\omega} \),
	we show \( \mathcal{B} \hookrightarrow \left\{ 0,1 \right\}^{\omega}\). 
	
	Observe that countability allows us to write down each 
	\( B \in \mathcal{B} \) as
	\( B=\left\{ x^{(1)}, x^{(2)},\cdots \right\} \)
	and that constructing a function that assigns, to 
	\( \left\{ x^{(1)}, x^{(2)},\cdots \right\} \),
	\begin{equation*}
		\left(
		x^{(1)}_1,
		x^{(1)}_2,x^{(2)}_1,
		x^{(1)}_3,x^{(2)}_2,x^{(3)}_1,
		x^{(1)}_4,x^{(2)}_3,x^{(3)}_2,x^{(4)}_1,
		x^{(1)}_5,x^{(2)}_4,x^{(3)}_3,x^{(4)}_2,x^{(5)}_1,\cdots
		\right)
	\end{equation*}
	finishes the proof.
	We do that by considering a function
	\( \varphi : \mathcal{B} \to \left\{ 0,1 \right\}^{\omega} \)
	given by
	\begin{equation*}
		\varphi
		\left( \left\{ \left( x^{(1)}_i \right)_{i \in \mathbb{Z}_{+}},
		\left( x^{(2)}_i \right)_{i \in \mathbb{Z}_{+}},\cdots \right\} \right)
		=
		\left( x^{(i - s(n-1))}_{s(n-1)+n-i+1} \right)_{i \in \mathbb{Z}_{+}}
	\end{equation*}
	where \( n \in \mathbb{Z}_{+} \) is uniquely determined by
	given \( i \) and the inequality
	\begin{equation*}
		s(n-1) < i \le s(n),
	\end{equation*}
	and \( s(\cdot) \) is defined as \( s(0):=0 \), and
	\( s(n):=\sum_{j=1}^n j \)
	for \( n \ge 1 \).
	It is easy to verify that \( \varphi \) is injective.
	
	See Exercise 2, which shares the same underlying idea as this one.
	Here we have considered the following way of lining up the elements of \( B \):
	\begin{eqnarray*}
		&&x^{(1)}_1,\\
		&&x^{(1)}_2,x^{(2)}_1,\\
		&&x^{(1)}_3,x^{(2)}_2,x^{(3)}_1,\\
		&&x^{(1)}_4,x^{(2)}_3,x^{(3)}_2,x^{(4)}_1,\\
		&&x^{(1)}_5,x^{(2)}_4,x^{(3)}_3,x^{(4)}_2,x^{(5)}_1,\cdots.
	\end{eqnarray*}
	Compare this with \( g \) there.
	\qed\end{sol}

\begin{exe}\leavevmode \par
	\begin{enumerate}
		\item
		      The formula
		      \begin{eqnarray*}
			      h(1)&=&1,\\
			      h(2)&=&2,\\
			      h(n)&=&\left[ h(n+1) \right]^2 - \left[ h(n-1) \right]^2
		      \end{eqnarray*}
		      for \( n\ge 2 \)
		      is not one to which the principle of recursive definition applies.
		      Show that nevertheless there does exist a function
		      \( h:\mathbb{Z}_{+}\to \mathbb{R}\)
		      satisfying this formula.
		      
		\item
		      Show that the formula of part (a) does not determine \( h \)
		      uniquely.
		      
		\item
		      Show that there is no function \( h:\mathbb{Z}_{+} \to \mathbb{R} \)
		      satisfying the formula
		      \begin{eqnarray*}
			      h(1)&=&1,\\
			      h(2)&=&2,\\
			      h(n)&=&\left[ h(n+1) \right]^2 + \left[ h(n-1) \right]^2.
		      \end{eqnarray*}
		      for \( n\ge 2 \).
	\end{enumerate}
\end{exe}\begin{sol}\leavevmode \par
	\fbox{(a)}
	Let \( h \) satisfy the following formula:
	\begin{eqnarray*}
		h(1)&=&1,\\
		h(2)&=&2,\\
		h(n)&=&\sqrt{h(n-1)+\left[ h(n-2) \right]^2)},
	\end{eqnarray*}
	for \( n \ge 3 \).
	Then, principle of recursive definition yields unique
	\( h: \mathbb{Z}_{+} \to \mathbb{R}_{+}\)
	that satisfies the formula above.
	It is clear that \( h \) also satisfies the given formula.
	
	\fbox{(b)}
	Consider the formula
	\begin{eqnarray*}
		f(1)&=&1,\\
		f(2)&=&2,\\
		f(3)&=&-\sqrt{3},\\
		f(n)&=&\sqrt{f(n-1)+\left[ f(n-2) \right]^2)},\\
		f(n)&>&0,
	\end{eqnarray*}
	for all \( n \ge 4 \).
	This defines unique \( f:\mathbb{Z}_{+} \to \mathbb{R} \)
	also satisfying the formula at (a),
	but \( f \neq h \),
	where \( h \) is the function we have construct in (a).
	
	\fbox{(c)}
	The given formula requires
	\( h(3) \pm 1\),
	and
	\( h(4)^2=h(3)-4 \);
	the latter is impossible.
	\qed\end{sol}
\end{document}