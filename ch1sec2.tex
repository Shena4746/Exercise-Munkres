\documentclass[a4paper,12pt]{article}
\usepackage{mystyle}
\usepackage{commands}
\mathtoolsset{showonlyrefs=true}

\begin{document}
\section{Functions}
% initialize exercise numbering 
\setcounter{exe}{0}

\begin{rem}[Empty Function]
	You can skip this note.
	Nothing proved in this note is used when working on exercises in \S2.
	(We need the concept of empty function to let several exercise make sense; principle of recursive definition is one such example; see \S8 Exercise 8, \S10 Exercise 10.)
	We consider here whether we are able to define a function from or to empty set \( \emptyset \).
	Remembering a function is defined to be a certain subset of the catersian product of two sets,
	the question is reduced to the one that "could empty set qualify as a function, and if so, in what cases?"
	The answer is partially "yes";
	there exist a unique function from \( \emptyset \) to \( \emptyset \),
	and a unique function from \( \emptyset \) to nonempty set,
	each called an \textit{empty function},
	but no way could there be a function from nonempty set to \( \emptyset \).
	
	First of all, note that the definition of function is expressed as follows:
	\begin{equation*}
		f\in \func{X}{Y}
		\equiv
		\left[ f \subset X \times Y \right]
		\wedge
		\Forall{ x\in X }
		\unique{ y\in Y }
		\left[ (x,y)\in f \right],
	\end{equation*}
	where \( X \)
	and
	\( Y \)
	are sets, of course.
	Suppose \( X=\emptyset \).
	We claim \( \func{\emptyset}{Y}=\{\emptyset\} \).
	In fact, we see that
	\begin{equation*}
		f \subset \emptyset \times Y
		\equiv
		f = \emptyset,
	\end{equation*}
	and that the statements
	\begin{equation*}
		\Forall{ x\in X }
		\unique{ y\in Y }
		\left[ (x,y)\in f \right]
		\equiv
		\Forall{ x }
		\left[ x \in \emptyset
			\Rightarrow
			\unique{ y\in Y } \left[ (x,y)\in \emptyset \right]
			\right]
	\end{equation*}
	are true since the latter is vacuously true.
	Note that this argument is valid if \( Y=\emptyset \).
	Thus, the claim follows.
	
	On the other hand, suppose \( X \neq \emptyset \) and \( Y = \emptyset \).
	We now insist \( \func{X}{\emptyset}= \emptyset \).
	It suffices to show that \( f = \emptyset \) does not qualify as a function.
	Indeed, the statements
	\begin{equation*}
		\Forall{ x\in X }
		\unique{ y\in Y }
		\left[ (x,y)\in f \right]
		\equiv
		\Forall{ x\in X }
		\unique{ y }
		\left[ y\in \emptyset \wedge (x,y)\in \emptyset \right]
	\end{equation*}
	are both false since, in general, there holds
	\begin{equation*}
		\Forall{ z }\left[ z \notin \emptyset \right]
	\end{equation*}
	and so, \( y\in \emptyset \) is false.
	\qed\end{rem}


\begin{rem}[Equivalent conditions for injectivity and surjectivity]
	We summarize here some equivalent expressions for injectivity and surjectivity,
	which we frequently exploit for establishing results that are related to those property.
	Let
	\( f \in \func{A}{B} \).
	
	For injectivity, we claim that
	\begin{eqnarray}
		f \in \inj{A}{B}
		&\equiv& 
		\Forall{ a_1 \in A}
		\Forall{ a_2 \in A}
		\left[ a_1 \neq a_2 \Rightarrow f(a_1) \neq  f(a_2) \right]\notag\\
		&\equiv&
		\Forall{ a_1 \in A}
		\Forall{ a_2 \in A}
		\left[ f(a_1) = f(a_2) \Rightarrow a_1 =a_2 \right]\notag\\
		&\equiv&
		\Forall{ a \in A }
		\Forall{ A_0 \subset A }
		\left[ f(a)\in f(A_0) \Leftrightarrow a \in A_0 \right]\label{note:equiv:inj1}\\
		&\equiv&
		\Forall{ a \in A }
		\Forall{ A_0 \subset A }
		\left[ f(a)\notin f(A_0) \Leftrightarrow a \notin A_0 \right].
	\end{eqnarray}
	First two equivalence are obvious by definition.
	We prove the third.
	Suppose 
	\( f \in \inj{A}{B} \).
	Let
	\( a \in A \)
	and let
	\( A_0 \)
	be a subset of
	\( A \).
	It is obvious that  we have
	\begin{equation*}
		a \in A_0 \Rightarrow f(a)\in f(A_0).
	\end{equation*}
	So, suppose 
	\( f(a)\in f(A_0) \).
	Then, there exists
	\( x \in A_0 \)
	such that
	\( f(a) = f(x) \).
	But injectivity gives
	\( x=a \)
	and so
	\( a \in A_0\).
	Conversely, suppose (\ref{note:equiv:inj1}) holds.
	Let
	\( a,a' \in A \)
	with
	\( f(a) = f(a') \).
	Setting
	\( A_0 := \{a'\} \)
	in (\ref{note:equiv:inj1})
	yields
	\( a\in \{a'\} \),
	which means
	\( a=a' \)
	establishing the injectivity.
	
	For surjectivity, we insist that
	\begin{eqnarray*}
		f \in \surj{A}{B}
		&\equiv&
		f(A)=B\\
		&\equiv&
		\Forall{b \in B}
		\Exists{ a \in A}
		\left[ f(a)=b \right]\\
		&\equiv&
		\Forall{b \in B}
		\left[f ^{-1}(b) \neq \emptyset \right].
	\end{eqnarray*}
	First equivalence is due to the definition.
	Second owes to the definition of image,
	third to that of preimage.
	
	We can see more equivalent conditions as we proceed in this section.
	\qed\end{rem}

\begin{rem}[Injectivity of Empty function]
	Let \( f_{\emptyset}:\emptyset \to Y \)
	be an empty function.
	\( f_{\emptyset} \) is injective, but not surjective.
	In Fact, we see that
	injectivity is vacuously satisfied, and that there holds
	\begin{equation*}
		\Forall{y \in Y}
		\left[f_{\emptyset}^{-1}(y) = \emptyset \right]
	\end{equation*}
	since, in general, preimage is a subset of the domain of a function.
	Thus, \( f_{\emptyset} \) is not surjective.
	\qed\end{rem}

\begin{exe}
	Let \( f:A\to B \).
	Let \( A_0\subset A \) and \( B_0 \subset B \).
	\begin{enumerate}
		\item
		      Show that \( A_0 \subset f ^{-1}(f(A_0)) \)
		      and that equality holds if \( f \) is injective.
		      
		\item
		      Show that \( f(f^{-1}(B_0)) \subset B_0 \)
		      and that equality holds if \( f \) is surjective.
	\end{enumerate}
\end{exe}\begin{sol}\leavevmode \par
	\fbox{(a)}
	Let
	\(a \in A_0\).
	It follows that 
	\(f(a) \in f(A_0)\),
	and that
	\(f ^{-1} \left( f(a) \right) \subset f^{-1} \left( f(A_0)\right)\).
	It is clear that
	\(a \in f  ^{-1}\left( f(a) \right)\),
	and so
	\(a \in f  ^{-1}\left( f(A_0) \right)\).
	Thus,
	\( A_0 \subset f ^{-1}(f(A_0)) \).
	
	We prove the second implication.
	In general, we trivially have
	\begin{equation*}
		\Forall{a \in f ^{-1}(f(A_0))}
		\Exists{a' \in A_0}
		\left[ f(a) = f(a')\right].
	\end{equation*}
	So, if we assume 
	\(f \in \inj{A}{B}\),
	then
	\(a = a' \in A_0\).
	Thus, 
	\( f ^{-1}(f(A_0)) \subset A_0 \).
	
	\fbox{(b)}
	In general, we have,
	\begin{equation*}
		\Forall{b \in f \left( f ^{-1}(B_0) \right)}
		\Exists{a \in f ^{-1}(B_0)}
		\left[ f(a) = b\right],
	\end{equation*}
	which yields, by definition of preimage,
	\(b = f(a) \in B_0\).
	Thus,
	\(f \left( f ^{-1}(B_0) \right) \subset B_0\).
	
	Suppose
	\( f \in \surj{A}{B} \).
	Then we have
	\begin{equation*}
		\Forall{b \in B_0 }
		\Exists{a \in f ^{-1}(B_0)}
		\left[ f(a) = b \right].
	\end{equation*}
	Hence,
	\(b  = f(a) \in f \left( f ^{-1}(B_0) \right)\).
	\qed\end{sol}

\begin{rem}[Another equivalent condition for injectivity and surjectivity ]
	We prove the converse of each of second implication of Exercise 1(a) and (b).
	Let
	\( f\in \func{A}{B}\),
	and let
	\( A_0 \)
	be a subset of \( A \),
	and
	\( B_0\)
	be a subset of \( B \).
	
	Suppose
	\( A_0 = f ^{-1}(f(A_0)) \)
	holds.
	For
	\(a_1, a_2 \in A_0\)
	with
	\(f(a_1) = f(a_2)\),
	we deduce
	\begin{eqnarray*}
		\left\{ a_1 \right\}
		&=& f ^{-1}\left( f(a_1) \right)\\
		&=& f ^{-1}\left( f(a_2) \right)\\
		&=& \left\{ a_2 \right\}.
	\end{eqnarray*}
	Thus,
	\( f \)
	is injective.
	
	To see the converse of Exercise 2(b),
	Set \( B_0 := B \)
	and obtain
	\(f \left( f ^{-1}(B) \right) = B\).
	Thus,
	\(f\)
	is surjective.
	\qed\end{rem}

\begin{exe}
	Let \( f:A\to B \) and 
	let \( A_i\subset A \) and \( B_i \subset B \)
	for \( i=0 \) and \( i=1 \).
	Show that \( f ^{-1} \) preserves inclusions, unions , intersections,
	and differences of sets;
	\begin{enumerate}
		\item
		      \( B_0 \subset B_1 \Rightarrow f ^{-1}(B_0) \subset f ^{-1}(B_1) \).
		      
		\item
		      \( f ^{-1}(B_0 \cup B_1) = f ^{-1}(B_0)\cup f ^{-1}(B_1) \).
		      
		\item
		      \( f ^{-1}(B_0 \cap B_1) = f ^{-1}(B_0)\cap f ^{-1}(B_1) \).
		      
		\item
		      \( f ^{-1}(B_0 \setminus B_1) = f ^{-1}(B_0)\setminus f ^{-1}(B_1) \).\leavevmode \par
		      \noindent\text{Show that \( f \) preserves inclusions and unions only;}
		      
		\item
		      \( A_0 \subset A_1 \Rightarrow f(A_0) \subset f(A_1) \).
		      
		\item
		      \( f(A_0 \cup A_1) = f(A_0)\cup f(A_1) \).
		      
		\item
		      \( f(A_0 \cap A_1) \subset f(A_0)\cap f(A_1) \);
		      show that equality holds if \( f \) is injective.
		      
		\item
		      \( f(A_0 \setminus A_1) \supset f(A_0)\setminus f(A_1) \);
		      show that equality holds if \( f \) is injective.
	\end{enumerate}
\end{exe}\begin{sol}
	We can safely omit and actually do Exercise 2(b), (c) and (f)
	since we are anyway required to show the generalized versions of them in Exercise 3.
	The proof there of course proves the (b), (c), (f) of Exercise 2.
	
	Let
	\( f \in \func{A}{B} \),
	and let
	\( A_i \)
	and
	\( B_i\)
	be subsets of \( A \)
	and
	\( B\),
	respectively,
	for
	\( i=0,1 \).
	
	\fbox{(a)}
	Let
	\( a\in f ^{-1}(B_0) \).
	By definition and hypothesis, we have
	\( f(a) \in B_0 \subset B_1\),
	which means
	\( a \in f ^{-1}(B_1) \).
	Thus,
	\( f ^{-1}(B_0) \subset  f ^{-1}(B_1)\).
	
	\fbox{(d)}
	Observe
	\begin{eqnarray*}
		a \in f ^{-1}(B_0 \setminus B_1)
		&\equiv&
		f(a) \in B_0 \setminus B_1 \\
		&\equiv&
		f(a) \in B_0 \wedge \neg \left( f(a) \in B_1 \right)\\
		&\equiv&
		a \in  f ^{-1}(B_0) \wedge \neg \left( a \in  f ^{-1}(B_1) \right)\\
		&\equiv&
		a \in  f ^{-1}(B_0) \setminus f ^{-1}(B_1),
	\end{eqnarray*}
	which completes the proof.
	
	\fbox{(e)}
	Let
	\( b \in f(A_0) \).
	Ther exists
	\( a \in A_0 \)
	such that
	\( b=f(a) \).
	Since hypothesis gives
	\( a \in A_1 \),
	we conclude
	\( b = f(a) \in f(A_1) \).
	Thus,
	\( f(A_0) \in f(A_1) \).
	
	\fbox{(g)}
	Since
	\( A_0 \cap A_1 \subset A_0 \)
	and
	\( A_0 \cap A_1 \subset A_1 \),
	(e) yields
	\( f(A_0 \cap A_1) \subset f(A_0) \)
	and
	\( f(A_0 \cap A_1) \subset f(A_1) \),
	and so
	\( f(A_0 \cap A_1) \subset f(A_0) \cap f(A_1)\).
	
	We proceed to proving the second claim.
	In general, we have
	\begin{equation*}
		\Forall{ b\in f(A_0)\cap f(A_1) }
		\Exists{ a_i \in A_i}
		\left[ b = f(a_i) \right]
	\end{equation*}
	for \( i=0,1 \).
	So, if we assume
	\( f \in \inj{A}{B} \),
	then
	\( b = f(a_0) = f(a_1) \)
	implies
	\( a_0 = a_1 \).
	Setting
	\( a:=a_0(=a_1) \),
	we see that
	\( a \in A_0 \cap A_1 \)
	and that
	\( b = f(a) \in f(A_0 \cap A_1) \).
	This establishes the claim.
	
	\fbox{(h)}
	We generally see that
	\begin{eqnarray}
		b \in f(A_0)\setminus f(A_1)
		&\equiv&
		\Exists{a}
		\left[
			a\in A_0
			\wedge
			f(a)=b
			\wedge
			f(a)\in f(A_0)
			\wedge
			f(a) \notin f(A_1)
			\right]\notag\\
		&\equiv&
		\Exists{a}
		\left[
			a\in A_0
			\wedge
			f(a)=b
			\wedge
			f(a) \notin f(A_1)
			\right]\notag\\
		&\Rightarrow&
		\Exists{a}
		\left[
			a\in A_0
			\wedge
			f(a)=b
			\wedge
			a \notin A_1
			\right]\label{equiv:sec2Ex2h1}\\
		&\equiv&
		b \in f(A_0 \setminus A_1)\notag,
	\end{eqnarray}
	where we note that, in general, there holds
	\begin{equation}
		\left[ f(x)\notin f(P) \Rightarrow x \notin P  \right]
		\equiv
		\left[ x \in P \Rightarrow f(x)\in f(P) \right]\label{equiv:sectionEx2h2},
	\end{equation}
	and that RHS is obviously true.
	
	Next we show the second implication.
	If we assume that 
	\( f \)
	is injective,
	or equivalently, if we assume 
	\begin{equation*}
		f(x)\notin f(P) \Leftrightarrow x \notin P,
	\end{equation*}
	holds for any subset \( P \) of \( B \), then we can replace "\( \Rightarrow \)" with "\( \equiv \)"
	at (\ref{equiv:sec2Ex2h1}),
	which proves the result.
	\qed\end{sol}

\begin{rem}[Yet another equivalent condition for injectivity]
	We prove here the converse of second implications of Exercise 2(g) and (h).
	In other words, we claim that 
	\begin{eqnarray}
		f \in \inj{A}{B}
		&\equiv&
		\Forall{A_0 \subset A }
		\Forall{A_1 \subset A }
		\left[ f(A_0 \cap A_1) = f(A_0) \cap f(A_1) \right]\label{note:equiv:2}\\
		&\equiv&
		\Forall{A_0 \subset A }
		\Forall{A_1 \subset A }
		\left[ f(A_0)\setminus f(A_1) = f(A_0 \setminus A_1) \right]\label{note:equiv:3}.
	\end{eqnarray}
	We have already proved
	\( \left[ f \in \inj{A}{B} \Rightarrow (\ref{note:equiv:2})\right] \)
	and
	\( \left[ f \in \inj{A}{B} \Rightarrow (\ref{note:equiv:3})\right] \).
	So, we establish each of the converse.
	Let
	\( a_0,a_1 \in A \)
	with
	\( f(a_0) = f(a_1) \).
	If we assume (\ref{note:equiv:2}),
	then
	\( f(\left\{ a_0 \right\} \cap \left\{ a_1 \right\}) \neq \emptyset \)
	and so
	\( \left\{ a_0 \right\} = \left\{ a_1 \right\} \);
	if we assume (\ref{note:equiv:3}),
	then
	\( f(\left\{ a_0 \right\} \setminus \left\{ a_1 \right\}) = \emptyset \)
	and so
	\( \left\{ a_0 \right\} \setminus \left\{ a_1 \right\} = \emptyset \),
	which means
	\( \left\{ a_0 \right\} = \left\{ a_1 \right\} \).
	\qed\end{rem}

\begin{exe}
	Show that (b), (c), (f), and (g) of Exercise 2 hold for arbitrary unions and intersections.
\end{exe}\begin{sol}
	Let
	\( f \in \func{A'}{B'} \),
	and let
	\( A \)
	be a subset of
	\( A' \)
	for all
	\( A \in \mathcal{A} \),
	and 
	\( B\)
	be a subset of
	\( B' \)
	for all
	\( B \in \mathcal{B} \).
	
	\fbox{(b)}
	Observe that
	\( \bigcup_{B \in \mathcal{B}} B \supset B\)
	for all 
	\( B \in \mathcal{B} \),
	and that, by Exercise 2(a),
	\(f^{-1} \left( \bigcup_{B \in \mathcal{B}} B \right)
	\supset f^{-1} \left( B \right)\)
	for all 
	\( B \in \mathcal{B} \),
	which yields
	\(f^{-1} \left( \bigcup_{B \in \mathcal{B}} B \right)
	\supset 
	\bigcup_{B \in \mathcal{B}} f^{-1} \left( B \right)\).
	On the other hand, we have, in general, 
	\begin{equation*}
		\Forall{a\in
			f^{-1} \left( \bigcup_{B \in \mathcal{B}} B \right)
		}
		\Exists{B_0 \in \mathcal{B}}
		\left[ f(a)\in B_0 \right],
	\end{equation*}
	from which it follows that
	\(
	a \in f^{-1} \left( f(a) \right)
	\subset
	f ^{-1}(B_0)
	\subset
	\bigcup_{B \in \mathcal{B}} f ^{-1} \left( B \right).
	\)
	Thus, the result follows.
	
	\fbox{(c)}
	As we have done in (b), we deduce
	\(f^{-1} \left( \bigcap_{B \in \mathcal{B}} B \right)
	\subset 
	\bigcap_{B \in \mathcal{B}} f^{-1} \left(  B \right)
	\).
	We show the opposite inclusion.
	Let
	\(a \in \bigcap_{B \in \mathcal{B}} f^{-1} \left(  B \right)\).
	Definition of intersection gives
	\(a \in f^{-1} \left(  B \right)\)
	for all
	\(B \in \mathcal{B}\),
	or equivalently
	\(f(a) \in B\)
	for all
	\(B \in \mathcal{B}\).
	This means
	\( f(a) \in \bigcap_{B \in \mathcal{B}} B \),
	and hence
	\( a \in f^{-1} \left( \bigcap_{B \in \mathcal{B}} B \right) \).
	Thus,
	\(
	\bigcap_{B \in \mathcal{B}} f^{-1} \left(  B \right)
	\subset 
	f^{-1} \left( \bigcap_{B \in \mathcal{B}} B \right)
	\).
	
	\fbox{(f)}
	As before, we deduce that 
	\(f \left( \bigcup_{A \in \mathcal{A}} A \right)
	\supset 
	\bigcup_{A \in \mathcal{A}} f \left(  A \right)
	\).
	
	On the other hand,
	we see, in general, that  
	\begin{equation*}
		\Forall{b \in
			f \left(\bigcup_{A \in \mathcal{A}} A \right)
		}
		\Exists{ A_0 \in \mathcal{A} }
		\Exists{ a \in A_0 }
		\left[ b = f(a) \right],
	\end{equation*}
	and so
	\(b = f(a) \in f(A_0) \subset \bigcup_{A \in \mathcal{A}} f \left(  A \right)\).
	
	\fbox{(g)}
	The fact
	\(\bigcap_{A \in \mathcal{A}} A \subset A\)
	for all
	\(A \in \mathcal{A}\)
	implies 
	\(f \left( \bigcap_{A \in \mathcal{A}} A \right)
	\subset 
	f \left(  A \right)
	\).
	The same argument as Exercise 2 shows the equality holds
	if \( f \) is injective.
	\qed\end{sol}

\begin{exe}
	Let \( f:A \to B \)
	and \( g:B \to C \).
	\begin{enumerate}
		\item
		      If \( C_0 \subset C \),
		      show that
		      \( (g \circ f)^{-1}(C_0) = f ^{-1} \left( g ^{-1}(C_0) \right) \).
		      
		\item
		      If \( f \) and \( g \) are injective, show that \( g \circ f \) is injective.
		      
		\item
		      If \( g \circ f \) is injective, what can you say about injectivity of \( f \) and \( g \).
		      
		\item
		      If \( f \) and \( g \) are surjective, show that \( g \circ f \) is surjective.
		      
		\item
		      If \( g \circ f \) is surjective, what can you say about surjectivity of \( f \) and \( g \).
		      
		\item
		      Summarize your answers to (b)--(e) in the form of a theorem.
	\end{enumerate}
\end{exe}\begin{sol}
	Let
	\( f \in \func{A}{B} \)
	and
	\( g \in \func{B}{C} \).
	
	\fbox{(a)}
	Observe that 
	\begin{eqnarray*}
		(g \circ f)^{-1}(C_0)
		&=&	\left\{ a \in A \pipe
		(g \circ f)(a) \in C_0
		\right\}\\
		&=&	\left\{ a \in A \pipe
		f(a) \in g^{-1}(C_0)
		\right\}\\
		&=& f ^{-1} \left( g ^{-1}(C_0) \right).
	\end{eqnarray*}
	
	\fbox{(b)}
	Suppose
	\( f \)
	and
	\( g \)
	are injective.
	
	For any
	\( a,a' \in A \)
	with
	\( a \neq a' \),
	injectivity gives
	\( f(a)\neq f(a') \),
	and
	\( g(f(a)) \neq g(f(a')) \).
	Hence,
	\( g \circ f \in \inj{A}{C}\).
	
	\fbox{(c)}
	Suppose
	\( g \circ f \in \inj{A}{C}\).
	We claim in this case that
	\( f \) 
	is injective.
	For \( a,a' \in A \)
	with
	\( f(a) = f(a') \),
	we have
	\( g \circ f(a) = g \circ f(a') \),
	which means, by assumption, 
	\( a=a' \).
	Thus,
	\( f \)
	is injective.
	
	\fbox{(d)}
	Suppose
	\( f \)
	and
	\( g \)
	are surjective.
	In general, (a) implies that for every
	\( c \in C \),
	we have
	\( \left( g \circ f \right)^{-1}(c)
	= \left( f^{-1} \circ g^{-1}(c) \right)
	= \left( f^{-1}\left( g^{-1}(c) \right) \right)
	\).
	Assumption then yields
	\( g^{-1}(c) \neq \emptyset \)
	and
	\( f^{-1}\left( g^{-1}(c) \right) \neq \emptyset \).
	Thus, 
	\( g \circ f \)
	is surjective.
	
	\fbox{(e)}
	Suppose \( g \circ f \in \surj{A}{C}\).
	We insist that \( g \) is surjective.
	For every
	\( c \in C \),
	assumption implies that 
	\( \left( g \circ f \right)^{-1}(c) = f^{-1}\left( g^{-1}(c) \right)\)
	is nonempty,
	which necessarily means that
	\( g^{-1}(c) \neq \emptyset \).
	Thus,
	\( g \) is surjective.
	
	\fbox{(f)}
	We summarize (a)-(e) to state the following:
	\begin{thm}
		Let
		\( f \in \func{A}{B} \)
		and
		\( g \in \func{B}{C} \).
		If both
		\( f \)
		and
		\( g \)
		are injective (or surjective),
		so is
		\( g \circ f \).
		Conversely,
		If
		\( g \circ f \)
		is  injective,
		so is
		\( f \);
		and if 
		\( g \circ f \)
		is  surjective,
		so is
		\( g \).
		\qed\end{thm}\end{sol}

\begin{exe}
	In general, let us denote the \textbf{\textit{identity function}}
	for a set \( C \) by \( i_C \).
	That is, define \( i_C:C\to C \) to be the function given by the rule
	\( i_C(x)=x \) for all \( x \in C \).
	Given \( f:A\to B \), we say that
	a function \( g:B \to A \) is a \textbf{\textit{left inverse}} for \( f \)
	if \( g \circ f = i_A \);
	and we say that \( h:B \to A \)  is a \textbf{\textit{right inverse}} for \( f \)
	if \( f \circ h = i_B \).
	\begin{enumerate}
		\item
		      Show that if \( f \) has a left inverse, \( f \) is injective;
		      and if \( f \) has a right inverse, \( f \) is surjective.
		      
		\item
		      Give an example of a function that has a a left inverse but no right inverse.
		      
		\item
		      Give an example of a function that has a a right inverse but no left inverse.
		      
		\item
		      Can a function have more than one left inverse? More than one right inverse?
		      
		\item
		      Show that if \( f \) has both a left inverse \( g \) and a right inverse \( h \),
		      then \( f \) is bijective and \( g=h=f ^{-1} \).
	\end{enumerate}
\end{exe}\begin{sol}\leavevmode \par
	\fbox{(a)}
	This is a direct  consequence of Exercise 4(c) and (e).
	
	\fbox{(b)}
	A function given by
	\begin{equation*}
		f:\mathbb{R}_{+} \ni x \mapsto \exp{x} \in \mathbb{R}_{+}
	\end{equation*}
	admits a left inverse
	\begin{equation*}
		\ell : \mathbb{R}_{+} \to \mathbb{R}_{+} :
		x \mapsto 	\begin{cases}
			1      & \mathrm{\colon\;} x \in (0,1) \\
			\log x & \mathrm{\colon\;} x > 1,
		\end{cases}
	\end{equation*}
	but no right inverse.
	
	\fbox{(c)}
	A function given by
	\begin{equation*}
		g:\mathbb{R} \to \left\{ 0,1 \right\}:
		x \mapsto \begin{cases}
			1 & \mathrm{\colon\;} x \in \mathbb{Q}    \\
			0 & \mathrm{\colon\;} x \notin \mathbb{Q}
		\end{cases}
	\end{equation*}
	has a right inverse
	\begin{equation*}
		r:\left\{ 0,1 \right\}\to \mathbb{R}:
		x \mapsto \begin{cases}
			1 & \mathrm{\colon\;} x =1  \\
			e & \mathrm{\colon\;} x =0,
		\end{cases}
	\end{equation*}
	but no left inverse.
	
	\fbox{(d)}
	Yes. For instance, setting
	\( \ell :=1/2 \) on \( [0,1) \) in (b),
	and
	\( r(0) := \pi \) in (c)
	does not spoil their inverse property.
	
	\fbox{(e)}
	Apply Lemma 2.1.
	\qed\end{sol}

\begin{exe}
	Let \( f:\mathbb{R} \to \mathbb{R} \) be the function \( f(x)=x^3 -x \).
	By restricting the domain and range of \( f \) appropriately,
	obtain from \( f \) a bijective function \( g \).
	Draw the graphs of \( g \) and \( g ^{-1} \).
	There are several possible choices for \( g \).)
\end{exe}\begin{sol}
	Consider a strictly increasing function given by
	\begin{equation*}
		g : (1,\infty) \ni x \mapsto x^3 -x \in \mathbb{R}_{+}.
	\end{equation*}
	\( g \)
	possesses a inverse
	\begin{equation*}
		g ^{-1}: \mathbb{R}_{+} \ni x \mapsto \frac{1}{x^3-x} \in (1,\infty).
	\end{equation*}
	The drawing part is left to readers.
	\qed\end{sol}
\end{document}