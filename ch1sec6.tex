\documentclass[a4paper,12pt]{article}
\usepackage{mystyle}
\usepackage{commands}
\mathtoolsset{showonlyrefs=true}
\begin{document}
\section{Finite Sets}
\setcounter{exe}{0}

\begin{rem}[Direct proof of Corollary 6.3]
	Corollary 6.3. is equivalent to the following statements:
	\begin{eqnarray*}
		&&
		\Exists{ n \in \mathbb{Z}_{+} }
		\left[ \bij{A}{S_{n+1}} \neq \emptyset \right]
		\Rightarrow
		\left[
			B \subsetneq A
			\Rightarrow
			\bij{A}{B} = \emptyset
			\right]\\
		&\equiv&
		\left[
			\Exists{ n \in \mathbb{Z}_{+} }
			\left[ \bij{A}{S_{n+1}} \neq \emptyset \right]
			\Rightarrow
			\left[
				\bij{A}{B} \neq \emptyset
				\Rightarrow
				\neg \left( B \subsetneq A \right)
				\right]
			\right]\\
		&\equiv&
		\left[
			\Exists{ n \in \mathbb{Z}_{+} }
			\left[ \bij{A}{S_{n+1}} \neq \emptyset \right]
			\Rightarrow
			\left[
				\bij{B}{S_{n+1}} \neq \emptyset
				\Rightarrow
				\neg \left( B \subsetneq A \right)
				\right]
			\right]\\
		&\equiv&
		\left[
			\Exists{ n \in \mathbb{Z}_{+} }
			\left[ \bij{A}{S_{n+1}} \neq \emptyset \right]
			\Rightarrow
			\left[
				B \subsetneq A
				\Rightarrow
				\bij{B}{S_{n+1}} = \emptyset
				\right]
			\right].
	\end{eqnarray*}
	The last statement is exactly what Theorem 6.2 insists.
	\qed\end{rem}

\begin{rem}[Direct proof of Corollary 6.5.]
	It suffices to show that
	\begin{equation*}
		\bij{A}{S_n} \neq \emptyset \wedge \bij{A}{S_m}\neq \emptyset
		\Rightarrow
		n=m.
	\end{equation*}
	Note that we have
	\begin{equation*}
		\bij{A}{S_n}\neq \emptyset \wedge \bij{A}{S_m}\neq \emptyset
		\equiv
		\bij{A}{S_n} \neq \emptyset \wedge \bij{S_n}{S_m}\neq \emptyset,
	\end{equation*}
	and that one of \( S_n \) and \( S_m \) is a subset of the other.
	But, \( \bij{S_n}{S_m}\neq \emptyset \) and contrapositive of Theorem 6.2 together implies it cannot be a proper subset,
	which gives
	\( S_n =S_m \).
	Thus, \( n=m \).
	
	(Strictly speaking, we prove
	%\begin{center}
	\centering{\textit{\( S_n =S_m \) implies \( n=m \)}}
	%\end{center}
	as follows:
	\( E \setminus S_n \) and \( E \setminus S_m \)
	coincide and both sets have smallest element \( n \) and \( m \) respectively.
	From the uniqueness of a smallest element,
	we conclude that \( n=m \). )
	\qed\end{rem}

\begin{rem}[Cardinality of the set of injective maps]\label{note:number_inj}
	Let
	\(A:= \left\{ a_1,a_2,\dots,a_m \right\}\)
	and
	\(B:= \left\{ b_1,b_2,\dots,b_n \right\}\),
	and suppose
	\( 1 \le m \le n \).
	We claim that cardinality of \( \inj{A}{B} \) is equal to
	\( m \cdot (m-1) \cdot \cdots \cdot(m-n+1) \).
	
	\( \inj{A}{B} \) is nonempty
	since identity function is injective.
	Let \( f \in \inj{A}{B} \).
	There are \( n \) possible choices for the value of \( f(a_1) \)
	since \( f(a_1) \) can be any of the element of \( \left\{ b_1,b_2,\dots,b_n \right\} \),
	and
	\( n-1 \) choices available for \( f(a_2) \)
	since it can be any of
	\( \left\{ b_1,b_2,\dots,b_n \right\} \setminus f(a_1) \).
	So, in general, there are \( n-(i-1) \) choices for \( f(a_i) \)
	since it could be any of
	\( \left\{ b_1,b_2,\dots,b_n \right\}
	\setminus
	\left\{ f(a_1) \cup f(a_2) \cup \dots \cup f(a_{i-1}) \right\}
	\).
	
	Now it is easy to verify the claim by induction.
	\qed\end{rem}

\begin{exe}\leavevmode \par
	\begin{enumerate}
		\item
		      Make a list of all injective maps
		      \begin{equation*}
			      f:\left\{ 1,2,3 \right\} \to \left\{ 1,2,3,4 \right\}.
		      \end{equation*}
		      Show that none is injective.
		      
		\item
		      How many injective maps
		      \begin{equation*}
			      f:\left\{ 1,\cdots,8 \right\} \to \left\{ 1,\cdots,10 \right\}
		      \end{equation*}
		      are there?
	\end{enumerate}
	
\end{exe}\begin{sol}\leavevmode \par
	\fbox{(a)}
	For simplicity,
	let us ONLY in this Exercise use the nortation
	\( h:\left\{ a,b \right\} \mapsto \left\{ b,a\right\}\)
	to mean
	\( h(a)=b \)
	and
	\( h(b)=a \).
	
	\refer{Note}{note:number_inj} tells us that there are
	\( 4 \cdot 3 = 12\) injective functions.
	The list of them is given by
	\begin{eqnarray*}
		f_{1}:\left\{1,2,3 \right\} &\mapsto& \left\{1,2,3  \right\}\\
		f_{2}:\left\{1,2,3 \right\} &\mapsto& \left\{ 1,2,4 \right\}\\
		f_{3}:\left\{1,2,3 \right\} &\mapsto& \left\{ 1,3,2 \right\}\\
		f_{4}:\left\{1,2,3 \right\} &\mapsto& \left\{ 1,3,4 \right\}\\
		f_{5}:\left\{1,2,3 \right\} &\mapsto& \left\{ 2,1,3 \right\}\\
		f_{6}:\left\{1,2,3 \right\} &\mapsto& \left\{ 2,1,4 \right\}\\
		f_{7}:\left\{1,2,3 \right\} &\mapsto& \left\{ 2,3,1 \right\}\\
		f_{8}:\left\{1,2,3 \right\} &\mapsto& \left\{ 2,3,4 \right\}\\
		f_{9}:\left\{1,2,3 \right\} &\mapsto& \left\{ 3,1,2 \right\}\\
		f_{10}:\left\{1,2,3 \right\} &\mapsto& \left\{ 3,1,4 \right\}\\
		f_{11}:\left\{1,2,3 \right\} &\mapsto& \left\{ 3,2,1 \right\}\\
		f_{12}:\left\{1,2,3 \right\} &\mapsto& \left\{ 3,2,4 \right\}.
	\end{eqnarray*}
	It is obvious that none of them is bijective.
	
	\fbox{(b)}
	\refer{Note}{note:number_inj} shows there are
	\( 10 \cdot 9 \cdot 8\cdot 7 \cdot 6 \cdot 5 \cdot 4\cdot 3 \)
	injective functions.
	\qed\end{sol}

\begin{exe}
	Show that if \( B \) is not finite and \( B \subset A \),
	then \( A \) is not finite.
\end{exe}\begin{sol}
	The statement is equivalent to the following:
	\begin{eqnarray*}
		&&
		\neg \Exists{ n\in \mathbb{Z}_{+} }
		\left[ \bij{B}{S_n} \neq \emptyset \right]
		\wedge
		B \subset A
		\Rightarrow
		\neg \Exists{ m \in \mathbb{Z}_{+} }
		\left[ \bij{A}{S_m} \neq \emptyset \right]\\
		&\equiv&
		B \subset A
		\Rightarrow
		\left[
			\neg \Exists{ n\in \mathbb{Z}_{+} }
			\left[ \bij{B}{S_n} \neq \emptyset \right]
			\Rightarrow
			\neg \Exists{ m \in \mathbb{Z}_{+} }
			\left[ \bij{A}{S_m} \neq \emptyset \right]
			\right]\\
		&\equiv&
		B \subset A
		\Rightarrow
		\left[
			\Exists{ m \in \mathbb{Z}_{+} }
			\left[ \bij{A}{S_m} \neq \emptyset \right]
			\Rightarrow
			\Exists{ n\in \mathbb{Z}_{+} }
			\left[ \bij{B}{S_n} \neq \emptyset \right]
			\right].
	\end{eqnarray*}
	The last one is what Corollary 6.6 justifies.
	\qed\end{sol}

\begin{exe}
	Let \( X \) be the set two-element set \( \left\{ 0,1 \right\} \).
	Find a bijective correspondence between \( X^{\omega} \)
	and a proper subset of itself.
\end{exe}\begin{sol}
	A function
	\begin{equation*}
		f:X^{\omega} \ni \left( x_{i} \right)_{i\in \mathbb{Z}_{+}}
		\mapsto
		(0,x_1,x_2,\cdots) \in \left\{ 0 \right\} \times X \times X \cdots
	\end{equation*}
	is bijective since it admits the inverse given by
	\begin{equation*}
		g:\left\{ 0 \right\} \times X \times X \cdots \ni (0,x_1,x_2,\cdots)
		\mapsto
		(x_1,x_2,\cdots) \in X^{\omega}.
	\end{equation*}
	\qed\end{sol}

\begin{exe}\leavevmode \par
	\begin{enumerate}
		\item
		      Show that \( A \) has a largest element.
		      
		\item
		      Show that \( A \) has the order type of a section of the positive integers.
	\end{enumerate}
\end{exe}\begin{sol}
	Let \( A \) be a nonempty finite simply ordered set.
	
	\fbox{(a)}
	Let \( X \) be a subset of \( \mathbb{Z}_{+} \) such that \( A \) has a largest  element if there holds \( \card{A} \in X \).
	It is clear that \( 1 \in X\).
	Assuming \( n \in X\), consider the case \( \card{A} = n+1 \).
	Let \( a \in A \).
	By assumption, there exists a largest element
	\( m \)
	of
	\( A \setminus \left\{ a \right\} \).
	Then, we have
	\( \max{A} = \max \left\{ a,m \right\} \),
	that is,
	\( A \)
	has a largest element,
	which implies
	\( n+1 \in X \).
	Thus, induction establishes our claim.
	
	\fbox{(b)}
	Let \( Y \) be a subset of \( \mathbb{Z}_{+} \)
	such that if
	\( \card{A} \in Y \),
	then there exists an order isomorphism from \( S_{n} \) to \( A \)
	for some \( n \in \mathbb{Z}_{+} \).
	Remember \( S_n \) is the section of \( n \).
	We prove that \( Y \) is inductive.
	Obviously we have \( 1 \in Y \).
	Suppose \( n \in Y \)
	and
	\( \card{A}=n+1 \).
	Let \( m \) be a largest element of \( A \),
	and let \( f:S_{n+1} \to A\setminus \left\{ m \right\}\)
	be an order isomorphism,
	whose existence is guaranteed by the assumption here.
	Then, define a function \( g:S_{n+2} \to A \) by setting
	\begin{equation*}
		g(x):=\begin{cases}
			f(x) & \;\mathrm{:}\; x\in S_{n+1} \\
			m    & \;\mathrm{:}\; x=n+1.
		\end{cases}
	\end{equation*}
	It turns out that \( g \) is an order isomorphism by construction.
	Thus, \( n+1 \in Y \).
	\qed\end{sol}

\begin{exe}
	If \( A\times B \) is finite,
	does it follow that \( A \) and \( B \) are finite?
\end{exe}\begin{sol}
	No. Observe \( \mathbb{Z}_{+} \times \emptyset =\emptyset \) as a counterexample.
	\qed\end{sol}

\begin{exe}\leavevmode \par
	\begin{enumerate}
		\item
		      Let \( A=\left\{ 1,\cdots,n \right\} \).
		      Show there is a bijection of \( \mathcal{P}(A) \)
		      with the cartesian product \( X^{n} \),
		      where \( X \) is a two-element set \( \left\{ 0,1 \right\} \).
		      
		\item
		      Show that if \( A \) is finite, then \( \mathcal{P}(A) \) is finite.
	\end{enumerate}
\end{exe}\begin{sol}\leavevmode \par
	\fbox{(a)}
	Firstly, for \( B \in \mathcal{P}(A) \),
	define a function \( x^B:A \to \left\{ 0,1 \right\}\),
	called \textit{a characteristic function of \( B \)}, by setting
	\begin{equation*}
		x^B_i:=\begin{cases}
			1 & \;\mathrm{:}\;i \in B             \\
			0 & \;\mathrm{:}\;i \in A\setminus B.
		\end{cases}
	\end{equation*}
	Then, each of the function \( f \) and \( g \) given by
	\begin{eqnarray*}
		&&f:\mathcal{P}(A) \ni B \mapsto (x^B_i)_{i \in A}\in X^{n},\\
		&&g:X^{n} \ni (x_i)_{i \in A}
		\mapsto
		\left\{ i\in A \pipe x_i=1 \right\} \in \mathcal{P}(A)
	\end{eqnarray*}
	is easily checked to be the inverse of the other.
	Thus, both are bijective.
	
	\fbox{(b)}
	Corollary 6.8 implies that if \( A \) is finite, then so is \( A^n \),
	which proves, by (a), \( \mathcal{P}(A) \) is also finite.
	
	\qed\end{sol}

\begin{exe}
	If \( A \) and \( B \) are finite, show that the set of all function
	\( f:A\to B \)
	is finite.
\end{exe}\begin{sol}
	Let \( A \) and \( B \) be finite sets.
	Note that, in general,
	\( \func{A}{B} \)
	and 
	\(B^{\card{A}} \)
	are in bijective correspondence.
	Thus, the claim immediately follows from Corollary 6.8.
	
	Or you might prove the result based on the fact
	\( \func{A}{B} \subset \mathcal{P}(A \times B) \).
	\qed\end{sol}
\end{document}